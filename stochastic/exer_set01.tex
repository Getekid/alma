% -*- mode: LaTeX; coding: utf-8 -*-
% Typeset with: XeLaTeX

\documentclass[a4paper,11pt]{article}
\usepackage{a4wide}

% Greek fonts
\RequirePackage{fontspec}
\defaultfontfeatures{Ligatures=TeX}
  % you may want to try: {Liberation Serif} or {Times New Roman}
\setmainfont{FreeSerif}
  % you may want to try: {Liberation Sans} or {Arial}
\setsansfont[Scale=MatchLowercase]{FreeSans}
  % you may want to try: {FreeMono} or {Courier New}
\setmonofont[Scale=MatchLowercase]{FreeMono}

\usepackage{amsmath}

\newcommand{\indeq}[1]{\stackrel{\text{#1}}{=}}
% Commands for wrapping properly common expressions.
\newcommand{\Exp}{\mathrm{Exp}}
\newcommand{\Expect}{{\rm I\kern-.3em E}}
\newcommand{\Var}{\mathrm{Var}}
\newcommand{\Cov}{\mathrm{Cov}}

% Main document
\begin{document}
\title{Στοχαστικές Ανελίξεις - 1ο πακέτο Ασκήσεων}
\author{Θωμάς Παππάς}
\date{}
\maketitle

\section*{Άσκηση 1}

\begin{flalign*}
  P(Z \leq z) &= 1-P(Z>z) &\\
    &= 1-P(min\{X_1+X_2,X_3\}>z)\\
    &= 1-P(X_1+X_2>z,X_3>z)\\
    &\indeq{ανεξ.} 1-P(X_1+X_2>z)P(X_3>z)
\end{flalign*}
Για να υπολογίσω το $P(X_1+X_2>z)$ θα δεσμεύσω ως προς $X_1$.
\begin{flalign*}
  P(X_1+X_2>z) &= \int_0^\infty P(X_1+X_2>z | X_1=x) \cdot P(X_1=x)dx &\\
    &= \int_0^\infty P(X_2+x \leq z)) f_{X_1}(x)dx\\
    &= \int_0^\infty (1-P(X_2 \leq z-x)) \lambda_1 e^{-\lambda_1x}dx\\
    &= \int_0^\infty e^{-\lambda_2(z-x)} \lambda_1 e^{-\lambda_1x}dx
    % We could calculate the integral
    % = \lambda_1 \int_0^\infty e^{-(\lambda_2z+(\lambda_1-\lambda_2)x)}dx\\
    %&= \lambda_1 e^{-\lambda_2z} \int_0^\infty e^{(\lambda_2-\lambda_1)x}dx
    % = \frac{\lambda_1}{\lambda_2-\lambda_1} e^{-\lambda_2z} \left[e^{(\lambda_2-\lambda_1)x}\right]_0^\infty dx\\
    %&= \frac{\lambda_1}{\lambda_2-\lambda_1} e^{-\lambda_2z}(0-1)
    % = \frac{\lambda_1}{\lambda_1-\lambda_2} e^{-\lambda_2z}
    % or use LST
     = e^{-\lambda_2z} \int_0^\infty e^{-\lambda_2x} \lambda_1 e^{-\lambda_1x}dx\\
    &= e^{-\lambda_2z} \widetilde{F}_{X_1}(-\lambda_2)
     = e^{-\lambda_2z} \frac{\lambda_1}{\lambda_1-\lambda_2}
\end{flalign*}
Επίσης
\[P(X_3>z) = 1-P(X_3 \leq z) = 1- (1-e^{-\lambda_3z}) = e^{-\lambda_3z}\]
και άρα
\[P(Z \leq z) = 1-\frac{\lambda_1}{\lambda_1-\lambda_2} e^{-(\lambda_2+\lambda_3)z}\]


\section*{Άσκηση 2}

Αν $Z$ τυχαία μεταβλητή που αναπαριστά το χρόνο ζωής της μηχανής τότε $Z \sim \Exp(\lambda)$ και $Z = X-Y$.
Για να βρω την $\Expect[X]$ θα δεσμεύσω ως προς $Z \leq T$

\begin{flalign*}
  \Expect[X] &= \Expect[X|Z \leq T] \cdot P(Z \leq T) + \Expect[X|Z>T] \cdot P(Z>T) &\\
    &= T (1-e^{-\lambda T}) + (T + \Expect[X]) e^{-\lambda T}\\
    &= T - T e^{-\lambda T} + T e^{-\lambda T} + \Expect[X] e^{-\lambda T}\\
  \Rightarrow \Expect[X] &= \frac{T}{1-e^{-\lambda T}}
\end{flalign*}

\begin{flushleft}
	Για την $\Expect[Y]$ έχουμε
\end{flushleft}
\[\Expect[Y] = \Expect[X-Z] = \Expect[X]-\Expect[Z] \Rightarrow \Expect[X] = \frac{T}{1-e^{-\lambda T}} - \frac{1}{\lambda}\]


\section*{Άσκηση 3}

$X_1$: χρόνος ζωής 1ης μηχανής από σύστημα $A$, $X_1 \sim \Exp(\lambda)$\\
$X_2$: χρόνος ζωής 2ης μηχανής από σύστημα $A$, $X_2 \sim \Exp(\lambda)$\\
$Y$: χρόνος ζωής 1ης μηχανής από σύστημα $B$, $Y \sim \Exp(\mu)$
\\[8pt]
Για να βρω το $P(\max\{X_1,X_2\}<Y)$ θα δεσμεύσω ως προς $Y$ και μετά θα χρησιμοποιήσω την ανεξαρτησία των $X_1,X_2$.
\begin{flalign*}
  P(\max\{X_1,X_2\}<Y) &= P(X_1<Y,X_2<Y) &\\
    &= \int_0^\infty P(X_1<Y,X_2<Y|Y=y) f_Y(y)dy\\
    &= \int_0^\infty P(X_1<y,X_2<y) f_Y(y)dy
     \indeq{ανεξ.} \int_0^\infty P(X_1<y \cdot P(X_2<y) f_Y(y)dy\\
    &= \int_0^\infty (1-e^{-\lambda y})(1-e^{-\lambda y}) \mu e^{-\mu y} dy
     = \int_0^\infty (1-e^{-\lambda y})^2 \mu e^{-\mu y} dy\\
    &= \int_0^\infty (1-2e^{-\lambda y}+e^{-2\lambda y}) \mu e^{-\mu y} dy\\
    &= \int_0^\infty \mu e^{-\mu y} dy - 2\int_0^\infty e^{-\lambda y} \mu e^{-\mu y} dy + \int_0^\infty e^{-2\lambda y} \mu e^{-\mu y} dy\\
    &= 1-2\widetilde{F}_Y(\lambda)+\widetilde{F}_Y(2\lambda)\\
  \Rightarrow P(\max\{X_1,X_2\}<Y) &=1-2\frac{\mu}{\mu+\lambda}+\frac{\mu}{\mu+2\lambda}
\end{flalign*}


\section*{Άσκηση 4}

Για αρχή έχουμε
\[P(N(t)=k|N(t+s)=k+m) = \frac{P(N(t)=k,N(t+s)=k+m)}{P(N(t+s)=k+m)}\]
οπότε τώρα παίρνουμε
\begin{flalign*}
  P(N(t)=k,N(t+s)=k+m) &= P(N(t)=k,N(t+s)-k=m) &\\
    &= P(N(t)=k,N(t+s)-N(t)=m)\\
    &= P(N(t)=k,N(s)=m)\\
    &\indeq{ανεξ. προσ.} P(N(t)=k)) \cdot P(N(s)=m)
\end{flalign*}
άρα

\begin{flalign*}
  P(N(t)=k|N(t+s)=k+m) &= \frac{P(N(t)=k) \cdot P(N(s)=m)}{P(N(t+s)=k+m} &\\
    &= \frac{e^{-\lambda t}\frac{(\lambda t)^k}{k!} e^{-\lambda s}\frac{(\lambda s)^m}{m!}}{e^{-\lambda (t+s)}\frac{(\lambda (t+s))^{(k+m)}}{(k+m)!}}\\
    &= \frac{(k+m)!}{k!m!} \frac{t^k s^m}{(t+s)^{k+m}}
\end{flalign*}


\section*{Άσκηση 5}

\paragraph{α)}
$A =$ ο επόμενος πελάτης που θα φτάσει να βρει όλους τους υπηρέτες απασχολημένους

$X$: χρόνος άφιξης του επόμενου πελάτη, $X \sim \Exp(\lambda)$

$Z_i$: χρόνος εξυπηρέτησης του i-οστού πελάτη, $Z_i \sim \Exp(\mu)$

$Y$: χρόνος μέχρι την 1η εξυπηρέτηση
\[Y = \min\{Z_1,Z_2,\dots,Z_s\} \sim \Exp(s\mu)\]
\[P(A) = P(X<Y) = \frac{\lambda}{s\mu+\lambda}\]

\paragraph{β)} $N =$ \# αφίξεων πριν την πρώτη εξυπηρέτηση
\begin{flalign*}
  P(N=n)&= ??? &\\
    &= \left(\frac{\lambda}{\lambda+s\mu}\right)^n \frac{s\mu}{\lambda+s\mu}
\end{flalign*}

\paragraph{γ)} $Y \sim \Exp(\mu)$

$Y_2$: χρόνος από την εξυπηρέτηση του 1ου μέχρι του 2ου πελάτη, δεδομένου ότι δεν έχει φτάσει άλλος, $Y_2 \sim \Exp((s-1)\mu)$
\begin{flalign*}
  P(X>Y+Y_2) &= P(X>Y+Y_2|X \geq Y) \cdot P(X \geq Y) + P(X>Y+Y_2|X<Y) \cdot P(X<Y) &\\
    &= P(X>Y+Y_2|X \geq Y) \cdot P(X \geq Y) + 0 \cdot P(X<Y)\\
    &= P(Y_2<X-Y|X>Y) \cdot P(X \geq Y)\\
    &= \frac{(s-1)\mu}{\lambda+(s-1)\mu} \cdot \frac{s\mu}{\lambda+s\mu}
\end{flalign*}


\section*{Άσκηση 6}

Για αρχή θεωρούμε ότι το $t$ μετράει ώρες, οπότε αφού $\Expect[N(1)] = 8$ και $N(t) \sim PP(\lambda)$ τότε
\[\Expect[N(1)]=8 \Rightarrow \lambda \cdot 1 = 8 \Rightarrow \lambda = 8\]

\paragraph{α)}
\[\Expect[N(8)] = \lambda \cdot 8 = 8 \cdot 8 = 64\]
\[\Var[N(8)] = \lambda \cdot 8 = 8 \cdot 8 = 64\]

\paragraph{β)}
\[P(N(1/4) > 4) = 1-P(N(1/4) \leq 4) = 1 - \sum_{i=0}^4 e^{-\frac14 \cdot 8} \frac{\frac14 \cdot 8}{i!} = 1 - \sum_{i=0}^4 e^{-2} \frac{2}{i!}\]

\paragraph{γ)}
\[p=\frac{\Cov[N(11)-N(9),N(12)-N(10)]}{\sqrt{\Var[N(11)-N(9)]} \cdot \sqrt{\Var[N(12)-N(10)]}}\]
οπότε παίρνουμε
\begin{flalign*}
  \Cov&[N(11)-N(9),N(12)-N(10)] = &\\
    &= \Cov[N(11),N(12)] - \Cov[N(11),N(10)] - \Cov[N(9),N(12)] + \Cov[N(9),N(10)] &\\
    &= \Cov[N(11),N(12)-N(11)+N(11)] - \Cov[N(11)-N(10)+N(10),N(10)]\\
    &\;\; - \Cov[N(9),N(12)-N(9)+N(9)] + \Cov[N(9),N(10)-N(9)+N(9)]\\
    &= \Cov[N(11),N(11)] - \Cov[N(10),N(10)] - \Cov[N(9),N(9)] + \Cov[N(9),N(9)]\\
    &= \Var[N(11)] - \Var[N(10)] = 8 \cdot 11 - 8 \cdot 10 = 8
\end{flalign*}
\begin{flalign*}
  \Var[N(11)-N(9)] &= ??? &\\
\end{flalign*}
\begin{flalign*}
  \Var[N(12)-N(10)] &= ??? &\\
\end{flalign*}


\section*{Άσκηση 7}

Διαδικασία αφίξεων $PP(10)$ $\{N(t), t \geq 0\}$ $40\%$ γυναίκες, $60\%$ άντρες.\\
Έχουμε διαχωρισμό Bernouli με $p=0.4$.
\\[8pt]
Διαδικασία αφίξεων γυναικών $\{N_1(t), t \geq 0\}$ είναι $PP(10 \cdot p) = PP(4)$\\
Διαδικασία αφίξεων αντρών $\{N_2(t), t \geq 0\}$ είναι $PP(10 \cdot (1-p)) = PP(6)$\\
\[\{N_1(t), t \geq 0\},\{N_2(t), t \geq 0\} \text{ ανεξάρτητες}\]
άρα
\[\Expect[N_1(1)|N_2(1)=10] \indeq{ανεξ.} \Expect[N_1(1)]=4\]

\end{document}
