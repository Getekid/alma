% -*- mode: LaTeX; coding: utf-8 -*-
% Typeset with: XeLaTeX

\documentclass[a4paper,11pt]{article}
\usepackage{a4wide}

% Greek fonts
\RequirePackage{fontspec}
\defaultfontfeatures{Ligatures=TeX}
  % you may want to try: {Liberation Serif} or {Times New Roman}
\setmainfont{FreeSerif}
  % you may want to try: {Liberation Sans} or {Arial}
\setsansfont[Scale=MatchLowercase]{FreeSans}
  % you may want to try: {FreeMono} or {Courier New}
\setmonofont[Scale=MatchLowercase]{FreeMono}

\usepackage{amsmath}

\newcommand{\indeq}[1]{\stackrel{\text{#1}}{=}}
% Commands for wrapping properly common expressions.
\newcommand{\Exp}{\mathrm{Exp}}
\newcommand{\Expect}{{\rm I\kern-.3em E}}
\newcommand{\Var}{\mathrm{Var}}
\newcommand{\Cov}{\mathrm{Cov}}

% Main document
\begin{document}
\title{Στοχαστικές Ανελίξεις - 2ο πακέτο Ασκήσεων}
\author{Θωμάς Παππάς}
\date{}
\maketitle

\section*{Άσκηση 1}

Αρχικά έχουμε ότι $P(N(t)=k) = G_k(t)-G_{k+1}(t)$. Θα υπολογίσω το $\widetilde{G}(s)$ οπότε εφαρμόζω LST στην $G(t)$.
\begin{flalign*}
  \widetilde{G}(s) &= \int_0^\infty e^{-st} dG(t) = \int_0^\infty e^{-st} g(t) dt
     = \int_0^\infty e^{-st} \lambda^2 t e^{-\lambda t} dt = \int_0^\infty \lambda^2 t e^{-(\lambda+s)t} dt\\
    &= \frac{\lambda^2}{\lambda+s} \int_0^\infty t (\lambda+s) e^{-(\lambda+s)t} dt
     = \frac{\lambda^2}{\lambda+s} \cdot E[Y] \text{, όπου $Y \sim \Exp(\lambda+s)$} \\
    &= \frac{\lambda^2}{\lambda+s} \cdot \frac1{\lambda+s} = \left(\frac{\lambda}{\lambda+s}\right)^2
\end{flalign*}
οπότε και
\[\widetilde{G}_k(s) = \left[\widetilde{G}(s)\right]^k = \left(\frac{\lambda}{\lambda+s}\right)^{2k}\]
που σημαίνει ότι $S_k \sim \mathrm{Erlang}(2k,\lambda)$ και άρα $G_k(t) = 1 - \sum_{n=0}^{2k-1} e^{-\lambda t} \frac{(\lambda t)^n}{n!}$.
Οπότε εντέλει έχουμε
\begin{flalign*}
  P(N(t)=k) &= G_k(t)-G_{k+1}(t) = \left(1 - \sum_{n=0}^{2k-1} e^{-\lambda t} \frac{(\lambda t)^n}{n!}\right) - \left(1 - \sum_{n=0}^{2(k+1)-1} e^{-\lambda t} \frac{(\lambda t)^n}{n!}\right) &\\
    &= \sum_{n=0}^{2k+1} e^{-\lambda t} \frac{(\lambda t)^n}{n!} - \sum_{n=0}^{2k-1} e^{-\lambda t} \frac{(\lambda t)^n}{n!} = e^{-\lambda t} \frac{(\lambda t)^{2k}}{(2k)!} - e^{-\lambda t} \frac{(\lambda t)^{2k+1}}{(2k+1)!}
\end{flalign*}


\section*{Άσκηση 2}
Έχουμε
\[H(t) = \Expect\left[S_{N(t)+k}\right] = \int_0^\infty \Expect\left[S_{N(t)+k} | X_1=u\right] dG(u)\]
\begin{itemize}
	\item αν $u \leq t$ τότε $\Expect\left[S_{N(t)+k} | X_1=u\right] = u + \Expect\left[S_{N(t-u)+k}\right] = u + H(t-u)$
	\item αν $u > t$ τότε $\Expect\left[S_{N(t)+k} | X_1=u\right] = \Expect\left[S_k | X_1 = u\right] = \Expect\left[u+S_{k-1}\right] = u + \tau (k-1)$,\\
	όπου $\tau = \Expect[X]$
\end{itemize}
οπότε
\begin{flalign*}
  H(t) &= \int_0^t u + H(t-u) dG(u) + \int_t^\infty u + \tau (k-1) dG(u) &\\
    &= \int_0^t u dG(u) + \int_0^t H(t-u) dG(u) + \int_t^\infty u dG(u) + \tau (k-1) \int_t^\infty dG(u) \\
    &= \int_0^\infty u dG(u) + \int_0^t H(t-u) dG(u) + \tau (k-1) (1-G(t)) \\
    &= \tau + \tau (k-1) (1-G(t)) + (H*G)(t)
\end{flalign*}
Μετά παίρνω LST οπότε
\begin{flalign*}
  \widetilde{H}(s) &= \tau k - \tau (k-1) \widetilde{G}(s) + \widetilde{H}(s) \widetilde{G}(s) &\\
  \Rightarrow \widetilde{H}(s) &= \frac{\tau k - \tau (k-1) \widetilde{G}(s)}{1-\widetilde{G}(s)}
     = \tau k \frac1{1-\widetilde{G}(s)} - \tau (k-1) \frac{\widetilde{G}(s)}{1-\widetilde{G}(s)}\\
    &= \tau k \left(1 + \frac{\widetilde{G}(s)}{1-\widetilde{G}(s)}\right) - \tau (k-1) \frac{\widetilde{G}(s)}{1-\widetilde{G}(s)}\\
    &= \tau k (1 + \widetilde{M}(s)) - \tau (k-1) \widetilde{M}(s) = \tau k + \tau \widetilde{M}(s)
\end{flalign*}
και τέλος αντιστρέφοντας τον LST παίρνουμε
\[H(t) = \Expect\left[S_{N(t)+k}\right] = \tau (M(t) + k)\]

\section*{Άσκηση 3}

Όταν καταγράφουμε ένα γεγονός $X_{p,k}$ τότε οι επόμενοι χρόνοι $X_n, X_{p,n}$ είναι ανεξάρτητοι του παρελθόντος,
άρα η $\{N_p(t), t \geq 0\}$ είναι ανανεωτική διαδικασία.
\\[5pt]
Επίσης παρατηρούμε ότι αν το πρώτο γεγονός της $N_p(t)$ συμβεί μετά από $n$ γεγονότα της $N(t)$, τότε αφού η καταγραφή είναι ένα πείραμα Bernoulli παίρνουμε ότι $N_p(t) \sim \mathrm{Bin}(n,p)$.
Οπότε:
\begin{flalign*}
  \Expect[N_p(t)] &= \sum_{n=0}^\infty \Expect\left[N_p(t) | N(t)=n\right] \cdot P(N(t)=n)
     = \sum_{n=0}^\infty np \cdot P(N(t)=n) = p \sum_{n=0}^\infty n \cdot P(N(t)=n) &\\
    &= p \ \Expect[N(t)]
\end{flalign*}
και άρα
\[M_p(t) = \Expect[N_p(t)] = p \ \Expect[N(t)] = p \ M_(t)\]
\[\Rightarrow \widetilde{M}_p(s) = p \ \widetilde{M}(s) = p \ \frac{\widetilde{G}(s)}{1-\widetilde{G}(s)}\]


\section*{Άσκηση 4}

Έχουμε $H(t)) = P(A(t) \leq x)$. Χρησιμοποιώ ανανεωτικό επιχείρημα και δεσμεύω ως προς $S_1$:
\[H(t) = \int_0^\infty P(A(t) \leq x | S_1=u)dG(u)\]
\begin{itemize}
	\item αν $u \leq t$ τότε $P(A(t) \leq x | S_1=u) = P(A(t-u) \leq x) = H(t-u)$
	\item αν $u > t$ τότε $A(t) = t$ και άρα $P(A(t) \leq x | S_1=u) =
	  \begin{cases}
	    1, & t \leq x\\
	    0, & t > x
	  \end{cases} = 1_{\{t \leq x\}}$
\end{itemize}
άρα
\begin{flalign*}
  H(t) &= \int_0^t H(t-u) dG(u) + \int_t^\infty 1_{\{t \leq x\}} dG(u)
     = (H*G)(t) + 1_{\{t \leq x\}} \int_t^\infty dG(u) &\\
    &= 1_{\{t \leq x\}} (1-G(t)) + (H*G)(t)
\end{flalign*}
η οποία είναι ανανεωτική εξίσωση με $D(t) = 1_{\{t \leq x\}} (1-G(t))$.
\\[8pt]
Για να βρούμε τώρα το $\lim_{t\rightarrow\infty} H(t)$, βλέπουμε ότι $D(t)$ φθίνουσα, φραγμένη και μη αρνητική με
\[\int_0^\infty |D(t)| = \int_0^\infty 1_{\{t \leq x\}} (1-G(t)) dt = \int_0^x 1-G(t)dt + \int_t^\infty 0 dt \leq  \int_0^\infty 1-G(t)dt = \tau < \infty\]
όπου $\tau = \Expect[X_n]$, και άρα από το Β.Α.Θ. έχουμε
\[\lim_{t\rightarrow\infty} H(t) = \frac1\tau \int_0^\infty D(t)dt = \frac1\tau \int_0^x 1-G(t)dt\]


\section*{Άσκηση 5}

Έχουμε $H(t) = P(N(t) \text{περιττός})$. Χρησιμοποιώ ανανεωτικό επιχείρημα:
\[H(t) = \int_0^\infty P(N(t) \ \text{περιττός} | S_1=u)dG(u)\]
\begin{itemize}
	\item αν $u \leq t$ τότε
	  \[P(N(t) \ \text{περιττός} | S_1=u) = P(N(t-u) \ \text{άρτιος}) = 1-P(N(t-u) \ \text{περιττός}) = 1-H(t-u)\]
	\item αν $u > t$ τότε $P(N(t) \ \text{περιττός} | S_1=u) = 0$ αφού $N(t)=0$
\end{itemize}
άρα
\begin{flalign*}
  H(t) &= \int_0^t 1-H(t-u) dG(u) + \int_t^\infty 0 dG(u) = \int_0^t dG(u) - \int_0^t H(t-u) dG(u) &\\
    &= G(t) - (H*G)(t)
\end{flalign*}
η οποία ΔΕΝ είναι ανανεωτική εξίσωση.
\\[8pt]
Έστω τώρα ότι η $\{N(t), t \geq 0\}$ είναι $PP(\lambda)$, και οπότε έχουμε $\widetilde{G}(s) = \frac{\lambda}{s+\lambda}$. Για να λύσω λοιπόν την $H(t)$ θα εφαρμόσω LST:
\begin{flalign*}
  \widetilde{H}(s) &= \widetilde{G}(s) - \widetilde{H}(s) \widetilde{G}(s)
    \Rightarrow \widetilde{H}(s) + \widetilde{H}(s) \widetilde{G}(s) = \widetilde{G}(s)
    \Rightarrow \widetilde{H}(s)(1 + \widetilde{G}(s)) = \widetilde{G}(s) &\\
  \Rightarrow \widetilde{H}(s) &= \frac{\widetilde{G}(s)}{1+\widetilde{G}(s)}
    = \frac{\frac{\lambda}{s+\lambda}}{1+\frac{\lambda}{s+\lambda}}
    = \frac{\frac{\lambda}{s+\lambda}}{\frac{s+2\lambda}{s+\lambda}}
    = \frac{\lambda}{s+2\lambda} = \frac12 \cdot \frac{2\lambda}{s+2\lambda}
\end{flalign*}
Θεωρούμε τυχαία μεταβλητή $X \sim \Exp(2\lambda)$ για την οποία έχουμε $\widetilde{F}_X(s) = \frac{2\lambda}{s+2\lambda}$ και άρα
\[\widetilde{H}(s) = \frac12 \cdot \widetilde{F}_X(s)\]
και αντιστρέφοντας τον LST παίρνουμε
\[H(t) = \frac12 F_X(t) = \frac12 \left(1-e^{-2\lambda t}\right)\]


\section*{Άσκηση 6}

\paragraph{α)} $H(t) = \Expect[N^2(t)]$. Θα χρησιμοποιήσω ανανεωτικό επιχείρημα:
\[H(t) = \int_0^\infty \Expect\left[N^2(t)|S_1=u\right] dG(u)\]
\begin{itemize}
	\item αν $u \leq t$ τότε
	  \begin{flalign*}
	  	\Expect\left[N^2(t)|S_1=u\right] &= \Expect\left[(1+N(t-u))^2\right] = \Expect\left[1+2N(t-u)+N^2(t-u)\right] &\\
	  	  &= 1+2\Expect[N(t-u)]+\Expect\left[N^2(t-u)|S_1=u\right]\\
	  	  &= 1+2M(t-u)+H(t-u)
	  \end{flalign*}
	\item αν $u > t$ τότε $\Expect\left[N^2(t)|S_1=u\right] = 0$
\end{itemize}
οπότε έχουμε
\begin{flalign}
  H(t) &= \int_0^t 1+2M(t-u)+H(t-u) dG(u) + \int_t^\infty 0 dG(u) &\nonumber\\
    &= \int_0^t dG(u) + 2\int_0^t M(t-u) dG(u) + \int_0^t H(t-u) dG(u) \nonumber\\
    &= G(t) + 2 (M*G)(t) + (H*G)(t) \label{h1}
\end{flalign}
όμως $M(t) = G(t) + (M*G)(t)$ άρα
\[H(t) = G(t) + 2M(t) - 2G(t) + (H*G)(t) = 2M(t) - G(t) + (H*G)(t)\]
η οποία είναι ανανεωτική εξίσωση με $D(t) = 2M(t) - G(t)$

\paragraph{β)} παίρνω LST στην $H(t)$ στη μορφή της στην \eqref{h1}
\begin{flalign*}
  \widetilde{H}(s) &= \widetilde{G}(s) + 2 \widetilde{M}(s) \widetilde{G}(s) + \widetilde{H}(s) \widetilde{G}(s) &\\
  \Rightarrow \widetilde{H}(s) &= \frac{\widetilde{G}(s)}{1-\widetilde{G}(s)} (1 + 2 \widetilde{M}(s))
     = \widetilde{M}(s)(1 + 2 \widetilde{M}(s)) = \widetilde{M}(s) + 2 \widetilde{M}(s)\widetilde{M}(s)
\end{flalign*}
και αντιστρέφοντας τον LST παίρνουμε
\[H(t) = M(t) + 2 (M*M)(t) = M(t) + 2 \int_0^t M(t-u) dM(u)\]

\end{document}
