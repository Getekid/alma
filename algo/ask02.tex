% -*- mode: LaTeX; coding: utf-8 -*-
% Typeset with: XeLaTeX

\documentclass[a4paper,11pt]{article}
\usepackage{a4wide}
\usepackage{amsmath}

% Greek fonts
\RequirePackage[cm-default]{fontspec}
\defaultfontfeatures{Mapping=tex-text}
  % you may want to try: {FreeSerif} or {Times New Roman}
\setmainfont{Liberation Serif}
  % you may want to try: {FreeSans} or {Arial}
\setsansfont[Scale=MatchLowercase]{Liberation Sans}
  % you may want to try: {FreeMono} or {Courier New}
\setmonofont[Scale=MatchLowercase]{Liberation Mono}
\setlength{\parskip}{1em}

\newcommand\nlambda[1]{\ensuremath{\lambda #1.\,}}
\newcommand\nred{\ensuremath{\longrightarrow}}

% Main document
\begin{document}
\title{Αλγόριθμοι - 2η Σειρά Ασκήσεων}
\author{Θωμάς Παππάς}
\date{15 Δεκεμβρίου 2019}
\maketitle
\begin{center}ΑΜ: AL1180011\end{center}

\section*{Ασκηση 1: Μεταφορά Δεμάτων}
(α)

1. Η στοίβαξη με βάση το βάρος ΔΕΝ οδηγεί πάντα σε ασφαλή στοίβαξη.\\
Αντιπαράδειγμα: Έστω πως έχουμε τα πακέτα $(4, 8), (6, 2)$. Τότε ενώ υπάρξει ασφαλής στοίβαξη, η ${(4, 8), (6, 2)}$, αν τα στοιβάξουμε με βάση το βάρος τότε η στοίβαξη ${(6, 2) , (4, 8)}$ δεν είναι ασφαλής.

2. Η στοίβαξη με βάση την αντοχή ΔΕΝ οδηγεί πάντα σε ασφαλή στοίβαξη.\\
Αντιπαράδειγμα: Έστω πως έχουμε τα πακέτα $(3, 9), (10, 4)$. Τότε ενώ υπάρξει ασφαλής στοίβαξη, η ${(10, 4), (3, 9)}$, αν τα στοιβάξουμε με βάση το βάρος τότε η στοίβαξη ${(3, 9), (10, 4)}$ δεν είναι ασφαλής.

3. Η στοίβαξη με βάση το άθροισμα βάρους και αντοχής οδηγεί πάντα σε ασφαλή στοίβαξη. Για να το δείξουμε αυτό, θα δείξουμε ότι οποιαδήποτε ασφαλής σύνταξη είναι ισοδύναμη με μια η οποία έχει τα πακέτα στοιβαγμένα με βάση το άθροισμα βάρους και αντοχής (με το μεγαλύτερο στη βάση της στοίβας).

Έστω ότι για τα πακέτα $(w_1, d_1),...,(w_n, d_n)$ υπάρχει μια ασφαλής στοίβαξη $(1,...,n)$. Για λόγους απλότητας, έστω ${(w_1, d_1),..., (w_n, d_n)}$ αυτή η στοίβαξη. Έστω λοιπόν ότι για κάποιο $k$ στη στοίβαξη έχουμε $w_k + d_k \leq w_{k+1} + d_{k+1}$.

Ισχύει ότι $d_k \geq \sum_{i=k+1}^nw_i$ και άρα από την παραπάνω ανισότητα έχουμε
$$d_{k+1} \geq w_k + d_k - w_{k+1} \geq w_k + \sum_{i=k+1}^nw_i - w_{k+1} = w_k + \sum_{i=k+2}^nw_i$$
οπότε αν αλλάξουμε τη στοίβαξη σε
$$\{(w_1, d_1),...,(w_{k+1}, d_{k+1}), (w_k, d_k),...,(w_n, d_n)\}$$
τότε αυτή παραμένει ασφαλής αφού $d_{k+1} \geq w_k + \sum_{i=k+2}^nw_i$ και $d_k \geq \sum_{i=k+1}^nw_i \geq \sum_{i=k+2}^nw_i$.

Άρα αν επαναλάβουμε την παραπάνω διαδικασία σε όσες περιπτώσεις έχουμε $w_k + d_k \leq w_{k+1} + d_{k+1}$ μπορούμε να βρούμε μια ισοδύναμη ασφαλής στοίβαξη όπου $\forall k \leq n$ έχουμε $w_{k+1} + d_{k+1} \leq w_k + d_k$.

(β) Θα χρησιμοποιήσουμε DP. Ταξινομούμε τα πακέτα σύμφωνα με το άθροισμα βάρους και αντοχής. Έστω το εξής υποπρόβλημα $P$:
\begin{center}
  $P(i,w):$ το μέγιστο κόστος με $i$ πακέτα το πολύ βάρους $w$.
\end{center}
Άρα το πρόβλημά μας λύνεται ψάχνοντας το $P(n,W)$ όπου $W = \sum_{i=1}^nw_i$.

Παρατηρούμε τώρα ότι
$$P(i,w) =
  \begin{cases}
    0, & \text{$w < w_i$}\\
    max
    \begin{cases}
      P(i-1,w)\\
      p_i + P(i-1,min\{w-w_1, d_i\})
    \end{cases} & \text{$w \geq w_i$}
  \end{cases}$$
όπου το $min\{w-w_1, d_i\}$ μας εξασφαλίζει ότι το πακέτο $i$ θα μπορεί να στηρίξει τη στοίβα που δημιουργεί το συγκεκριμένο κόστος, αλλά και ότι το συνολικό βάρος δεν θα ξεπεράσει το $w$. Άρα φτιάχνοντας έναν πίνακα $W \times n$ και με τις αρχικές τιμές
$$P(1,w) =
  \begin{cases}
    0, & \text{$w < w_i$}\\
    p_i, & \text{$w \geq w_i$}
  \end{cases}$$
υπολογίζουμε τις τιμές του. Κρατάμε εν τω μεταξύ σε μια μεταβλητή την $max$ τιμή από όλα τα $P(i,w)$ και στο τέλος παίρνουμε την τιμή αυτή και το σημείο του πίνακα όπου βρίσκεται ώστε πηγαίνοντας προς τα πίσω να βρούμε όλα τα στοιχεία που αποτελούν αυτήν την ασφαλή στοίβαξη.

Πιο συγκεκριμένα, αν ξεκινήσουμε από το $P(n,W)$ τότε για κάθε $P(i,w)$ που "κοιτάμε", αν $P(i-1,w) < P(i,w)$ τότε το $i$ πακέτο ανήκει στη βέλτιστη λύση ενώ διαφορετικά προχωράμε στο στοιχείο $P(i-1,min\{w-w_1,d_i\})$.

Ο χρόνος εκτέλεσης του παραπάνω αλγορίθμου είναι $\mathcal{O}(nW)$.

\section*{Άσκηση 2: Αναμνηστικά}
Θα χρησιμοποιήσουμε DP. Έστω το εξής υποπρόβλημα $S$:
\begin{center}
  $P(i,c):$ Η μέγιστη συναισθηματική αξία που μπορούμε να συλλέξουμε πηγαίνοντας σε $i$ χώρες με κόστος το πολύ $c$.
\end{center}
Άρα το πρόβλημά μας λύνεται ψάχνοντας το $P(n,C)$.

Εφόσον έχουμε αρκετό προϋπολογισμό για το φθηνότερο αντικείμενο από κάθε χώρα μπορούμε να θεωρήσουμε το $C_{min}(i)$ ως εξής:
\begin{center}
  $C_{min}(i):$ το συνολικό κόστος των φθηνότερων αντικειμένων μέχρι και τη χώρα $i$
\end{center}

Θεωρούμε τώρα το $P_i(j)$ τη μέγιστη συναισθηματική αξία που συλλέγουμε από $i$ χώρες επιλέγοντας το $j$ αντικείμενο από τη χώρα $i$. Άρα
$$P_i(j) =
  \begin{cases}
    0, & \text{$C_{min}(i-1) > c-c_{ij}$}\\
    p_{ij} + P(i-1,c-c_{ij}), & \text{$C_{min}(i-1) \leq c-c_{ij}$}
  \end{cases}$$
και έχουμε λοιπόν ότι
$$P(i,c) = max\{P_i(j) | 1 \leq j \leq k_i\}$$
οπότε και φτιάχνουμε έναν $n \times C$ πίνακα (ξεκινάμε από το $C_{min}(n)$ για ευκολία) όπου θα κρατάει τη μέγιστη συναισθηματική αξία για $i$ χώρες με κόστος $c$ καθώς και το στοιχείο $j$ το οποίο χρησιμοποιήθηκε από τη χώρα $i$ για αυτήν την περίπτωση. Βάζουμε 0 όπου το $c$ δεν αρκεί για να πάρουμε ακόμα και τα φθηνότερα αναμνηστικά από κάθε χώρα.

Η αρχικοποίηση των τιμών θα γίνει θέτοντας $P(i,0) = 0, \forall i \leq n$.

Η απάντηση προκύπτει από το να πάρουμε το κελί με τη μέγιστη συναισθηματική αξία από το τέλος και πηγαίνοντας προς τα πίσω να πάρουμε τα στοιχεία $j$ που χρησιμοποιήθηκαν για κάθε περίπτωση.

Εφόσον ο πίνακας θα πρέπει να υπολογίσει τιμές για κάθε χώρα $i \leq n$ για όλα τα κόστη $c \leq C$ και να κάνει σε κάθε περίπτωση $k_i$ υπολογισμούς, ο χρόνος εκτέλεσης του αλγορίθμου θα είναι $\mathcal{O}(nCk_{max})$, όπου $k_{max} = max\{k_i | 1 \leq i \leq n\}$.

\section*{Άσκηση 3: Σοκολατάκια}
Θα χρησιμοποιήσουμε DP για να υπολογίσουμε το κόστος κάθε διαδρομής. Θεωρούμε 2 μετρητές $c_l, c_r$ για τα βήματα που κάνουμε από το αρχικό σημείο αριστερά και δεξιά αντίστοιχα. Για να φτάσουμε στο ζητούμενο θα πρέπει να υπολογίσουμε το χρόνο για όλες τις περιπτώσεις.

Στο σημείο $(i, j)$ μπορούμε να φτάσουμε με 2 περιπτώσεις, είτε από το το σημείο $(i-1, j)$ (από το δεξιό κουτί με κίνηση αριστερά) είτε από το $(i, j-1)$ (από το αριστερό κουτί με κίνηση δεξιά). Κρατάμε δύο πίνακες $n \times n$ με το χρόνο και την ποσότητα των σοκολάτων που έχουμε φάει. Για το χρόνο $C(i, j)$ έχουμε λοιπόν
$$C(i, j) = max
  \begin{cases}
    0, & \text{$q_{i-1} > q_i, q_{j-1} > q_j, t_i = t_{i-1}, t_j = t_{j-1}$}\\
    C(i-1, j)\\
    C(i, j-1)
  \end{cases}$$
με προφανώς αρχική τιμή $C(0,0) = 0$.

Για να γεμίσουμε τους παραπάνω πίνακες ο χρόνος εκτέλεσης που χρειαζόμαστε είναι $\mathcal{O}(n^2)$.
  
\end{document}
