% -*- mode: LaTeX; coding: utf-8 -*-
% Typeset with: XeLaTeX

\documentclass[a4paper,11pt]{article}
\usepackage{a4wide}
\usepackage{amsmath}

\usepackage{listings}% http://ctan.org/pkg/listings
\lstset{
  basicstyle=\ttfamily,
  mathescape
}

% Greek fonts
\RequirePackage[cm-default]{fontspec}
\defaultfontfeatures{Mapping=tex-text}
  % you may want to try: {FreeSerif} or {Times New Roman}
\setmainfont{Liberation Serif}
  % you may want to try: {FreeSans} or {Arial}
\setsansfont[Scale=MatchLowercase]{Liberation Sans}
  % you may want to try: {FreeMono} or {Courier New}
\setmonofont[Scale=MatchLowercase]{Liberation Mono}
\setlength{\parskip}{1em}

\newcommand\nlambda[1]{\ensuremath{\lambda #1.\,}}
\newcommand\nred{\ensuremath{\longrightarrow}}

% Main document
\begin{document}
\title{Αλγόριθμοι και Πολυπλοκότητα - 4η Σειρά Ασκήσεων}
\author{Θωμάς Παππάς}
\date{16 Φεβρουαρίου 2020}
\maketitle
\begin{center}ΑΜ: AL1180011\end{center}

\section*{Ασκηση 1: Επιβεβαίωση και Αναπροσαρμογή Συντομότερων Μονοπατιών}

\subsection*{(α)}
Δεδομένου ότι έχουμε έναν (ενδεχομένως λάθος) ισχυρισμό για όλα τα συντομότερα μονοπάτια από μια κορυφή $u_1$, μπορούμε να χρησιμοποιήσουμε αυτήν την πληροφορία για να υπολογίσουμε το Δέντρο Συντομότερων Μονοπατιών από το $u_1$ σε γραμμικό χρόνο.

Αν ο ισχυρισμός είναι λάθος, τότε θα υπάρχει ένα μονοπάτι $(u_1, u_i)$ τέτοιο ώστε $d(u_1, u_i) \neq \delta_i$, και προχωρώντας επαγωγικά στις κορυφές αυτού του μονοπατιού μπορούμε να βρούμε το μικρότερο $j$ για το οποίο $d(u_1, u_j) \neq \delta_j$. Παρατηρούμε ότι εφόσον μας αρκεί ένα λάθος για να ξέρουμε ότι ο αρχικός ισχυρισμός είναι εσφαλμένος, τότε αρκεί να ελέγξουμε μόνο την τελευταία ακμή ενός μονοπατιού, με το δεδομένο ότι η απόσταση του γονέα του με το $u_1$ είναι η ισχυριζόμενη. Τέλος μένει να κρατήσουμε σε έναν πίνακα ποιες κορυφές έχουν επιβεβαιωμένες τις αποστάσεις τους, ώστε στο τέλος θα μπορούμε να αποφανθούμε αν είναι όλες σωστές.

Ξεκινώντας λοιπόν φτιάχνουμε μια integer λίστα \verb|parents[]| $n$ μεγέθους όπου θα κρατάμε τους γονείς των κορυφών που έχουν επιβεβαιώσει την απόστασή τους $d(u_1, u_i) = \delta_i$. Αρχικοποιούμε τη λίστα βάζοντας παντού $-1$.

Από την κορυφή $u_1$ Θα τρέξουμε BFS (σε χρόνο $\mathcal{O}(|G|+|V|)$) και σε κάθε αναδρομή μιας κορυφής $u_p$ θα συγκρίνουμε για κάθε παιδί $u_c$ τις τιμές $\delta_c$ και $\delta_p + w(u_p,u_c)$ ως εξής:
\begin{itemize}
  \item Αν $\delta_c > \delta_p + w(u_p,u_c)$, τότε έχουμε βρει συντομότερο μονοπάτι από το ισχυριζόμενο, άρα τερμάτισε και επέστρεψε \verb|false|
  \item Αν $\delta_c = \delta_p + w(u_p,u_c)$, τότε σημειώνουμε \verb|parents[c] = p| και συνεχίζουμε
  \item Αν $\delta_c < \delta_p + w(u_p,u_c)$, τότε δεν κάνουμε τίποτα και συνεχίζουμε
\end{itemize}
Στο τέλος ελέγχουμε αν για τη λίστα \verb|parents| υπάρχει κάποιο στοιχείο $k$ τέτοιο ώστε \verb|parents[k] = -1|. Αν υπάρχει, τότε τερματίζουμε και επιστρέφουμε \verb|false|. Αν όχι τότε σημαίνει ότι όλες οι αποστάσεις $\delta_1, ..., \delta_n$ έχουν επιβεβαιωθεί ως σωστές και άρα επιστρέφουμε τον πίνακα \verb|parents| ο οποίος αποτελεί την περιγραφή του ΔΣΜ με ρίζα την κορυφή $u_1$.

Τα operations που τρέχουμε σε κάθε αναδρομή του BFS είναι χρόνου $\Theta(1)$ ενώ ο τελευταίος έλεγχος της λίστας \verb|parents| θα πάρει το πολύ $\mathcal{O}(n)$. Άρα συνολικά ο χρόνος που απαιτείται για τον παραπάνω αλγόριθμο είναι $\mathcal{O}(|G|+|E|)$.

\subsection*{(β)}
Παρατηρούμε ότι για να αναπροσαρμόσουμε τις ελάχιστες αποστάσεις μεταξύ όλων των κορυφών σε νέες τιμές $d^\prime(u_i,u_j)$, αρκεί να ελέγξουμε τα μονοπάτια στα οποία συμμετέχει η ακμή $e$. Αν η ακμη $e = (x,y)$ συμμετέχει σε ένα τέτοιο μονοπάτι (και εφόσον δεν έχουμε κύκλους αρνητικού μήκους) τότε αυτό θα έχει απόσταση
$$d_e^\prime(u_i,u_j) = d(u_i,x) + w^\prime(x,y) + d(y,u_j)$$

Επίσης από την τριγωνική ανισότητα ξέρουμε ότι για κάθε (καινούργιο) συντομότερο μονοπάτι ισχύει ότι
$$d(u_i,u_j) \leq d(u_i,x) + w(x,y) + d(y,u_j)$$
ενώ επίσης
$$d^\prime(u_i,u_j) \leq d(u_i,x) + w^\prime(x,y) + d(y,u_j)$$
με τις ισότητες να ισχύουν όταν η ακμή $e$ είναι μέρος του συντομότερου μονοπατιού. Προφανώς ισχύει ότι $d^\prime(u_i,u_j) \leq d(u_i,u_j)$.

Διακρίνουμε λοιπόν τις εξής περιπτώσεις:
\begin{itemize}
  \item Αν $d(u_i,u_j) < d_e^\prime(u_i,u_j)$ τότε το αρχικό συντομότερο μονοπάτι δεν χρησιμοποιούσε την ακμή $e$ και εφόσον παραμένει μικρότερο τότε παίρνουμε $d^\prime(u_i,u_j) = d(u_i,u_j)$
  \item Αν $d(u_i,u_j) \geq d_e^\prime(u_i,u_j)$ τότε το μονοπάτι με την ακμή $e$ έχει συντομότερη ή ίση απόσταση από πριν, και άρα παίρνουμε $d^\prime(u_i,u_j) = d_e^\prime(u_i,u_j)$
\end{itemize}
  
Οι πράξεις που κάνουμε για κάθε ζευγάρι απαιτούν χρόνο $\Theta(1)$ και εφόσον ελέγχουμε για κάθε κορυφή σε κάθε άλλη κορυφή, άρα ο συνολικός απαιτούμενος χρόνος είναι $\mathcal{O}(n^2)$.

\subsection*{(γ)}
Αν το μήκος μιας ακμής $e = (x,y)$ αυξηθεί σε $w^\prime(x,y) > w(x,y)$ τότε ΔΕΝ μπορούμε να επεκτείνουμε τον αλγόριθμο του (β) για να βρούμε τις νέες αποστάσεις. Ο λόγος είναι ότι δεν έχουμε κάποιο κάτω φράγμα για το $d^\prime(u_i,u_j)$ (όπως είχαμε πριν άνω φράγμα με την τριγωνική ανισότητα).

Πιο συγκεκριμένα, εφόσον η ακμή $e$ αυξήθηκε, τότε έχουμε πάντα
$d(u_i,u_j) < d_e^\prime(u_i,u_j)$ χωρίς να ξέρουμε σε ποια περίπτωση του αλγορίθμου βρισκόμαστε. Ακόμα και αν ξέραμε ότι η ακμή $e$ συμμετείχε στο μονοπάτι με $d(u_i,u_j)$ και πάλι δεν μπορούμε να γνωρίζουμε αν με την επαυξημένη ακμή θα υπάρχει άλλο μονοπάτι με $d(u_i,u_j) < d^\prime(u_i,u_j) < d_e^\prime(u_i,u_j)$, και άρα ο αλγόριθμος δεν μπορεί να δουλέψει.

\section*{Ασκηση 2: Σύστημα Ανισοτήτων}

Αρχικά παρατηρούμε ότι μπορούμε να δούμε το σύστημα ανισωτήτων $S$ ως έναν γράφο $G(V,E)$ με $|V| = n, |E| = m$, ο οπου για κορυφές τοποθετούμε τις $x_i$ μεταβλητές, ενώ από κάθε ανισότητα $x_i-x_j\leq b_{ij}$ βάζουμε την ακμή $(x_j, x_i)$ με βάρος $w(x_j, x_i) = b_{ij}$.

Επίσης παρατηρούμε ότι ένα μονοπάτι από $k$ κορυφές $(x_{i_1},...,x_{i_k})$ αντιστοιχεί στην ανισότητα που προκύπτει αν προσθέσουμε κατά μέλη τις $k-1$ ανισότητες που αντιστοιχούν στις ακμές του μονοπατιού
$$x_{i_2}-x_{i_1}\leq b_{i_2i_1}$$
$$x_{i_3}-x_{i_2}\leq b_{i_3i_2}$$
$$...$$
$$x_{i_k}-x_{i_{k-1}}\leq b_{i_ki_{k-1}}$$
από τις οποίες και παίρνουμε
$$x_{i_k}-x{i_1}\leq \sum_{j=1}^{k-1}b_{i_{j+1}i_j}$$
όπου $\sum_{j=1}^{k-1}b_{i_{j+1}i_j}$ είναι και το άθροισμα των ακμών του μονοπατιού.

\subsection*{(α)}
Χρησιμοποιώντας τον γράφο $G$ μπορούμε να δούμε το εξής: Το σύστημα S είναι ικανοποιήσιμο αν και μόνον αν ο γράφος $G$ δεν περιέχει κύκλο αρνητικού μήκους.

\subsubsection*{Απόδειξη}
Έστω ότι το σύστημα $S$ είναι ικανοποιήσιμο. Θα δείξουμε ότι ο γράφος $G$ δεν μπορεί να περιέχει κύκλο αρνητικού μήκους. Αν υποθέσουμε ότι έχει, τότε αυτό σημαίνει ότι υπάρχει κύκλος από $k$ κουρφές $(x_{i_1},...,x_{i_k},x_{i_1})$ για τον οποίο ισχύει ότι το άθροισμα των ακμών του είναι αρνητικό. Με την παρατήρηση για το άθροισμα των ακμών ενός μονοπατιού από προηγουμένως παίρνουμε $x_{i_1}-x{i_1} < 0 \Rightarrow 0 < 0$, το οποίο προφανώς και είναι αδύνατον. Άρα ο γράφος $G$ δεν περιέχει κύκλους αρνητικού μήκους.

Έστω τώρα ότι ο γράφος $G$ δεν περιέχει κύκλους αρνητικού μήκους. Θα δείξουμε ότι το σύστημα $S$ είναι ικανοποιήσιμο. Εφόσον o $G$ δεν περιέχει κύκλους αρνητικού μήκους, τότε μπορώ να βρω τα συντομότερα μονοπάτια από μια κορυφή $s$. Για να σιγουρευτούμε ότι θα φτάσουμε σε όλες τις μεταβλητές, προσθέτουμε μια κορυφή $x_0$ την οποία και συνδέουμε με όλες τις άλλες κορυφές με ακμή $(x_0,x_i)$ όπου $w(x_0,x_i) = 0$. Τότε μια λύση του συστήματος είναι η $(d(x_0,x_1),d(x_0,x_2),...,d(x_0,x_n))$.

Όντως εφόσον $d(x_0,x_i)$ το συντομότερο μονοπάτι από το $x_0$ στο $x_i$, τότε για κάθε ακμή $(x_i,x_j)$ παίρνουμε από την τριγωνική ανισότητα ότι
$$d(x_0,x_j) \leq d(x_0,x_i) + w(x_i,x_j) \Rightarrow x_j - x_i \leq b_{ji}$$
που σημαίνει ότι ο αντίστοιχος περιορισμός ισχύει.

\subsubsection*{Αλγόριθμος}
Αρχικά θα κατασκευάσουμε το γράφο $G$ σε χρόνο $\Theta(n+m)$. Για την εύρεση κύκλου αρνητικού μήκους θα τρέξουμε τον Αλγόριθμο Bellman-Fort για τον επαυξημένο γράφο $G(V\cup\{x_0\}, E\cup\{(x_0,x_i) | x_i \in E\})$. Όμως εφόσον θέτουμε $w(x_0,x_i)=0, \forall x_i\in E$ και γνωρίζουμε ότι η κορυφή $x_0$ δεν έχει εξερχόμενες ακμές (και άρα ούτε συμμετέχει σε κύκλο αρνητικού μήκους) άρα μπορούμε να αγνοήσουμε την επαυξημένη κορυφή και τις ακμές της.

Η πολυπλοκότητα του Bellman-Ford για γράφο με $n$ κορυφές και $m$ ακμές είναι $\Theta(nm)$, η οποία είναι και η συνολική πολυπλοκότητα του αλγορίθμου μας.

\subsection*{(β)}
\subsubsection*{Εύρεση τιμών}
Για την εύρεση τιμών ο προηγούμενος αλγόριθμος αρκεί να επιστρέφει τη λίστα με τα συντομότερα μονοπάτια όταν δεν βρίσκει κύκλο αρνητικού μήκους. Έτσι όπως δείξαμε τα συντομότερα μονοπάτια $(d(x_0,x_1),d(x_0,x_2),...,d(x_0,x_n))$ αποτελούν μια λύση του συστήματος $S$. Η πολυπλοκότητα του αλγορίθμου παραμένει στο $\Theta(nm)$.

\subsubsection*{Ελάχιστο μη ικανοποιήσιμο σύστημα}
Για να υπολογίσουμε το ελάχιστο μη ικανοποιήσιμο σύστημα $S^\prime$ θα πρέπει να βρούμε τον ελάχιστο κύκλο αρνητικού μήκους που καθιστά το $S$ μη ικανοποιήσιμο.

Φτιάχνουμε δύο λίστες $n$ μεγέθους \verb|degree[]| και \verb|pathSum[]|, αρχικοποιημένες με όλα τα στοιχεία τους ίσα με $0$). Επίσης φτιάχνουμε και μια λίστα από λίστες αγνώστου μεγέθους |\verb|negCyclePaths| όπου θα κρατάμε τους κύκλους αρνητικού μεγέθους που βρίσκουμε.
Ο έλεγχος για κύκλους αρνητικού μήκους \verb|D[u] > D[v] + w(v,u)| μας δίνει την πληροφορία ότι η κορυφή $v$ ανήκει σε τουλάχιστον έναν κύκλο αρνητικού μήκους.
Για να βρούμε τις ακμές του, θα τρέξουμε BFS από την $v$, και σε κάθε αναδρομή για την ακμή $(x_i, x_j)$ αυξάνουμε το βαθμό των κορυφών που βρίσκουμε κατά $1$ (\verb|degree[j]++|) και ενημερώνουμε τo άθροισμα του μονοπατιού (\verb|pathSum[j] = pathSum[i] + w(xi,xj)|).
Όταν φτάσουμε σε κορυφή $x_i$ που έχουμε ήδη επισκευθεί, τότε αρχικά ελέγχουμε αν το μονοπάτι είναι αρνητικού μήκους.
Αν όχι συνεχίζουμε, αν ναι τότε αν \verb|degr| είναι ο βαθμός της αναδρομής που βρισκόμαστε τότε ο κύκλος έχει μήκος \verb|degr-degree[i]|. Αποθηκεύουμε το μονοπάτι στο \verb|negCyclePaths| και συνεχίζουμε.

Αφού ελέγξουμε όλες τις κορυφές που ανήκουν σε κύκλο αρνητικού μήκους, επιστρέφουμε τους περιορισμούς που αντιστοιχούν στο μονοπάτι με το μικρότερο μήκος.

Για να βελτιώσουμε την πολυπλοκότητα του αλγορίθμου μας, μπορούμε να κρατήσουμε έναν global πίνακα \verb|visited| για όσα BFS χρειαστεί να τρέξουμε.
Έτσι, αν έχουμε πολλαπλούς κύκλους αρνητικού μήκους στο $G$ τότε αυτές οι κορυφές θα εξεταστούν μια φορά η καθεμιά.
Άρα στη χειρότερη περίπτωση όπου θα χρειαστεί να ελέγξουμε όλες τις κορυφές θα χρειαστούμε χρόνο $\Theta(n+m)$, και άρα ο συνολικός χρόνος για τον παραπάνω αλγόριθμο είναι $\Theta(nm+n+m)=\Theta(nm)$.

\subsubsection*{(γ)}
Τροποποιώντας το αλγόριθμο του υποερωτήματος (β), μπορούμε να προσθέσουμε τη λίστα $n$ μεγέθους \verb|weightSum| όπου θα αποθηκεύουμε το συνολικό βάρος ενός μονοπατιού.
Ενώ τρέχουμε το BFS λοιπόν σε κάθε βήμα για μια ακμή $(x_i, x_j)$ εκτελούμε \verb|weightSum[j] = weightSum[i] + w[i][j]| όπου \verb|w| ο $n\times n$ πίνακας με τα βάρη των ακμών.
Όταν από κορυφή $x_i$ βρίσκουμε κορυφή $x_j$ που έχουμε ήδη επισκευθεί τότε το συνολικό βάρος του μονοπατιού είναι \verb|weight[i]+w[i][j]-weight[j]|.
Στο τέλος επιστρέφουμε το μονοπάτι με το μικρότερο συνολικό βάρος.

Εφόσον απλά προσθέσαμε κάποιες $\Theta(1)$ πράξεις στον αλγόριθμο του (β), η πολυπλοκότητα του αλγορίθμου παραμένει στο $\Theta(nm)$.

\subsection*{(δ)}
Το πρόβλημα του υποερωτήματος (γ) διατυπώνεται ως πρόβλημα απόφασης $C(S,K)$ ως εξής:
Για το σύστημα ανισώσεων $S$ και έναν αριθμό $Κ$, υπάρχει υποσύστημα $S^\prime$ βάρους το πολύ $K$ τέτοιο ώστε το $S^\prime$ μη ικανοποιήσιμο;

Για να λύσουμε το πρόβλημα για ένα συγκεκριμένο αριθμό $K$, χρειάζόμαστε χρόνο $\Theta(nm)$, αλλά αφού $m\leq n(n-1)$ τότε ο χρόνος αυτός είναι το πολύ $\mathcal{O}(n^3)$. Άρα $C \in P$.

Για να λύσουμε το πρόβλημα γενικά, αρχικά το λύνουμε για το άθροισμα $W$ όλων των βαρών των περιορισμών, δηλαδή για $W = \sum_{1 \leq i,j \leq n} w_{ij}$ (θεωρούμε $w_{ij}=0$ αν δεν υπάρχει περιορισμός $x_i-x_j$).
Αν η απάντηση είναι \verb|false| (όπου σημαίνει ότι το $S$ είναι ικανοποιήσιμο) τότε επιστρέφουμε \verb|false|.
Διαφορετικά ψάχνουμε με δυαδική αναζήτηση αριθμό $w$ τέτοιο ώστε $P(S,w) = \verb|true|$ και $P(S,w-1) = \verb|false|$.
Αυτό σημαίνει ότι ο αλγόριθμος του (γ) θα τρέξει το πολύ $log(W)$ φορές, η οποία τιμή έχει ρυθμό ανάπτυξης μικρότερο του πολυωνυμικού, και άρα το πρόβλημα παραμένει στο $P$.

\section*{Άσκηση 3: Ταξιδεύοντας με Ηλεκτρικό Αυτοκίνητο}

\subsection*{(α)}
Εφόσον οι ακμές του δικτύου έχουν θετικά μήκη μπορούμε να βρούμε το συντομότερο $s-t$ μονοπάτι με τον αλγόριθμο του Dijkstra.
Για να σιγουρευτούμε όμως ότι θα περάσουμε από τουλάχιστον $1$ πόλη με σταθμό φόρτισης, για κάθε κορυφή $u$ θα κρατάμε το $D_0[u]$, το συντομότερο μονοπάτι από το $s$ στο $u$ με κανένα σταθμό φόρτισης, καθώς και το $D_1[u]$, το συντομότερο μονοπάτι από το $s$ στο $u$ με τουλάχιστον $1$ σταθμό φόρτισης.
Ομοίως και για τις λίστες $p_0$ και $p_1$ για να κρατάμε τους γονείς των κορυφών στα συντομότερα μονοπάτια.

Αρχικοποιούμε τις λίστες $D_0$ και $D_1$ με άπειρο σε όλα τα στοιχεία εκτός από το $D_0[s]$ το οποίο και αρχικοποιούμε με $0$.
Επίσης αρχικοποιούμε τις λίστες $p_0$ και $p_1$ με \verb|NULL| σε όλα τους τα στοιχεία.
Τρέχουμε τον αλγόριθμο Dijkstra. Για καθε ακμή $u$ που επισκέφθεται ο αλγόριθμος:
\begin{lstlisting}
for all v $\in$ AdjList[u] do
  if hasChargeStation(v) do
    minD = min($D_0$[u],$D_1$[u])
    if $D_1$[v] > minD + w(u,v) do
      $D_1$[v] = minD + w(u,v)
      $p_1$[v] = u
  else
    if $D_0$[v] > $D_0$[u] + w(u,v) do
      $D_0$[v] = $D_0$[u] + w(u,v)
      $p_0$[v] = u;
    if $D_1$[v] > $D_1$[u] + w(u,v) do
      $D_1$[v] = $D_1$[u] + w(u,v) 
      $p_1$[v] = u; 
\end{lstlisting}

Αφού ο αλγόριθμος ξεκινάει από το $s$ με $D_0[s]=0$ και $D_1[s]=\infty$ τότε για κάθε επόμενη κορυφή $v$ που δεν έχει σταθμό φόρτισης το $D_0[v]$ ανανεώνεται όπως και στην κλασική έκδοση του αλγορίθμου, και το $D_1[v]$ παραμένει άπειρο.
Όταν συναντήσουμε την πρώτη κορυφή με σταθμό φόρτισης, τότε ο αλγόριθμος ελέγχει αν το μονοπάτι από το $D_0$ ή το $D_1$ είναι το πιο σύντομο και επιλέγει αυτό για να υπολογίσει το $D_1[v]$, ενώ το $D_0[v]$ παραμένει άπειρο, αφού δεν υπάρχει μονοπάτι με $0$ σταθμούς φόρτισης που να περνάει από εκεί.
Απο εδώ και πέρα αν ξανασυναντήσουμε κορυφή χωρίς σταθμό φόρτισης τότε ο κάθε έλεγχος στα $D_0$ και $D_1$ μας διαβεβαιώνει ότι και η $u$ έχει μονοπάτι με $0$ ΄$1$ σταθμούς φόρτισης αντίστοιχα (ή και τα δύο) και ανανεώνει τους πίνακες αναλόγως.
Σε όλα τα παραπάνω βήματα οι πίνακες $p_0$ και $p_1$ συμπληρώνονται αντίστοιχα μόνο αν υπάρχει το αντίστοιχο μονοπάτι που περνάει από την ακμή $(u,v)$.

Το ελάχιστο μονοπάτι προκύπτει αν ξεκινήσουμε από την κορυφή $u$ που έχει την ελάχιστη τιμή στο $D_1$, η οποία ξέρουμε ότι δεν μπορεί να είναι άπειρο αφού $|C|\geq 3$, και πάρουμε αναδρομικά τους προγόνους της κορυφής από τις λίστες $p_0$ και $p_1$.
Ξεκινάμε με το $p_1$ της κορυφής και μετά για κάθε γονέα κορηφής $p$ ελέγχουμε αν $D_0[p] < D_1[p]$.
Αν είναι αληθής για συνέχεια του μονοπατιού παίρνουμε το $p_0[p]$ και συνεχίζουμε με την $p_0$ μέχρι τέλους.
Αν δεν είναι αληθής τότε παίρνουμε το $p_1[p]$ και στη συνέχεια επαναλαμβάνουμε τη διαδικασία (με τον έλεγχο) ξανά για αυτό το στοιχείο.

Εάν χρησιμοποιήσουμε Fibonacci heap για την εκτέλεση του Dijkstra τότε επιτυγχάνουμε χρόνο εκτέλεσης $\mathcal{O}(m+nlogn)$. Οι επιπλέον πράξεις που προσθέσαμε είναι χρόνου $\Theta(1)$ ενώ το τελευταίο κομμάτι που κατασκευάζει το μονοπάτι χρειάζεται χρόνο $\Theta(n)$, άρα ο συνολικός χρονος εκτέλεσης του αλγορίθμου μας είναι $\mathcal{O}(m+nlogn)$.

\end{document}
