% -*- mode: LaTeX; coding: utf-8 -*-
% Typeset with: XeLaTeX

\documentclass{beamer}
\usetheme{Madrid}
\centering

% Greek fonts
\RequirePackage[cm-default]{fontspec}
\defaultfontfeatures{Mapping=tex-text}
  % you may want to try: {FreeSerif} or {Times New Roman}
\setmainfont{Liberation Serif}
  % you may want to try: {FreeSans} or {Arial}
\setsansfont[Scale=MatchLowercase]{Liberation Sans}
  % you may want to try: {FreeMono} or {Courier New}
\setmonofont[Scale=MatchLowercase]{Liberation Mono}

% Main document
\begin{document}
\title{Algorithms in Git}
\author{Thomas Pappas}
%\date{4 March 2020}
\maketitle

% These three lines create an automatically generated table of contents.
\begin{frame}{Outline}
  \tableofcontents
\end{frame}

\begin{frame}
  \begin{block}{Disclaimer}
    This presentation contains a simplified, for introductory purposes, representation of how the Git framwork functions.
    The actual models of a Git repo, commit, branch etc. are more complex than described here.
    For more info check the references.
  \end{block}
\end{frame}

\section{What is Git?}

\begin{frame}{What is Git?}
  \begin{block}{Git is a decentralised VCS}
    \begin{itemize}
    \item Version Control System (VCS)
      \begin{itemize}
        \item tracks and keeps a history of changes in a directory tree of text files\\(code in most cases)
        \item facilitates collaborative development
        \item allows sharing code
        \item versioning
        \item etc.
      \end{itemize}
    \item Decentralised: Every user has a full copy of the repository (repo)
    \item Created by Linus Torvalds in 2005 for development of the Linux kernel
  \end{itemize}
  \end{block}
\end{frame}

\subsection{Git commits}

\begin{frame}{Git commits}
  \begin{block}{What are Commits?}
    Commits are the basic units of a Git repo.
    It contains information about
    \begin{itemize}
      \item Hash (for unique identification)
      \item For each file, two file states $a$ (before) and $b$ (after) to track changed
      \begin{itemize}
        \item Name
        \item Lines (in blobs)
        \item Meta-data
      \end{itemize}
      \item Commit message
      \item Meta-data (author, commiter, date)
      \item Parent(s) commit hash
    \end{itemize}
    A Git commit captures the full state of the file directory tree the moment the commit was created.
  \end{block}
\end{frame}

\begin{frame}{Git commits}
  \begin{block}{Diff blobs}
    When viewing a diff, a "blob" is the file section where the line changes are shown. For the two file state $a$ and $b$ the following are shown:
    \begin{itemize}
      \item line start and length (for the blob size)
      \item lines "removed" from $a$ to $b$ appended with "$-$"
      \item lines "added" from $a$ to $b$ appended with "$+$"
    \end{itemize}
  \end{block}
  \begin{block}{Commits as Nodes}
    We can thus see the Git repo as a graph with the commits as its nodes. A commit without a parent commit hash is called a "root" commit. 
  \end{block}
\end{frame}

% TODO: Add an example
%\begin{frame}{Git commit Example}

%\end{frame}

\subsection{Branches}

\begin{frame}{Git branches}
  \begin{block}{What are Branches?}
    Branches are "pointers" to commits. In practice, though, a branch is the subgraph that has all the ancestor commits of that commit.
  \end{block}
  \begin{block}{Some properties}
    \begin{itemize}
      \item One commit can belong to multiple branches
      \item The Git repo keeps a "current" branch in a HEAD variable
      \item When a commit is created, the branch's "tip" becomes that commit
      \item Benefit: Branches can facilitate independent feature development
    \end{itemize}
  \end{block}
\end{frame}

\section{Merging branches}

\begin{frame}{Merging branches}
  \begin{block}{What is a Merge?}
    Incorporates changes from the named commits (since the time their histories diverged from the current branch) into the current branch.
  \end{block}
  \begin{block}{(Common) Merge types}
    \begin{itemize}
      \item Fast Forward Merge
      \item 3-way Merge
    \end{itemize}
  \end{block}
  \begin{block}{Fast Forward Merge}
    Trivial case where the tip commit in the current branch is an ancestor of the named commit. Simply move the tip of the current branch to the named commit.
  \end{block}
\end{frame}

\begin{frame}{Merging branches}
  \begin{block}{3-way Merge: Algorithm}
    Given I currently am at "branchA" and I merge with "branchB", or in other words: Given the hashes of the two commits A and B on the branches' tips.
    \begin{itemize}
      \item Find a merge base C, the best common ancestor of commits A and B (LCA)
      \item Create a diff $D_A$ with the combined commits from C to A\\(and $D_B$ from C to B respectively)
      \item For each changed file, compare the blocks between the two diffs:
      \begin{itemize}
        \item If they apply changes to different blocks of the file, apply all changes
        \item If they apply changes to the same block and the change is the same, apply that change
        \item If they apply changes to the same block and the change is NOT the same, return a conflict
      \end{itemize}
    \end{itemize}
  \end{block}
\end{frame}

\section{Constructing the Git Commit Graph}

\begin{frame}{Constructing the Git Commit Graph}
  \begin{block}{Topological Order}
    In order to construct the Git Commit Graph we need to have the commits in an order when no commit will appear after its parent, ie. in topological order.
  \end{block}
\end{frame}
\begin{frame}[fragile]
  \frametitle{Constructing the Git Commit Graph}
  \begin{block}{Kahn's algorithm, Kahn, Arthur B. (1962)}
    \begin{verbatim}
    L ← Empty list that will contain the sorted elements
    S ← Set of all nodes with no incoming edge

    while S is non-empty do
        remove a node n from S
        add n to tail of L
        for each node m with an edge e from n to m do
            remove edge e from the graph
            if m has no other incoming edges then
                insert m into S

    if graph has edges then
        return error   (graph has at least one cycle)
    else
        return L   (a topologically sorted order)
    \end{verbatim}
  \end{block}
\end{frame}
\begin{frame}{Constructing the Git Commit Graph}
  \begin{block}{Topological Sorting in Git}
    The standard algorithm for topological sorting is Kahn’s Algorithm.
    \begin{itemize}
      \item We first initialise $S$ with the 1 node: the current branch tip commit
      \item We consider the graph with the edges inverted (since we want to start from a graph leaf)
    \end{itemize}
    The basic way this algorithm works is by walking the commit graph in two passes:
    \begin{enumerate}
      \item The first pass computes the “in-degree” of each commit: how many commits have this commit as a parent? (This in-degree only cares about commits reachable from our starting position.)
      \item Walk the commit graph by selecting a commit with in-degree zero and decrementing the in-degree of its parents.
    \end{enumerate}
  \end{block}
\end{frame}

\begin{frame}{Constructing the Git Commit Graph}
  \begin{block}{Generation Numbers}
    We define the generation number of a commit as follows:
    \begin{itemize}
      \item If a commit has no parents, then its generation number is 1
      \item Otherwise, the generation number is the maximum generation number of its parents $+1$
    \end{itemize}
    Can be easily implemented using $DFS$.\\
    Note that the generation numbers of a commit and its parent can have a big difference.
  \end{block}
  \begin{block}{Property}
    Suppose A and B are commits with generation numbers gen(A) and gen(B). If gen(A) < gen(B) then A cannot reach B.
  \end{block}
\end{frame}

\begin{frame}{Constructing the Git Commit Graph}
  \begin{block}{Generation Number usages}
    Improvements in algorithms for
    \begin{itemize}
      \item “Can X Reach Y?” Questions
      \item Merge-Base Questions
      \item Topological Sorting
    \end{itemize}
  \end{block}
\end{frame}

\begin{frame}{Constructing the Git Commit Graph}
  \begin{block}{Practical issue with the Topological Sorting}
    In practice, when viewing the branch history, we only need the last few commits.
    The problem is, we need to walk the entire graph in order to guarantee the in-degree of a commit is zero ($\mathcal{O}(m+n)$).
  \end{block}
  \begin{block}{Generation Numbers to the rescue}
    The trick is to keep two commit walks going at the same time.
    One walk is in charge of computing in-degrees while the other is for walking the commits in topological order. Then the following applies:
    \begin{itemize}
      \item For each visited commit, store its Generation Number (in a priority queue)
      \item The in-degree walk only needs to advance when the topological order walk considers a commit for the next selection
      \item The in-degree walk advances by selecting a commit of maximum generation number
    \end{itemize}
    This way we can detect in constant time if we need to advance that walk.
  \end{block}
\end{frame}

\section{References}

\begin{frame}{References}
  \begin{itemize}
    \item \href{https://www.oreilly.com/library/view/version-control-with/9781449345037/ch04.html}{Version Control with Git, 2nd Edition\\Chapter 4: Basic Git Concepts} (O'Reilly)
    \item \href{https://git-scm.com/docs/git-merge}{git-merge} (Git documentation)
    \item \href{https://git-scm.com/docs/git-merge-base}{git-merge-base} (Git documentation)
    \item \href{https://stackoverflow.com/questions/14961255/how-does-git-merge-work-in-details}{Stack Overflow question: How does 'git merge' work in details?}
    \item \href{https://www.atlassian.com/git/tutorials/using-branches/git-merge}{Atlassian Git Tutorial: Git Merge}
    \item \href{https://devblogs.microsoft.com/devops/supercharging-the-git-commit-graph-iii-generations/}{Supercharging the Git Commit Graph III: Generations and Graph Algorithms} (Azure DevOps Blog)
    \item Kahn, Arthur B. (1962), "Topological sorting of large networks", Communications of the ACM, 5 (11): 558–562, doi:\href{https://doi.org/10.1145\%2F368996.369025}{10.1145/368996.369025}.
  \end{itemize}
\end{frame}

\end{document}
