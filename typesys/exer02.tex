% -*- mode: LaTeX; coding: utf-8 -*-
% Typeset with: XeLaTeX

\documentclass[a4paper,11pt]{article}
\usepackage{a4wide}

% Greek fonts
\RequirePackage[cm-default]{fontspec}
\defaultfontfeatures{Mapping=tex-text}
  % you may want to try: {FreeSerif} or {Times New Roman}
\setmainfont{Liberation Serif}
  % you may want to try: {FreeSans} or {Arial}
\setsansfont[Scale=MatchLowercase]{Liberation Sans}
  % you may want to try: {FreeMono} or {Courier New}
\setmonofont[Scale=MatchLowercase]{Liberation Mono}

% More packages and macros
\usepackage[inference]{semantic}
\usepackage{listings}% http://ctan.org/pkg/listings
\lstset{
  basicstyle=\ttfamily,
  mathescape
}

\newcommand\nlambda[1]{\ensuremath{\lambda #1.\,}}
\newcommand\nred{\ensuremath{\longrightarrow}}

% Main document
\begin{document}
\title{2η Σειρά Ασκήσεων}
\author{Θωμάς Παππάς}
\date{12 Ιανουαρίου 2020}
\maketitle

\section{Επεκτάσεις}


\subsection*{Άσκηση 2.1  Ascription}

\subsubsection*{(1)}
Θέτουμε λοιπόν την derived form του \verb|Ascription| ως εξής: ένας όρος $t\verb| as |T$ μπορεί να γραφτεί ως $(\nlambda{x:T} x)t$.
Θα δείξουμε τώρα ότι οι κανόνες evaluation και typing στα δύο type systems είναι ισοδύναμα.
Προφανώς αν ο όρος $t$ δεν είναι τύπου $T$ τότε ο typechecker θα επιστρέψει error και στις δύο περιπτώσεις.

Εστω λοιπόν $\lambda^E, \lambda^I, e$ η εξωτερική γλώσσα (με το \verb|Ascription|), η εσωτερική γλώσσα (το simply typed $\lambda$-calculus) και την elaboration function που μετατρέπει τους ascription όρους στην derived form τους.
Παρατηρούμε ότι (επαγωγική υπόθεση)
\begin{itemize}
  \item Αν $t = v_1\verb| as |T$, όπου $v_1$ value τότε
    \begin{itemize}
      \item $t = v_1\verb| as |T \nred_E v_1$
      \item $e(t) = (\nlambda{x:T} x)v_1 \nred_I v_1$
    \end{itemize}
  \item Αν $t = t_1\verb| as |T$ με $t_1 \nred_E t_1'$ για κάποιους όρους $t_1,t_1'$ που ΔΕΝ περιέχουν ascription (άρα και $t_1 \nred_I t_1'$) τότε
    \begin{itemize}
      \item $t \nred_E t'$ όπου $t' = t_1'\verb| as |T$ (από τον \textsc{E-Ascribe1})
      \item $e(t) = (\nlambda{x:T} x)t_1 \nred_I (\nlambda{x:T} x)t_1' = e(t')$
    \end{itemize}
\end{itemize}
Και άρα για έναν τυχαίο τύπο $t$ μπορούμε με τα παραπάνω να κάνουμε επαγωγή στους υποτύπους του και να πάρουμε ότι $t \nred_E t'$ iff $e(t) \nred_I e(t')$.

Επίσης βλέπουμε ότι $\Gamma \vdash^E t:T$ ανν $\Gamma \vdash^I e(t):T$ αφού $\Gamma \vdash^I (\nlambda{x:T} x): T \rightarrow T$, και άρα το \verb|Ascription| είναι derived form.

\subsubsection*{(2)}
Με τον κανόνα \textsc{E-AscribeEager} το \verb|Ascription| ΔΕΝ μπορεί να είναι derived form από τον simply $\lambda$-calculus.
Αυτό διότι αν θέσουμε οποιοδήποτε όρο που να συμπεριλαμβάνει τον $t$ τότε θα πρέπει να κάνουμε πρώτα αυτόν evaluate με τον $t$ χωρίς να έχει γίνει ο ίδιος evaluate, το οποίο δεν υπάρχει σε κάποιο υπάρχον evaluation rule, και άρα θα έχουμε μια καινούργια συμπεριφορά στο type system μας.


\subsection*{Άσκηση 2.2  Αθροίσματα και λογικές τιμές}

Ορίζουμε τους όρους \verb|true|, \verb|false| και \verb|if| ως εξής:
\begin{itemize}
  \item $\verb|true| := \verb|inl unit as Unit+Unit|$
  \item $\verb|false| := \verb|inr unit as Unit+Unit|$
  \item $\verb|if | t_1 \verb| then | t_2 \verb| else | t_3 := \verb|case | t_1 \verb| as Unit+Unit of inl | x \Rightarrow t_2 | \verb|inr | x \Rightarrow t_3$
\end{itemize}
Όπου στην περίπτωση του \verb|case| θα πρέπει οι μεταβλητές να είναι ελεύθερες σε σχέση με τις μεταβλητές των $t_2$ και $t_3$, δηλαδή $x \notin FV(t_2), x \notin FV(t_3)$, το οποίο και πετυχαίνουμε με τα κατάλληλα renames στις μεταβλητές.


\section{Αναφορές}

\subsection*{Άσκηση 2.3 Τελεστής σταθερού σημείου}

Κοιτάζοντας τις ασκήσεις 13.1.2, σελ. 158, και 13.5.8, σελ. 169 του βιβλίου βλέπουμε ότι μπορούμε να λύσουμε την τελευταία ως εξής (ελεγμένη και στην OCaml):
\begin{lstlisting}
  factorial = $\nlambda{n:Nat}$
    let $f =$ ref $(\nlambda{m:Nat} 0)$ in
      $f$ := $(\nlambda{m:Nat}$ if iszero $m$ then $1$ else times $(!f \; ($pred $m)) \; m)$; $!f \; n$
\end{lstlisting}
Παρόλα αυτά αυτός ο όρος δεν μπορεί να γενικευθεί για κάθε τύπο $T_1 \rightarrow T_2$ διότι για τον $T_2$ θα πρέπει να δώσουμε μια τυχαία αρχική τιμή (πχ. στο παραπάνω βάλαμε το $0$ για να αρχικοποιήσουμε την $f:Ref \; Nat \rightarrow Nat$).
Θα μπορούσαμε να ορίσουμε πχ. την FixBool ως εξής:
\begin{lstlisting}
  fixBool = $\nlambda{f:(T_1 \rightarrow Bool) \rightarrow (T_1 \rightarrow Bool)} \nlambda{x:T_1}$
    let $r =$ ref $(\nlambda{y:T_1}$ true $)$ in
      $r$ := $(\nlambda{y:T_1} f \; !r \; y)$; $!r \; x$
\end{lstlisting}

\subsubsection*{Συνάρτηση even}

Με το παραπάνω η υλοποίηση της συνάρτησης \verb|even| (στην OCaml) θα είναι ως εξής:
\begin{verbatim}
/* Exer02-2.3 OCaml implementation. */

fixNatBool = lambda f:(Nat->Bool)->(Nat->Bool). lambda x:Nat.
  let r = ref (lambda y:Nat. true) in
    (lambda u:Unit. !r x) (r := (lambda y:Nat. f (!r) y));

evenr = lambda ie:Nat->Bool. lambda n:Nat.
  if iszero n then true
    else if iszero (pred n) then false
    else ie (pred (pred n));

even = fixNatBool evenr;
\end{verbatim}

\subsection*{Άσκηση 2.4  Μοναδικότητα τύπων μνήμης}

Παρατηρούμε ότι στην περίπτωση που έχουμε κύκλους τότε υπάρχουν παραπάνω από ένας τρόπος να έχουμε well typed όρους.
Οπότε θα πάρουμε ένα context $\Gamma$ στο οποίο υπάρχει ο όρος \\ $\Gamma \vdash t = \nlambda{x:Nat} (!l) \; x$ και το store $\mu$ όπου $\mu = (l \rightarrow t)$.
Τότε στη συνέχεια μπορούμε να πάρουμε τα εξής δύο stores $\Sigma_1, \Sigma_2$ στα οποία:
\[ \Sigma_1(l) = Nat \rightarrow Nat \]
\[ \Sigma_2(l) = Nat \rightarrow Bool \]
και για τα οποία βλέπουμε ότι $\Gamma \; | \; \Sigma_1 \vdash \mu$ και $\Gamma \; | \; \Sigma_2 \vdash \mu$ ταυτόχρονα.

\end{document}
