% -*- mode: LaTeX; coding: utf-8 -*-
% Typeset with: XeLaTeX

\documentclass[a4paper,11pt]{article}
\usepackage{a4wide}

% Greek fonts
\RequirePackage[cm-default]{fontspec}
\defaultfontfeatures{Mapping=tex-text}
  % you may want to try: {FreeSerif} or {Times New Roman}
\setmainfont{Liberation Serif}
  % you may want to try: {FreeSans} or {Arial}
\setsansfont[Scale=MatchLowercase]{Liberation Sans}
  % you may want to try: {FreeMono} or {Courier New}
\setmonofont[Scale=MatchLowercase]{Liberation Mono}

% More packages and macros
\usepackage[inference]{semantic}

\newcommand\nlambda[1]{\ensuremath{\lambda #1.\,}}
\newcommand\nred{\ensuremath{\longrightarrow}}

% Main document
\begin{document}
\title{1η Σειρά Ασκήσεων}
\author{Θωμάς Παππάς}
\date{28 Δεκεμβρίου 2019}
\maketitle

\section{Λάμβδα λογισμός χωρίς τύπους}


\subsection*{Άσκηση 1.1  Αναπαράσταση λιστών}

Αρχικά παρατηρούμε ότι μια λίστα $t$, πχ με $3$ στοιχεία $[x, y, z]$ είναι της μορφής
$$t = \nlambda{c} \nlambda{n} c x (c y (c z n))$$

\subsubsection*{nil}
Για την αναπαράσταση του \verb|nil| θα θέλουμε η συνάρτηση $c$ να μην εφαρμοστεί καθόλου οπότε παίρνουμε
$$\verb|nil| = \nlambda{c} \nlambda{n} n$$
το οποίο και είναι το ίδιο ``pun'' με το \verb|false| στα Church booleans και το $c_0$ στα Church numerals.

\subsubsection*{cons}
Για να προσθέσουμε ένα στοιχείο $h$ σε μια λίστα $t$, τότε αρκεί να καλέσουμε τη συνάρτηση $c$ της $t$ με τη λίστα $t$ και το στοιχείο $h$. Δηλαδή
$$\verb|cons| = \nlambda{t} \nlambda{h} \nlambda{c} \nlambda{n} c\ h\ (t\ c\ n)$$

\subsubsection*{isnil}
Για την υλοποίηση του \verb|isnil| θα χρησιμοποιήσουμε ένα ανάλογο κόλπο με αυτό του \verb|iszro| για τα Church numerals. Η συνάρτηση θα πρέπει να επιστρέφει \verb|true| όταν η συνάρτηση \verb|c| δεν τρέξει καθόλου, και \verb|false| όταν τρέξει έστω και μία φορά.
$$\verb|isnil| = \nlambda{t} t\ (\nlambda{x} \nlambda{y} \verb|false|)\ \verb|true|$$

\subsubsection*{head}
Εδώ για να πάρουμε το πρώτο στοιχείο του πίνακα, αρκεί να εφαρμόσουμε τη συνάρτηση $c= \verb|true|$ στη λίστα (με αδιάφορο το \verb|n|, εδώ θα βάλουμε τυχαία \verb|false|) και αυτή θα επιστρέψει το πρώτο της στοιχείο.
$$\verb|head| = \nlambda{t} t\ \verb|true|\ \verb|false|$$

\subsubsection*{tail}
Όπως συμβουλεύει και η υπόδειξη στην άσκηση, θα θεωρήσουμε μια συνάρτηση που θα φτιάχνει ζευγάρια (\verb|pair|) λιστών τα οποία θα αυξάνονται κατά ένα κάθε φορά (με κάθε κλήση της συνάρτησης στη λίστα), και στο τέλος θα πάρω το πρώτο στοιχείο του \verb|pair| όπου θα βρίσκεται η λίστα της προηγούμενης κλήσης της συνάρτησης. Δηλαδή έχουμε:
\[\verb|nn = pair nil nil|\]
\[\verb|cc| = \nlambda{x} \nlambda{p} \verb|pair (snd p) (cons x (snd p))|\]
και οπότε
\[\verb|tail| = \nlambda{t} \verb|fst (t cc nn)|\]


\subsection*{Άσκηση 1.2 Λίστες και σταθερά σημεία}

Για να φτιάξουμε τη συνάρτηση \verb|sumarise| που αθροίζει τα στοιχεία μιας λίστας \verb|t|, αρκεί να γράψουμε μια αναδρομική συνάρτηση που να το υπολογίζει και να χρησιμοποιήσουμε τη συνάρτηση \verb|fix| για να την υλοποιήσουμε. Οπότε αν
\[\verb|s| = \nlambda{sum} \nlambda{t} \verb|if isnil t then c|_0\verb| else plus (head t) (sum (tail t))|\]
τότε έχουμε
\[\verb|summary| = \verb|fix s|\]


\subsection*{Άσκηση 1.3 Διαφορετικές στρατηγικές αποτίμησης}
Οι κανόνες αποτίμησης του Figure 5-3 στη σελίδα 72 του βιβλίου του Pierce θα πρέπει να προσαρμοστούν ως εξής για την κάθε στρατηγική που θα επιλέξουμε:

\subsubsection*{full beta reduction}
Αρχικά θα κρατήσουμε τον \textsc{E-App1} θα αντικαταστήσουμε τους \textsc{E-App2} και \textsc{E-AppAbs} με πιο τους πιο γενικούς κανόνες
\[
  \inference[\textsc{E-App2-Gen}: ]{%
    t_2 \nred t'_2
  }{%
    t_1 t_2 \nred t_1 t'_2
  }
\]
\[
  \textsc{E-AppAbs-Gen}: (\nlambda{x} t_{12}) t_2 \nred [x \mapsto t_2] t_{12}
\]
και τέλος θα επιτρέψουμε τις αποτιμήσεις μέσα στα abstractions.
\[
  \inference[\textsc{E-App3}: ]{%
    t_1 \nred t'_1
  }{%
    \nlambda{x} t_1 \nred \nlambda{x} t'_1
  }
\]
Επίσης λόγω του \textsc{E-App3} θα πρέπει να περιορίσουμε τα values στα abstractions που επιστρέφουν όρους με είτε μόνο μεταβλητές είτε values είτε ένα συνδιασμό των δύο.
Αν θέσουμε ως $\verb|t|_{\verb|v|}$ έναν τέτοιο όρο τότε έχουμε
\[
  \verb|v| := \nlambda{x} \verb|t|_{\verb|v|}
\]
% Question: Το από πάνω δεν ορίζει αυστηρά το "είτε μόνο μεταβλητές είτε values είτε ένα συνδιασμό των δύο"; Μάλλον όχι...

\subsubsection*{normal-order}
Εδώ θα κρατήσουμε τους κανόνες \textsc{E-App1} και \textsc{E-App2} αλλά θα αντικαταστήσουμε τον \textsc{E-AppAbs} με τον ``αντίστροφο'' (σχετικά με το ποιος όρος πρέπει να είναι value και ποιος όχι)
\[
  \textsc{E-AppAbs-Rev}: (\nlambda{x} \verb|v|_{12}) t_2 \nred [x \mapsto t_2] \verb|v|_{12}
\]
και τέλος θα πάρουμε και τον \textsc{E-App3} και τον ορισμό των values από το προηγούμενο ερώτημα ώστε να επιτρέψουμε αποτιμήσεις μέσα στα abstractions.

\subsubsection*{lazy evaluation (call by name)}
Ομοίως με το προηγούμενο χωρίς τον \textsc{E-App3}, οπότε κρατάμε μόνο τους κανόνες \textsc{E-App1}, \textsc{E-App2} και \textsc{E-AppAbs-Rev} καθώς και τον αρχικό ορισμό για τα values (από το Figure 5-3).


\subsection*{Άσκηση 1.4 Μέγιστος κοινός διαιρέτης}

\begin{verbatim}
/* Exer01-1.4 OCaml implementation. */

fix = lambda f.(lambda x.f (lambda y.x x y)) (lambda x.f (lambda y.x x y));

/* Recurcive function for subtracting two numerals. */
subr = lambda sub.lambda m.lambda n.if iszero n then m else sub (pred m) (pred n);
sub = fix subr;

/* Recurcive function MOD of two numerals. */
modr = lambda mod.lambda m.lambda n.if iszero (sub m n) then
  (if iszero (sub n m) then 0 else m) else mod (sub m n) n;
mod = fix modr;

/* The GCD function! */
gcdr = lambda gcd.lambda m.lambda n.if iszero n then m else gcd n (mod m n);
gcd = fix gcdr;

/* Print the examples. */
gcd 24 42;
gcd 42 17;
gcd 714 630;
gcd 85 204;
\end{verbatim}

\pagebreak

\section{Απλοί τύποι}

\subsection*{Άσκηση 1.5 Subject expansion}

\subsubsection*{1. Γλώσσα αριθμητικών εκφράσεων}
Για τη γλώσσα των αριθμητικών εκφράσεων ΔΕΝ ισχύει το Subject expansion.
Αν θέσουμε
$$t = \verb|if true then 0 else false|$$
$$t' = \verb|0|$$
τότε βλέπουμε ότι $t \nred t'$ με $t':\verb|Bool|$ ενώ το $t$ δεν είναι well-typed.

\subsubsection*{2. $\lambda$-λογισμός με απλούς τύπους}
Αρχικά παρατηρούμε ότι αν ο $t$ είναι well-typed τότε παίρνουμε το ζητούμενο, αφού αν $\Gamma \vdash t:T'$ με $T' \neq T$ τότε από το Preservation theorem παίρνουμε ότι $\Gamma \vdash t':T'$ το οποίο είναι άτοπο.
Αρκεί λοιπόν να δείξουμε ότι όλοι οι όροι του συναρτισιακού $\lambda$-λογισμού είναι well-typed.

Προφανώς ο $t$ δεν γίνεται να είναι value εφόσον τότε δεν υπάρχει κάποιο $t'$ τέτοιο ώστε $t\nred t'$.
Εάν τώρα ο $t$ είναι κάποιος base term του $\Gamma$ (που δεν περιέχει conditional expressions) τότε είτε $t:\verb|Bool|$ είτε $t:\verb|Nat|$ και άρα είναι well-typed.
Αυτό διότι οποιοσδήποτε άλλος ill-typed όρος (πχ. \verb|succ false|) πάλι δεν μπορεί να γίνει evaluate σε κάποιο $t'$.

\vspace{5mm}

Για την περίπτωση που $t$ είναι application term της μορφής $t_1 t_2$, θα δείξουμε το ζητούμενο με επαγωγή.
Έστω λοιπόν ότι το Subject expansion ισχύει για κάθε υποόρο του $t$, δηλαδή και για τα $t_1,t_2$.

Αν για το evaluation χρησιμοποιήσαμε τον κανόνα \textsc{E-App1} τότε υπάρχει ένα $t_1'$ τέτοιο ώστε $t'=t_1' t_2$ και άρα από το Inversion lemma (3) παίρνουμε ότι υπάρχει τύπος $T_{11}$ τέτοιος ώστε\\$\Gamma \vdash t_1':T_{11} \rightarrow T$ και $\Gamma \vdash t_2:T_{11}$.
Από τον πρώτο κανόνα και την επαγωγική υπόθεση παίρνουμε ότι $\Gamma \vdash t_1:T_{11}$ και άρα (αφού $\Gamma \vdash t_2:T_{11}$ και χρησιμοποιώντας τον \textsc{T-App}) έχουμε $\Gamma \vdash t:T$.

Ομοίως για την περίπτωση που για το evaluation χρησιμοποιήσαμε τον κανόνα \textsc{E-App2}.

Έστω τώρα ότι για το evaluation χρησιμοποιήσαμε τον κανόνα \textsc{E-AppAbs}, δηλαδή $t_1 = \lambda x:T_{11}.t_{12}$ και $t_2$ είναι value με $\Gamma \vdash t_2:T_{11}$.
Τότε θα πρέπει $\Gamma \vdash t_{12}:T$ αφού έχουμε ότι $t' = [x \mapsto t_2]t_{12}$ και $\Gamma \vdash t':T$, και άρα αν ο $t_{12}$ έχει άλλο τύπο ή δεν είναι well-typed τότε φτάνουμε σε άτοπο. Οπότε έχουμε $\Gamma \vdash t_1:T_{11} \rightarrow T, t_2:T_{11}$ και από τον \textsc{T-App1} παίρνουμε $\Gamma \vdash t:T$.

\vspace{5mm}

Άρα σε κάθε περίπτωση το Subject expansion ισχύει.

\subsection*{Άσκηση 1.6 Κατάσταση σφάλματος}

\subsubsection*{Περιγραφή της γλώσσας}
Για τη γλώσσα που προκύπτει από το συνδυασμό της απλής αριθμητικής και του $\lambda$-λογισμού με απλούς τύπους, θα επεκτείνουμε τους ορισμούς στα Figures 8.2 και 9.1 του βιβλίου.

\pagebreak

\paragraph*{Σύνταξη}:
Επεκτείνουμε τους όρους (terms), values και Types αντίστοιχα ως εξής

\verb|t ::=|

\begin{tabular}{l r}
  \verb|wrong| & \textit{constant wrong}
\end{tabular}

\verb|v ::=|

\begin{tabular}{l r}
  \verb|wrong| & \textit{wrong value}
\end{tabular}

\verb|Τ ::=|

\begin{tabular}{l r}
  \verb|Wrong| & \textit{wrong type}
\end{tabular}

\paragraph*{Λειτουργική Σημασιολογία}:
\[
  \verb|if 0 then | t_1 \verb| else | t_2 \nred \verb|wrong|
\]
\[
  \verb|if (succ | t_1 \verb|) then | t_2 \verb| else | t_3 \nred \verb|wrong|
\]
\[
  \verb|if (pred | t_1 \verb|) then | t_2 \verb| else | t_3 \nred \verb|wrong|
\]
\[
  \verb|if (| \nlambda{x} t_1 \verb|) then | t_2 \verb| else | t_3 \nred \verb|wrong|
\]
\[
  \verb|if wrong then | t_1 \verb| else | t_2 \nred \verb|wrong|
\]
\[
  \verb|succ true| \nred \verb|wrong|, \;
  \verb|pred true| \nred \verb|wrong|, \;
  \verb|iszero true| \nred \verb|wrong|
\]
\[
  \verb|succ false| \nred \verb|wrong|, \;
  \verb|pred false| \nred \verb|wrong|, \;
  \verb|iszero false| \nred \verb|wrong|
\]
\[
  \verb|succ(| \nlambda{x} t \verb|)| \nred \verb|wrong|, \;
  \verb|pred(| \nlambda{x} t \verb|)| \nred \verb|wrong|, \;
  \verb|iszero(| \nlambda{x} t \verb|)| \nred \verb|wrong|
\]
\[
  \verb|succ wrong| \nred \verb|wrong|, \;
  \verb|pred wrong| \nred \verb|wrong|, \;
  \verb|iszero wrong| \nred \verb|wrong|
\]

\paragraph*{Σύστημα Τύπων}:
\[
  \inference{
    \Gamma \vdash t_1:\texttt{Nat} & \Gamma \vdash t_2:T_1 & \Gamma \vdash t_3:T_2
  }{
    \Gamma \vdash \texttt{if } t_1 \texttt{ then } t_2 \texttt{ else } t_3: \texttt{Wrong}
  }
\]
\[
  \inference{
    \Gamma \vdash t_1:T_{11} \rightarrow T_{12} & \Gamma \vdash t_2:T_1 & \Gamma \vdash t_3:T_2
  }{
    \Gamma \vdash \texttt{if } t_1 \texttt{ then } t_2 \texttt{ else } t_3: \texttt{Wrong}
  }
\]
\[
  \inference{
    \Gamma \vdash t_1:\texttt{Wrong} & \Gamma \vdash t_2:T_1 & \Gamma \vdash t_3:T_2
  }{
    \Gamma \vdash \texttt{if } t_1 \texttt{ then } t_2 \texttt{ else } t_3: \texttt{Wrong}
  }
\]
\[
  \inference{
    \Gamma \vdash t_1:\texttt{Bool}
  }{
    \Gamma \vdash \texttt{succ } t_1: \texttt{Wrong}
  },
  \inference{
    \Gamma \vdash t_1:\texttt{Bool}
  }{
    \Gamma \vdash \texttt{pred } t_1: \texttt{Wrong}
  },
  \inference{
    \Gamma \vdash t_1:\texttt{Bool}
  }{
    \Gamma \vdash \texttt{iszero } t_1: \texttt{Wrong}
  }
\]
\[
  \inference{
    \Gamma \vdash t_1:T_{11} \rightarrow T_{12}
  }{
    \Gamma \vdash \texttt{succ } t_1: \texttt{Wrong}
  },
  \inference{
    \Gamma \vdash t_1:T_{11} \rightarrow T_{12}
  }{
    \Gamma \vdash \texttt{pred } t_1: \texttt{Wrong}
  },
  \inference{
    \Gamma \vdash t_1:T_{11} \rightarrow T_{12}
  }{
    \Gamma \vdash \texttt{iszero } t_1: \texttt{Wrong}
  }
\]
\[
  \inference{
    \Gamma \vdash t_1:\texttt{Wrong}
  }{
    \Gamma \vdash \texttt{succ } t_1: \texttt{Wrong}
  },
  \inference{
    \Gamma \vdash t_1:\texttt{Wrong}
  }{
    \Gamma \vdash \texttt{pred } t_1: \texttt{Wrong}
  },
  \inference{
    \Gamma \vdash t_1:\texttt{Wrong}
  }{
    \Gamma \vdash \texttt{iszero } t_1: \texttt{Wrong}
  }
\]
\[
  \inference{
    \Gamma \vdash t_1:T_{11} \rightarrow T_{12} & \Gamma \vdash t_2:T_2, T_{11} \neq T_2
  }{
    \Gamma \vdash t_1 t_2: \texttt{Wrong}
  }
\]

\subsubsection*{Type Safety}
Με τους παραπάνω κανόνες καταλήγουμε σε ένα σύστημα όπου δεν υπάρχουν ``stuck'' όροι, και άρα η ιδιότητα \textit{Progress} δεν μας προσφέρει κάτι. Ο ορισμός του \textit{Preservation} τώρα μπορεί να προσαρμοστεί ως εξής
\begin{center}
 \textit{Preservation:} Ένας όρος που ΔΕΝ έχει type \verb|Wrong| και κάνει ένα evaluation βήμα, τότε και ο όρος που θα επιστραφεί επίσης ΔΕΝ θα έχει type \verb|Wrong|
\end{center}

\end{document}
