% -*- mode: LaTeX; coding: utf-8 -*-
% Typeset with: XeLaTeX

\documentclass{beamer}
\usetheme{Madrid}
\centering

% Greek fonts
\RequirePackage[cm-default]{fontspec}
\defaultfontfeatures{Mapping=tex-text}
  % you may want to try: {FreeSerif} or {Times New Roman}
\setmainfont{Liberation Serif}
  % you may want to try: {FreeSans} or {Arial}
\setsansfont[Scale=MatchLowercase]{Liberation Sans}
  % you may want to try: {FreeMono} or {Courier New}
\setmonofont[Scale=MatchLowercase]{Liberation Mono}


% Main document
\begin{document}
\title{\#3SAT $\in$ IP (Proof sketch)}
\author{Thomas Pappas}
%\date{4 March 2020}
\maketitle

% These three lines create an automatically generated table of contents.
\begin{frame}{Outline}
  \tableofcontents
\end{frame}

\section{Constructing the boolean polynomial}

\begin{frame}{Constructing the "boolean" polynomial}
  \begin{block}{Technique by Example}
    Let $\phi = (\neg x_1 \vee x_2 \vee x_3) \wedge (x_1 \vee x_2 \vee \neg x_3).$\\
    We consider a polynomial $q(x_1,x_2,x_3)$ such that if $(x_1^\prime,x_2^\prime,x_3^\prime)$ is a truth assignment for $\phi$ then $q(x_1^\prime,x_2^\prime,x_3^\prime)=1$ and $=0$ otherwise.
    For the example $\phi$ we get
    $$q(x_1,x_2,x_3)=[1-x_1(1-x_2)(1-x_3)][1-(1-x_1)(1-x_2)x_3]$$
    $$deg(q) \leq 3m$$
  \end{block}
  \begin{block}{Equivalence}
    We see that $\phi \in \#3SAT$ for a constant $K$ iff
    $$\sum_{b_1=0}^1 \sum_{b_2=0}^1 \sum_{b_3=0}^1 q(b_1,b_2,b_3) = K$$
  \end{block}
\end{frame}

\begin{frame}{Constructing the single variable "boolean" polynomial}
  \begin{block}{To avoid exponential computations}
    Since the $n$ variables and $m$ clauses could cause exponential calculatios, we observe that given a large enough prime number $p$
    $$\sum_{b_1=0}^1 \sum_{b_2=0}^1 \sum_{b_3=0}^1 q(b_1,b_2,b_3) = K$$ iff $$\sum_{b_1=0}^1 \sum_{b_2=0}^1 \sum_{b_3=0}^1 q(b_1,b_2,b_3) = K modp$$
  \end{block}
  \begin{block}{But why a prime number?}
    By selecting a prime number for this task, we achieve that the resulting ring is a field.
  \end{block}
\end{frame}

\section{Constructing the single variable boolean polynomial}

\begin{frame}{Constructing the single variable "boolean" polynomial}
  \begin{block}{}
    Now consider the following single variable polynomial
    $$r(X_1) = \sum_{b_2=0}^1 \sum_{b_3=0}^1 q(x_1,b_2,b_3)$$
    $$deg(r) \leq 3m$$
    It follows that $r(0)+r(1)=K$.
  \end{block}
\end{frame}

\section{Prover-Verifier Interaction}

\begin{frame}{Prover-Verifier Interaction}
  \begin{block}{}
    For a Prover $P$ and a Verifier $V$:
    \begin{itemize}
      \item $P$ and $V$ both construct the polynomial $q$
      \item $P$ sends a prime number $p$ of $2n$ bits
      \item $P$ sends the $r(X_1)$ to $V$
      \item $V$ starts the verification
        \begin{itemize}
          \item if $r(0)+r(1)=K modp$ then check if $r(\alpha) = q(\alpha,b_2,b_3) modp$ for an $\alpha \neq 0,1$ and accept/reject accordingly with probability of success $\frac{3m}{2^{2n}}$
          \item if $r(0)+r(1)\neq K modp$ then reject
        \end{itemize}
    \end{itemize}
  \end{block}
\end{frame}

\section{References}

\begin{frame}{References}
  \begin{itemize}
    \item LUND, C., FORTNOW, L., KARLOFF, H., AND NISAN, N. 
      \href{https://users.cs.fiu.edu/~giri/teach/5420/f01/LundIPS.pdf}{Algebraic methods for interactive proof systems},
      In Proceedings of 31st Symposium on Fottndations of Computer Science (EEE, New York, 1990, pp. 2-90)
    \item \href{https://www.youtube.com/watch?v=yS9WrISP_Ug&list=PLm3J0oaFux3YL5vLXpzOyJiLtqLp6dCW2&index=26}{Undergrad Complexity at CMU - Lecture 25: Interactive Proofs: IP=PSPACE}
  \end{itemize}
\end{frame}

\end{document}
