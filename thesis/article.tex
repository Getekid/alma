\documentclass[a4paper,11pt]{article}
\usepackage{a4wide}

% Greek fonts
\RequirePackage{fontspec}
\defaultfontfeatures{Ligatures=TeX}
  % you may want to try: {Liberation Serif} or {Times New Roman}
\setmainfont{FreeSerif}
  % you may want to try: {Liberation Sans} or {Arial}
\setsansfont[Scale=MatchLowercase]{FreeSans}
  % you may want to try: {FreeMono} or {Courier New}
\setmonofont[Scale=MatchLowercase]{FreeMono}

% \usepackage{mathtools}
% \usepackage{tikz}
\usepackage{enumerate}
\usepackage{amssymb}
\usepackage{amsthm}

% Commands for wrapping properly common expressions.
\newcommand{\indeq}[1]{\stackrel{\text{#1}}{=}}
\newcommand{\RightarrowArg}[1]{\stackrel{#1}{\Longrightarrow}}
\newcommand{\LeftrightarrowArg}[1]{\stackrel{#1}{\Leftrightarrow}}
\newcommand{\NE}{\mathrm{N.E.}}
\newcommand{\as}{\mathrm{\alpha_s}}
% \newcommand{\Exp}{\mathrm{Exp}}
% \newcommand{\Expect}{{\rm I\kern-.3em E}}

% Theorem structures.
\newtheorem{definition}{Definition}[section]
\newtheorem{lemma}{Lemma}[section]

% Main document
\begin{document}
\title{Profit Optimisation in Heterogeneous 2-link Parallel Games}
\author{Thomas Pappas}
\date{}
\maketitle

\section*{Abstract}
We study (so far) a 3-level optimisation problem where, in a 2-link $s-t$ network, link owners compete for profit by assigning tolls to links, thus creating a toll congestion game.
In addition, each player responds to tolls in a heterogeneous way, described by a distribution function $\alpha(p)$.
We first introduce a new term $\as$ which, given a set of tolls, can help us describe how the flow is split between the 2 links, and then we will use it to prove some properties of the game.
We will show that even with affine latencies, there are cases where no Nash Equilibrium exists in the profit game between the toll owners.

Then what else? Maybe find cases where there always exists a $\NE$.


\section{Introduction}

In this thesis we study a three level optimisation problem.
On the basis there is a 2-node $(s, t)$, 2-link $s-t$ network, each link with a non-decreasing latency function that defines how the traffic increases as more players flow into that link.
On the first level a flow $[0, 1]$ wants to move from $s$ to $t$ and experience the minimum latency (traffic).
On the second level, the links might also have tolls which impose an additional cost to the flow players, plus in our case (heterogeneous users) each player has a different money-time trade-off and thus see the link costs differently.
Finally on the third level (are the levels really that way?), the tolls are owned by players who profit on the flow that uses that link and compete to maximise it.

We know from [insert ref] that regardless of the distribution function $a$ or the latency functions (as long as they are non-decreasing) a $\NE$ always exists.
In that $\NE$ the users are split with the money-sensitive ones on the link with the lower toll and the time-sensitive ones on the higher toll one, each one seeing their link as either lower or equal cost with the other (while they all see traffic equally).

We introduce a new term $\as=\frac{l_1(x)-l_2(1-x)}{t_2-t_1}$ which is the $a(p)$ value of a user $p$ which sees the link costs equally.


\section{Related Work}
TODO


\section{Preliminaries}

A 2-link $s-t$ parallel network is a graph with a source node $s$, a destination node $t$ and $2$ edges connecting $s$ to $t$.
Each link is associated with a latency function $l_1, l_2$
[TODO: Add figure]
A flow of infinitesimal users totaling up to a flow of $[0, 1]$ wishes to move from $s$ to $t$ through the link with the lowest cost.


\section{2-link parallel networks}

We start with the simple case of 2-link parallel networks.
[TODO: Add figure]
In this simple setup a user of the flow 

\subsection{Time-money sensitivity split}

Consider a 2-link parallel network with latency functions $l_1(x), l_2(x)$ and tolls $t=(t_1, t_2)$ assigned to them.
Without loss of generality we assume $t_1 > t_2$ which in turn means that for the resulting equilibrium flow $x = (x_1(t), x_2(t))$ it holds that $l_1(x_1(t)) < l_2(x_2(t))$.
This equilibrium is guaranteed to exist according to [TODO: add ref]).
For the 2 edges of the game we introduce the term time-money sensitivity split $\as(t)$ as the value for which if any player $p$ has $\alpha(p)=\as(t)$ then that player sees the $2$ edge total latencies as equal.
More specifically, $\as(t)$ is the value of the distribution function $\alpha$ for which it holds
\[l_1(x_1(t)) + \as(t) t_1 = l_2(x_2(t)) + \as(t) t_2\]
Solving for $\as$ we get
\[\as(t) = \frac{l_2(x_2(t)) - l_1(x_1(t))}{t_1 - t_2}\]
\\
Before we get into a more formal definition and description of $\as$, it'd help to first discuss the nature of the split that $\as$ captures.
Regardless of whether there exists a player $p$ such that $\alpha(p) = \as(t)$, it still holds that
\begin{itemize}
	\item if $\alpha(p) < \as$ then $l_1(x_1(t)) + \alpha(p) t_1 < l_2(x_2(t)) + \alpha(p) t_2$, thus $p$ is on edge $1$
	\item if $\alpha(p) > \as$ then $l_1(x_1(t)) + \alpha(p) t_1 > l_2(x_2(t)) + \alpha(p) t_2$, thus $p$ is on edge $2$
	\item if $\alpha(p) = \as$ then $l_1(x_1(t)) + \alpha(p) t_1 = l_2(x_2(t)) + \alpha(p) t_2$, thus $p$ is either on edge $1$ or $2$
\end{itemize}
Remember that we have assumed that $t_1 > t_2 \Rightarrow l_1(x_1(t)) < l_2(x_2(t))$.
Therefore the flow passing through the faster edge $1$ is contained with players where $\alpha(p) \le \as(t)$, while the flow in the cheaper edge $2$ is contained with players where $\alpha(p) \ge \as(t)$.
Since the players are sorted in increasing order according to $\alpha$, it helps to view the players in the first case as the "lower" part of the split and respectively the players in the second case as the "higher" part of the split.
[TODO: Alternative splits: rushed-relaxed, rich-poor]
Table \ref{table:split_summary} provides of a summary for each part of the split

\begin{table}[h!]
	\centering
	\caption{Summary of properties for the sensitivity split.}
	\begin{tabular}{| c || c | c |}
		\hline
		& $\alpha(p) \le \as(t)$ & $\alpha(p) \ge \as(t)$ \\ \hline
		sensitivity & time $\ge$ money & time $\le$ money \\ \hline
		latency & lower $(l_2)$ & higher $(l_1)$ \\ \hline
		toll & higher $(t_1)$ & lower $(t_2)$ \\ \hline
		edge & low-latency, high-toll $(1)$ & high-latency, low-toll $(2)$ \\ \hline
		$t_1 - t_2$ & lower & higher \\ \hline
		split & lower & higher \\ \hline
	\end{tabular}
	\label{table:split_summary}
\end{table}

Finally to acknowledge that we have arbitrarily handled any players with $\alpha(p) = \as(t)$.
Those players, if any, even though they see both edges with equal total cost, in the optimal flow $x$ they have settled in either edge $1$ or $2$.
We investigate more in depth the relation of $\as$ to the flow in [TODO: add ref].
\\[12pt]
We can now formally define $\as$ and prove some of its properties.
\begin{definition}
	Given a 2-link parallel heterogeneous game with latency functions $l_1, l_2$, we define the time-money sensitivity split function $\as: \mathbb{R}^2_+ \rightarrow (0, +\infty)$ where for given tolls $t = (t_1, t_2)$
	\[\as(t) = \frac{l_2(x_2(t)) - l_1(x_1(t))}{t_1 - t_2}\]
\end{definition}
The function is well defined since the given tolls $t$ result in a single optimal flow $x(t) = (x_1(t), x_2(t))$ [TODO: add ref].

We start with some basic properties.
\begin{lemma}
	\label{lemma:split_basic}
	Given a 2-link parallel heterogeneous game with distribution function $\alpha$ and time-money sensitivity split function $\as$, the following hold:
	\begin{enumerate}[(i)]
		\item $\as(t) > 0$
		\item if $x_1(t), x_2(t) > 0$ then $\as(t) \in [\alpha(0), \alpha(1)]$
		\item if $0 < x_1(t) < x_2(t)$ then $\alpha(x_1(t)) \le \as(t) \le \alpha(x_1(t) + \epsilon), \forall \epsilon > 0$ arbitrarily small
	\end{enumerate}
\end{lemma}

\begin{proof}
	$ $ % Just to make a line break after "Proof".
	\begin{enumerate}[(i)]
		\item Obviously $\as(t) \ne 0$ since $\as(t) = 0 \Rightarrow l_1(x_1(t)) = l_2(x_2(t))$ which is impossible in a heterogeneous setup.
		Then we only need to notice that $l_1(x_1(t)) < l_2(x_2(t)) \iff t_1 > t_2$ and we get $\as(t) > 0$ from definition.
		\item Since $x_1(t), x_2(t) > 0$ then $\exists p_1, p_2 \in [0, 1]$ each on a different edge, for which we assume w.l.o.g. that $\alpha(p_1) \le \alpha(p_2)$.
		It holds that $\alpha(p_1) \le \as(t) \le \alpha(p_2)$ so with $\alpha$ being non-decreasing it follows
		\[\alpha(0) \le \alpha(p_1) \le \as(t) \le \alpha(p_2) \le \alpha(1) \Rightarrow \as(t) \in [\alpha(0), \alpha(1)]\]
		\item We know that $x_1(t) < x_2(t)$ means that edge 1 has the lower split.
		Therefore we have
		\[\alpha(p) \le \as(t), \forall p \in [0, x_1(t)] \Rightarrow \alpha(x_1(t)) \le \as(t)\]
		Likewise for edge 2 we have
		\[\as(t) \le \alpha(p), \forall p \in (x_1(t), 1] \Rightarrow \as(t) \le \alpha(x_1(t) + \epsilon), \forall \epsilon \in (0, 1 - x_1(t))\]
	\end{enumerate}
\end{proof}

The next lemma will help us describe how $\as(t)$ is changing in relation to toll difference (and in extend to latency difference).
\begin{lemma}
	Given a 2-link parallel heterogeneous game with time-money sensitivity split function $\as$ and tolls $t, t^\prime$ such that $\frac{t_1 - t_2}{t_1^\prime - t_2^\prime} > 0$, the following hold:
	\begin{enumerate}[(i)]
		\item $\as(t) \ge \as(t^\prime)$ iff $\frac{t_1 - t_2}{t_1^\prime - t_2^\prime} \le 1$
		\item $\as(t) \le \as(t^\prime)$ iff $\frac{t_1 - t_2}{t_1^\prime - t_2^\prime} \ge 1$
	\end{enumerate}
\end{lemma}

\begin{proof}
	Before we get into the propositions, notice that the condition
	\begin{equation}
		\frac{t_1 - t_2}{t_1^\prime - t_2^\prime} > 0
		\label{eq:split_property_cond_tolls_0}
	\end{equation}
	ensures that the two tolls provide splits with same edges in their respective lower and higher sections.
	We show this for $t_1 > t_2$ and likewise it also holds for $t_1 < t_2$.
	\[t_1 > t_2 \Leftrightarrow t_1 - t_2 > 0 \LeftrightarrowArg{(\ref{eq:split_property_cond_tolls_0})} t_1^\prime - t_2^\prime > 0 \Leftrightarrow t_1^\prime > t_2^\prime\]
	%Also note that this property extends to the latencies, giving us
	%\begin{equation}
	%	\frac{l_1(x_1(t)) - l_2(x_2(t))}{l_1(x_1^\prime(t)) - l_2(x_2^\prime(t))} > 0
	%	\label{eq:split_property_cond_latencies_0}
	%\end{equation}
	Now moving on to the prepositions, we focus on proving $(i)$ in the right-to-left direction; then the same steps can be followed backwards to prove left-to-right.
	Finally proving $(ii)$ will be much easier as there are many steps symmetric to the ones for $(i)$.
	
	\paragraph{$(i)$}
	Begin by noticing that $\frac{t_1 - t_2}{t_1^\prime - t_2^\prime} \le 1 \Leftrightarrow |t_1 - t_2| \le |t_1^\prime - t_2^\prime|$ (opposite direction is ensured by $(\ref{eq:split_property_cond_tolls_0})$), i.e. the toll difference is increasing.
	Assume w.l.o.g. that $x_1 < x_2$.
	
	If $x_1(t) = 0$ then
	\[\as(t) = \frac{l_2(1)}{t_2 - t_1}, \as(t^\prime) = \frac{l_2(1)}{t_2^\prime - t_1^\prime}\] 
	so obviously $t_2 - t_1 \le t_2^\prime - t_1^\prime \Rightarrow \as(t) \ge \as(t^\prime)$.
	
	If $x_1(t) > 0$ then we know [TODO: do we?] that the latency difference might also increase, with flow from the lower latency edge, if any, moving towards the slower one, i.e. $x_1(t) \ge x_1(t^\prime)$.
	%\[|t_1 - t_2| \le |t_1^\prime - t_2^\prime| \Leftrightarrow |l_1(x_1(t)) - l_2(x_2(t))| \le |l_1(x_1(t^\prime)) - l_2(x_2(t^\prime))|\]
	%\begin{equation}
	%	|l_1(x_1(t)) - l_2(x_2(t))| \le |l_1(x_1(t^\prime)) - l_2(x_2(t^\prime))| \LeftrightarrowArg{(\ref{eq:split_property_cond_latencies_0})} \frac{l_1(x_1(t)) - l_2(x_2(t))}{|l_1(x_1(t^\prime)) - l_2(x_2(t^\prime))|} \le 1
	%	\label{eq:split_property_cond_latencies_1}
	%\end{equation}
	If no flow moves then we can easily show that $\as(t) \ge \as(t^\prime)$ similarly to the case with $x_1(t) = 0$.
	Otherwise we have $x_1(t) > x_1(t^\prime) \Rightarrow \exists \epsilon > 0: x_1(t) = x_1(t^\prime) + \epsilon$ and by carefully applying property $\ref{lemma:split_basic}(ii)$ we get
	\[\as(t^\prime) \le \alpha(x_1(t^\prime) + \epsilon) = \alpha(x_1(t)) \le \as(t)\]
	
	\paragraph{$(ii)$}
	Symmetrically to $(i)$ in this case the toll difference is decreasing, and therefore flow from the higher latency edge will move towards the lower one, i.e. $x_1(t) \le x_1(t^\prime)$ (assuming again $x_1 < x_2$).
	With a similar argument we then get $\as(t) \le \as(t^\prime)$.
\end{proof}

The next lemma gives us some insight of the behavior of $\as(t)$ around its edge values.
\begin{lemma}
	Given a 2-link parallel heterogeneous game with distribution function $\alpha$ and time-money sensitivity split function $\as$, the following hold:
	\begin{enumerate}[(i)]
		\item $\as(t) \rightarrow 0$ iff $|t_1 - t_2| \rightarrow \infty$
		\item $\lim_{t_1 \rightarrow t_2^-} \as(t) = \alpha(x_1), \lim_{t_1 \rightarrow t_2^+} \as(t) = \alpha(x_2)$, where ($x_1, x_2)$ the optimal flow under no tolls
	\end{enumerate}
\end{lemma}

\begin{proof}
	\begin{enumerate}[(i)]
		\item If $\alpha(0) > 0$ then...
		\item 
	\end{enumerate}
\end{proof}

\subsection{Example with no $\NE$}



\end{document}
