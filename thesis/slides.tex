\documentclass{beamer}
\usepackage{tikz}
\usepackage{multicol}
\usepackage[style=authoryear,backend=bibtex]{biblatex}
\addbibresource{ref.bib}

\mode<presentation>
{
	\usetheme[subsectionpage=simple,progressbar=foot,numbering=fraction,background=light,block=fill]{metropolis}
	\usecolortheme{default}
	\usefonttheme{default}
	\setbeamercolor{example text}{fg=mDarkTeal}
	\setbeamertemplate{navigation symbols}{}
	\setbeamertemplate{caption}[numbered]
	\setbeamertemplate{frame footer}{Thomas Pappas | ALMA }
}

% Commands for wrapping properly common expressions.
\newcommand{\indeq}[1]{\stackrel{\text{#1}}{=}}
\newcommand{\RightarrowArg}[1]{\stackrel{#1}{\Rightarrow}}
\newcommand{\LeftrightarrowArg}[1]{\stackrel{#1}{\Leftrightarrow}}
\newcommand{\NE}{\mathrm{N.E.}}
\newcommand{\as}{\mathrm{\alpha_s}}
\newcommand{\R}{\mathbb{R}}
\newcommand{\Gm}{\mathcal{G}}
\DeclareMathOperator*{\argmax}{arg\,max}

\title[]{Pricing Games in Heterogeneous Parallel Networks}
\author[Thomas Pappas]{Thomas Pappas}
\institute[ALMA]{
	\begin{columns}
		\column{0.4\textwidth}
		\begin{figure}
			\centering
			\includegraphics[scale=0.1]{alma.png}
		\end{figure}
		\column{0.5\textwidth}
		ALMA \\
		\tiny{INTER-INSTITUTIONAL GRADUATE PROGRAM \\
			"ALGORITHMS, LOGIC AND DISCRETE MATHEMATICS"}
	\end{columns}
}
%\date{\today}
\date{February 10, 2025}

\begin{document}
\AtBeginSection[]{
	\begin{frame}[noframenumbering,plain]
		\vfill
		\centering
		\begin{beamercolorbox}[sep=8pt,center,shadow=true,rounded=true]{title}
			\usebeamerfont{title}\insertsectionhead\par
		\end{beamercolorbox}
		\vfill
	\end{frame}
}


\begin{frame}[noframenumbering,plain]
  \titlepage
\end{frame}

\begin{frame}[noframenumbering,plain]
    \frametitle{Agenda}
    \tableofcontents[hideallsubsections]
\end{frame}


\section{Brief History}

\begin{frame}{\insertsection}
	Selfish routing can model and solve a variety of real-world problems in economics, transportation and more.
	\\[20pt]
	\begin{tikzpicture}[scale=0.9, every node/.style={font=\small,align=center}]
		\draw (0,-0.15) node[below] {Pigou} -- (0,0.15) node[above] {$1920$};
		\draw (0,0) -- (6,0);
		\draw (6,-0.15) node[below] {Beckman\\et al.} -- (6,0.15) node[above] {$1956$};
		\draw (6,0) -- (9,0);
		\draw (9,-0.15) node[below] {Rosenthal} -- (9,0.15) node[above] {$1973$};
		\draw (9,0) -- (10,0);

		\draw (0,-2.65) node[below] {Cole\\et al.} -- (0,-2.35) node[above] {$2003$};
		\draw (0,-2.5) -- (2,-2.5);
		\draw (2,-2.65) node[below] {Acemoglu\\\& Ozdaglar} -- (2,-2.35) node[above] {$2007$};
		\draw (2,-2.5) -- (7.5,-2.5);
		\draw (7.5,-2.65) node[below] {Correa\\et al.} -- (7.5,-2.35) node[above] {$2018$};
		\draw (7.5,-2.5) -- (9,-2.5);
		\draw (9,-2.65) node[below] {Harks\\et al.} -- (9,-2.35) node[above] {$2019$};
		\draw (9,-2.5) -- (10,-2.5);
	\end{tikzpicture}
\end{frame}

\section{Preliminaries}

\subsection{Our model}

\begin{frame}{Preliminaries | Our model}
	\begin{center}
		Heterogeneous Non-atomic $n$-link Parallel Network\\
		Toll Congestion Pricing Game
	\end{center}
\end{frame}
\begin{frame}{Preliminaries | Our model}
	\begin{center}
		\textbf{Heterogeneous} Non-atomic $n$-link Parallel Network\\
		Toll Congestion \textbf{Pricing} Game
	\end{center}
\end{frame}

\begin{frame}{Preliminaries | Our model}
	\;
	\begin{block}{$2$-level game}
		\begin{enumerate}
			\item Shelfish routing game, flow on minimum cost paths
			\item Profit maximisation game for link operators
		\end{enumerate}
	\end{block}
\end{frame}

\begin{frame}{Preliminaries | Selfish routing (1\textsuperscript{st} level)}
	\;
	\begin{block}{Non-atomic $n$-link Parallel Network Game}
		A \textbf{Parallel Network} is a directed graph $G = (\{s, t\}, N)$, where $N = \{1, \dots, n\}$ a set of $n$ parallel $s-t$ links.

		For a \textbf{non-atomic} game, a unit of traffic $[0, 1]$, endowed with Lebesque measure $\lambda$, wishes to travel from $s$ to $t$.

		Traffic creates congestion on links and each \textit{player} $p \in [0, 1]$:
		\begin{itemize}
			\item is selfish and wants to experience minimum congestion
			\item has infinitesimal effect on congestion
		\end{itemize}
	\end{block}

	\begin{center}
		\begin{tikzpicture}[scale=0.8, every node/.style={font=\small}]
			% Nodes
			\node[circle, draw, fill=blue!20, minimum size=0.8cm] (s) at (0, 0) {$s$};
			\node[circle, draw, fill=blue!20, minimum size=0.8cm] (t) at (4, 0) {$t$};

			% Edges
			\draw[->, thick] (s) to[bend left=35] node[midway, above] {$1$} (t);
			\draw[->, thick] (s) to[bend left=10] node[midway, above] {$2$} (t);
			\node[align=center] at (2, 0) {$\vdots$};
			\draw[->, thick] (s) to[bend right=35] node[midway, above] {$n$} (t);
		\end{tikzpicture}
	\end{center}
\end{frame}

\begin{frame}{Preliminaries | Selfish routing (1\textsuperscript{st} level)}
	\;
	\begin{block}{Non-atomic $n$-link Parallel Network Game}
		Flow $f: [0, 1] \rightarrow N$ is a Lebesque measurable function assigning players to links.

		\textit{Flow on paths} $x_i = \lambda(\{p \in [0, 1]: f(p) = i\})$
		\begin{itemize}
			\item $x = (x_i)_{i \in N}$ a (stochastic) vector with $x_i \ge 0$ and $\sum_{i \in N} x_i = 1$
		\end{itemize}
		Congestion on links is represented by affine latency functions $(\ell_i)_{i \in N}$ where $\ell_i \in \mathcal{L}_1$, making congestion on link $i$ equal to $\ell_i(x_i)$.
	\end{block}

	\begin{center}
		\begin{tikzpicture}[scale=1.2, every node/.style={font=\small}]
			% Nodes
			\node[circle, draw, fill=blue!20, minimum size=0.8cm] (s) at (0, 0) {$s$};
			\node[circle, draw, fill=blue!20, minimum size=0.8cm] (t) at (4, 0) {$t$};

			% Edges
			\draw[->, thick] (s) to[bend left=35] node[midway, above] {$\ell_1(x_1)$} (t);
			\draw[->, thick] (s) to[bend left=10] node[midway, above] {$\ell_2(x_2)$} (t);
			\node[align=center] at (2, 0) {$\vdots$};
			\draw[->, thick] (s) to[bend right=35] node[midway, above] {$\ell_n(x_n)$} (t);
		\end{tikzpicture}
	\end{center}
\end{frame}

\begin{frame}{Preliminaries | Selfish routing (1\textsuperscript{st} level) | Example}
\begin{center}
	\begin{tikzpicture}[scale=1.4, every node/.style={font=\small}]
		% Nodes
		\node[circle, draw, fill=blue!20, minimum size=0.8cm] (s) at (0,0) {$s$};
		\node[circle, draw, fill=blue!20, minimum size=0.8cm] (t) at (4,0) {$t$};

		% Links
		\draw[->, thick] (s) to[bend left=35] node[midway, above] {$2x$} (t);
		\draw[->, thick] (s) to node[midway, above] {$x + 1$} (t);
		\draw[->, thick] (s) to[bend right=35] node[midway, above] {$x + 2$} (t);
	\end{tikzpicture}
\end{center}
Flow $x = \left(\frac23, \frac13, 0\right)$ has all players on minimum latency links.
\end{frame}

\begin{frame}{Preliminaries | Selfish routing (1\textsuperscript{st} level)}
	Tolls $t \in \R_+^n$ can be assigned to links, affecting player cost.

	In \textbf{homogeneous} networks, player $p$ on link $i$ experiences total cost $c_i(p) = \ell_i(x_i) + t_i$.

	In \textbf{heterogeneous} networks, players value time-money differently, so we add a weight $\alpha$ to the toll (money) cost.

	Player $p$ on link $i$ experiences total cost $c_i(p) = \ell_i(x_i) + \alpha(p) \cdot t_i$.
	\begin{block}{Distribution function $\alpha: [0, 1] \rightarrow [0, +\infty]$}
		\begin{itemize}
			\item players are ordered w.r.t. sensitivity
			\item $\alpha$ is non-decreasing
		\end{itemize}
	\end{block}
\end{frame}

\begin{frame}{Preliminaries | Selfish routing (1\textsuperscript{st} level)}
	In homogeneous networks, all players view link costs the same.
	\begin{definition}
		For a given set of tolls $t$, a flow $x$ is a \textit{Wardrop equilibrium for $t$} if $\forall i, j \in N$ with $x_i > 0$ it holds that
		\begin{equation*}
			\ell_i(x_i) + t_i \leq \ell_j(x_j) + t_j
		\end{equation*}
	\end{definition}
	Links with $x_i > 0$ have $\ell_i(x_i) + t_i = K$ for some $K > 0$.

	Equilibrium is indifferent to uniform variations of tolls ($t^\prime = t + c$).
\end{frame}

\begin{frame}{Preliminaries | Selfish routing (1\textsuperscript{st} level)}
	\begin{lemma}[variational inequality\footcites{beckmann1956studies}{dafermos1973toll}]
		A flow $x$ is a Wardrop equilibrium for $t$ if and only if for all feasible flows $x^\prime$,
		\[\sum_{i \in N} (\ell_i(x_i) + t_i) \cdot (x_i - x_i^\prime) \leq 0\]
	\end{lemma}
\end{frame}

\begin{frame}{Preliminaries | Selfish routing (1\textsuperscript{st} level)}
	\begin{definition}
		For a given set of tolls $t$, a flow $x$ is a \textit{Nash equilibrium for $t$} if $\forall p \in [0, 1]$ and $\forall i \in N$ it holds that
		\begin{equation*}
			c_{f(p)}(p) \leq c_i(p)
		\end{equation*}
	\end{definition}
	N.E. always exists and is unique w.r.t. costs for homogeneous\footcite{beckmann1956studies} and heterogeneous\footcites{1973JSP.....7..295S}{MILCHTAICH1996111} networks.
	We denote it by $x(t)$.
\end{frame}

\begin{frame}{Preliminaries | Selfish routing (1\textsuperscript{st} level) | Example}
\begin{center}
	\begin{tikzpicture}[scale=1.4, every node/.style={font=\small}]
		% Nodes
		\node[circle, draw, fill=blue!20, minimum size=0.8cm] (s) at (0,0) {$s$};
		\node[circle, draw, fill=blue!20, minimum size=0.8cm] (t) at (4,0) {$t$};

		% Links
		\draw[->, thick] (s) to[bend left=35] node[midway, above] {$2x$} (t);
		\draw[->, thick] (s) to node[midway, above] {$x + 1$} (t);
		\draw[->, thick] (s) to[bend right=35] node[midway, above] {$x + 2$} (t);
	\end{tikzpicture}
\end{center}
For homogeneous players and tolls $t = (3, 2, 1)$
\[x(t) = \left(\frac15, \frac25, \frac25\right)\]
since $\ell_2(1/2) + t_2 = \ell_3(1/2) + t_3 = 7/2$ and $\ell_1(0) + t_1 = 4$.
\end{frame}

\begin{frame}{Preliminaries | Pricing competition (2\textsuperscript{nd} level)}
	Links are owned by agents who compete for profit.
	\begin{block}{Profit}
		For given set of tolls $t \in \R_+^n$, link operator $i$ gains profit
		\[\Pi_i(t) = x_i(t) \cdot t_i\]
	\end{block}
	\begin{block}{Best response}
		For fixed tolls assignments from remaining link operators $t_{-i} = t \setminus \{t_i\}$, link operator $i$ will try and maximise their profit
		\[B_i(t_{-i}) = \argmax_{t_i \ge 0} \Pi(t_i, t_{-i})\]
	\end{block}
\end{frame}

\begin{frame}{Preliminaries | Pricing competition (2\textsuperscript{nd} level)}
	\begin{block}{Nash Equilibrium ($t^*$)}
		A set of tolls $t^*$ is a \textit{Nash Equilibrium for the pricing game} if $\forall i \in N$ and $\forall t_i^\prime \in \R_+$
		\[\Pi_i(t_i^*, t_{-i}^*) \geq \Pi_i(t_i^\prime, t_{-i}^*)\]
	\end{block}
	\begin{block}{Nash Equilibrium ($t^*$)}
		A set of tolls $t^*$ is a \textit{Nash Equilibrium for the pricing game} if
		$\forall i \in N$ we have
		\[t^* = (B_i(t_{-i}^*))_{i \in N}\]
	\end{block}
\end{frame}

\begin{frame}{Preliminaries | Our model}
	We will describe a pricing game as the tuple $\Gm = (N, \ell, \alpha)$ where
	\begin{itemize}
		\item $N = \{1, 2, \cdots, n\}$ the links
		\item $\ell = (\ell_i)_{i \in N}$ the latency functions
		\item $\alpha$ the distribution function
	\end{itemize}
	A homogeneous game can be described as $(N, \ell, 1)$.
\end{frame}

\subsection{Game equivalence}

\begin{frame}{Preliminaries | Game equivalence}
	\begin{definition}[Game equivalence]
		Let $\Gm_1, \Gm_2$ be two $n$-link Network Congestion Games.
		$\Gm_1$ and $\Gm_2$ are called \textbf{equivalent} if and only if for all tolls $t \in \R_+^{|N|}$ it holds that $x^{(1)}(t) = x^{(2)}(t)$, with $x^{(1)}(t), x^{(2)}(t)$ being the Nash Equilibria for $t$ in $\Gm_1, \Gm_2$ respectively.
	\end{definition}
	Equivalency means same tolls cause same flow.

	The two pricing competition games played on two equivalent network congestion games are identical.
\end{frame}

\subsection{Homogeneous networks}

\begin{frame}{Preliminaries | Homogeneous networks}
	\begin{block}{Flow formula\footcite{Harks_2019}}
		Define $N(t) = \{i \in N | x_i(t) > 0\}$.
		For affine latency functions $\ell_i(x) = a_i x + b_i$, $a_i > 0, b_i \ge 0$ and by solving $\ell_i(x_i) + t_i = K$ with $\sum_{i \in N} x_i = 1$, calculate $K$ and get for all $i \in N$
		\begin{equation*}
			x_i(t) = \frac{1 + \sum_{j \in N(t)}\frac{b_j + t_j - b_i - t_i}{a_j}}{\sum_{j \in N(t)}\frac{a_i}{a_j}}
		\end{equation*}
	\end{block}
\end{frame}

\begin{frame}{Preliminaries | Homogeneous networks}
	\begin{block}{Full Wardrop support assumption}
		$x_i(0) > 0$ for all $i \in N$
	\end{block}
	\begin{block}{Pricing N.E. properties\footcite{Harks_2019}}
		$\Pi_i(t)$ is either linear or quadratic, therefore $|B_i(t_{-i})| = 1$.
		If the \textit{full Wardrop support assumption} is satisfied then $t^*$ is unique, $x_i(t^*) > 0$ and $t_i^* > 0$ for all $i \in N$ and
		\begin{equation*}
			t_i^* = \left(a_i + \frac{1}{\sum_{j \ne i} \frac{1}{a_j}}\right) \cdot x_i(t^*)
		\end{equation*}
	\end{block}
\end{frame}

%\begin{frame}{Preliminaries | Homogeneous players | Example}
%	\begin{example}
%	\end{example}
%	$x(0) = \left(\frac23, \frac13\right), \; x(3, 2) = \left(\frac14, \frac34\right)$
%\end{frame}

\section{Sensitivity split}

\begin{frame}{Sensitivity split | Heterogeneous networks}
	\begin{center}
		\begin{multicols}{2}
			% Left Column: Network Graph
			\begin{tikzpicture}[scale=1.2, baseline=40pt, every node/.style={font=\small}]
				% Nodes
				\node[circle, draw, fill=blue!20, minimum size=0.8cm] (s) at (0,0) {$s$};
				\node[circle, draw, fill=blue!20, minimum size=0.8cm] (t) at (4,0) {$t$};

				% Links
				\draw[->, thick] (s) to[bend left=35] node[midway, above] {$2x$} (t);
				\draw[->, thick] (s) to node[midway, above] {$x + 1$} (t);
				\draw[->, thick] (s) to[bend right=35] node[midway, above] {$x + 2$} (t);
			\end{tikzpicture}

			% Right Column: Distribution function Graph
			\begin{tikzpicture}[x=2.5cm, y=1.2cm, every node/.style={font=\small}]
				% Axes
				\draw[->] (0,0) -- (1.2,0) node[right] {$p$};
				\draw[->] (0,0) -- (0,2.2) node[above] {$\as(p)$};

				% Tick marks
				\draw (0,0) -- (0,-0.05) node[below] {$0$};
				\draw (1,0) -- (1,-0.05) node[below] {$1$};
				\draw (0,1) -- (-0.05,1) node[left] {$1$};
				\draw (0,2) -- (-0.05,2) node[left] {$2$};

				% Function graph
				\draw[thick] (0,1) -- (1,2);

				% Description
				\node[align=center] at (0.7, -0.8) {$\alpha(p) = p + 1$};
			\end{tikzpicture}
		\end{multicols}
	\end{center}
	For heterogeneous players with $\alpha(p) = p + 1$ and tolls $t = (3, 2, 1)$
	\[x(t) = \left(\frac{1}{11}, \frac{3}{11}, \frac{7}{11}\right)\]
\end{frame}

\tikzset{
	fixed_height/.style={
		inner sep=0,
		outer sep=0,
		minimum height=15pt,
		align=center
	}
}

\begin{frame}{Sensitivity split | Heterogeneous networks}
	Latencies are the same for all players.

	Toll ordering is the same for all players.

	Toll differences are viewed with different scaling.
	\\[15pt]
	\begin{tikzpicture}
		% Title
		\node[above,fixed_height] at (3.7,3.3) {$\alpha(0) = 1$};

		% Axes
		\draw[->] (0,0) -- (7,0) node[right=5,fixed_height] {$c\left(0\right)$};
		\draw[-] (0,0) -- (0,3.2);

		% Y-bars: Latencies
		\node[left] at (0,2.5) {$\frac{2}{11}$};
		\node[left] at (0,1.5) {$\frac{14}{11}$};
		\node[left] at (0,0.5) {$\frac{29}{11}$};
		\filldraw[fill=blue!20] (0,2.25) rectangle (2/11,2.75);
		\filldraw[fill=blue!20] (0,1.25) rectangle (14/11,1.75);
		\filldraw[fill=blue!20] (0,0.25) rectangle (29/11,0.75);

		% Y-bars: Tolls
		\filldraw[fill=orange!20] (2/11,2.25) rectangle (35/11,2.75) node[below right] {$3$};
		\filldraw[fill=orange!20] (14/11,1.25) rectangle (36/11,1.75) node[below right] {$2$};
		\filldraw[fill=orange!20] (29/11,0.25) rectangle (40/11,0.75) node[below right] {$1$};
	\end{tikzpicture}
\end{frame}

\begin{frame}{Sensitivity split | Heterogeneous networks}
	Latencies are the same for all players.

	Toll ordering is the same for all players.

	Toll differences are viewed with different scaling.
	\\[15pt]
	\begin{tikzpicture}
		% Title
		\node[above,fixed_height] at (3.7,3.3) {$\alpha\left(\frac{2}{11}\right) = \frac{13}{11}$};

		% Axes
		\draw[->] (0,0) -- (7,0) node[right=5,fixed_height] {$c\left(\frac{2}{11}\right)$};
		\draw[-] (0,0) -- (0,3.2);

		% Y-bars: Latencies
		\node[left] at (0,2.5) {$\frac{2}{11}$};
		\node[left] at (0,1.5) {$\frac{14}{11}$};
		\node[left] at (0,0.5) {$\frac{29}{11}$};
		\filldraw[fill=blue!20] (0,2.25) rectangle (2/11,2.75);
		\filldraw[fill=blue!20] (0,1.25) rectangle (14/11,1.75);
		\filldraw[fill=blue!20] (0,0.25) rectangle (29/11,0.75);

		% Y-bars: Tolls
		\filldraw[fill=orange!20] (2/11,2.25) rectangle (41/11,2.75) node[below right] {$\frac{39}{11}$};
		\filldraw[fill=orange!20] (14/11,1.25) rectangle (40/11,1.75) node[below right] {$\frac{26}{11}$};
		\filldraw[fill=orange!20] (29/11,0.25) rectangle (42/11,0.75) node[below right] {$\frac{13}{11}$};
	\end{tikzpicture}
\end{frame}

\begin{frame}{Sensitivity split | Heterogeneous networks}
	Latencies are the same for all players.

	Toll ordering is the same for all players.

	Toll differences are viewed with different scaling.
	\\[15pt]
	\begin{tikzpicture}
		% Title
		\node[above,fixed_height] at (3.7,3.3) {$\alpha(1) = 2$};

		% Axes
		\draw[->] (0,0) -- (7,0) node[right=5,fixed_height] {$c\left(1\right)$};
		\draw[-] (0,0) -- (0,3.2);

		% Y-bars: Latencies
		\node[left] at (0,2.5) {$\frac{2}{11}$};
		\node[left] at (0,1.5) {$\frac{14}{11}$};
		\node[left] at (0,0.5) {$\frac{29}{11}$};
		\filldraw[fill=blue!20] (0,2.25) rectangle (2/11,2.75);
		\filldraw[fill=blue!20] (0,1.25) rectangle (14/11,1.75);
		\filldraw[fill=blue!20] (0,0.25) rectangle (29/11,0.75);

		% Y-bars: Tolls
		\filldraw[fill=orange!20] (2/11,2.25) rectangle (64/11,2.75) node[below right] {$6$};
		\filldraw[fill=orange!20] (14/11,1.25) rectangle (58/11,1.75) node[below right] {$4$};
		\filldraw[fill=orange!20] (29/11,0.25) rectangle (51/11,0.75) node[below right] {$2$};
	\end{tikzpicture}
\end{frame}

\begin{frame}{Sensitivity split | Heterogeneous networks}
	Tolls define an ordering of the $N$ links where
	\begin{itemize}
		\item $t_1 \ge t_2 \ge \dots \ge t_n$
		\item $\ell_1(x_1(t)) \le \ell_2(x_2(t)) \le \dots \le \ell_n(x_n(t))$
		\item for player representatives from each link $p_1, p_2, \dots, p_n$ we also get $\alpha(p_1) \le \alpha(p_2) \le \dots \le \alpha(p_n)$
	\end{itemize}
	Ordering is unique only when the toll (and therefore also latency) inequalities are strict.
\end{frame}

\begin{frame}{Sensitivity split | Heterogeneous networks}
	For any pair of links $i, j$
	\begin{align*}
		c_i(&p) \le c_j(p) & c_i(p) \ge c_i(p&) \\[10pt]
		\alpha(p) \le &\frac{\ell_j(x_j) - \ell_i(x_i)}{t_i - t_j} & \frac{\ell_j(x_j) - \ell_i(x_i)}{t_i - t_j}& \le \alpha(p)
	\end{align*}
	We need the notion of the money-sensitivity value for which two link costs are equal.
\end{frame}

\begin{frame}{Sensitivity split}
	\begin{definition}
		Let $(N, \ell, \alpha)$, tolls $t$ where $t_i \ne t_j$ for links $i, j \in N$.
		We define the \textit{money sensitivity split function} for $i, j$ $\as^{(i, j)}: \{t \in \R_+^n|t_i \ne t_j\} \rightarrow (0, +\infty)$ as
		\[\as^{(i, j)}(t) = \frac{\ell_j(x_j(t)) - \ell_i(x_i(t))}{t_i - t_j}\]
	\end{definition}
	If $t_i = t_j \Leftrightarrow \ell_i(x_i) = \ell_j(x_j)$ then $c_i(p) = c_j(p)$ for \textbf{all} flow users.
\end{frame}

\begin{frame}{Sensitivity split}
	\begin{table}[h!]
		\centering
		\caption{Summary of properties for the sensitivity split.}
		\begin{tabular}{| c || c | c |}
			\hline
			& \textbf{lower split} & \textbf{upper split} \\ \hline
			$\alpha$ & $\alpha(p) \le \as^{(i, j)}(t)$ & $\alpha(p) \ge \as^{(i, j)}(t)$ \\ \hline
			link & low-latency, high-toll & high-latency, low-toll \\ \hline
			sensitivity & time $\ge$ money & time $\le$ money \\ \hline
		\end{tabular}
		\label{table:split_summary}
	\end{table}
\end{frame}

\begin{frame}{Sensitivity split}
	\begin{definition}
		Let $(N, \ell, \alpha)$.
		We define as $A_i(t), A_i: \R_+^n \rightarrow \mathcal{P}(\R_+)$ the set of all possible $\alpha(p)$ values of players $p$ using link $i$ in $x(t)$.
		\[A_i(t) = \{\alpha(p) | f(p) = j \wedge t_j = t_i\}\]
	\end{definition}
\end{frame}

\begin{frame}{Sensitivity split | Properties}
	\begin{lemma}
		Let $(N, \ell, \alpha)$ and tolls $t$ where for $i, j \in N$ $t_i \ne t_j$ and $x_i(t), x_j(t) > 0$.
		\begin{enumerate}[(i)]
			\item $\as^{(i, j)}(t) > 0$
			\item $\as^{(i, j)}(t) \in [\alpha(0), \alpha(1)]$
			\item if $\ell_i(x_i(t)) < \ell_j(x_j(t))$ then
			$\sup A_i(t) \le \as^{(i, j)}(t) \le \inf A_j(t)$
		\end{enumerate}
	\end{lemma}
\end{frame}

\begin{frame}{Sensitivity split | Properties}
	\begin{lemma}[Monotonicity]
		Let $t, t^\prime$ different only for $i, j \in N$ with the same toll ordering and $x_i(t), x_j(t), x_i(t^\prime), x_j(t^\prime) > 0$.
		Then
		\begin{enumerate}[(i)]
			\item if $\frac{t_i - t_j}{t_i^\prime - t_j^\prime} \le 1$ then $\as^{(i, j)}(t) \ge \as^{(i, j)}(t^\prime)$
			\item if $\frac{t_i - t_j}{t_i^\prime - t_j^\prime} \ge 1$ then $\as^{(i, j)}(t) \le \as^{(i, j)}(t^\prime)$
		\end{enumerate}
	\end{lemma}
\end{frame}

\begin{frame}{Sensitivity split | Properties}
	\begin{lemma}[Limit values]
		Let $(N, \ell, \alpha)$.
		Then:
		\begin{enumerate}[(i)]
			\item
			$\begin{aligned}[t]
				\lim_{|t_i - t_j| \rightarrow +\infty}\as^{(i, j)}(t) = 0
			\end{aligned}$
			\item
			$\begin{aligned}[t]
				\lim_{t_i - t_j \rightarrow 0^+} \as^{(i, j)}(t) = \limsup_{t_i - t_j \rightarrow 0^+} A_i(t)
			\end{aligned}$
		\end{enumerate}
	\end{lemma}
	\begin{corollary}
		$\begin{aligned}[t]
			\lim_{t_i - t_j \rightarrow 0^+} \as^{(i, j)}(t) = \alpha(x_i(0))
		\end{aligned}$
	\end{corollary}
\end{frame}

\begin{frame}{Sensitivity split | Properties}
	\begin{center}
		\begin{multicols}{2}
			% Left Column: Network Graph
			\begin{tikzpicture}[scale=1.2, baseline=40pt, every node/.style={font=\small}]
				% Nodes
				\node[circle, draw, fill=blue!20, minimum size=0.8cm] (s) at (0,0) {$s$};
				\node[circle, draw, fill=blue!20, minimum size=0.8cm] (t) at (4,0) {$t$};

				% Links
				\draw[->, thick] (s) to[bend left=35] node[midway, above] {$2x$} (t);
				\draw[->, thick] (s) to[bend right=35] node[midway, below] {$x + 1$} (t);
			\end{tikzpicture}

			% Right Column: Distribution function Graph
			\begin{tikzpicture}[x=2.5cm, y=1.2cm, every node/.style={font=\small}]
				% Axes
				\draw[->] (0,0) -- (1.2,0) node[right] {$p$};
				\draw[->] (0,0) -- (0,2.2) node[above] {$\as(p)$};

				% Tick marks
				\draw (0,0) -- (0,-0.05) node[below] {$0$};
				\draw (1,0) -- (1,-0.05) node[below] {$1$};
				\draw (0,1) -- (-0.05,1) node[left] {$1$};
				\draw (0,2) -- (-0.05,2) node[left] {$2$};

				% Function graph
				\draw[thick] (0,1) -- (1,2);

				% Description
				\node[align=center] at (0.7, -0.8) {$\alpha(p) = p + 1$};
			\end{tikzpicture}
		\end{multicols}
	\end{center}
	\[x(0) = \left(\tfrac23, \tfrac13\right) \qquad \alpha(x_1(0)) = \tfrac53 \quad \alpha(x_2(0)) = \tfrac43\]
\end{frame}

\begin{frame}{Sensitivity split | Properties}
	\begin{figure}
		\centering
		\begin{tikzpicture}[x=2cm, y=2cm, every node/.style={font=\small}]
			% Axes
			\draw[->] (0,0) -- (-1.1,0);
			\draw[->] (0,0) -- (2.1,0) node[right] {$t_1 - t_2$};
			\draw[->] (0,0) -- (0,1.2) node[above] {$x_1(t)$};

			% Tick marks
			\draw (-1,0) -- (-1,-0.05) node[below] {$-1$};
			\draw (0,0) -- (0,-0.05) node[below] {$t_2$};
			\draw (2,0) -- (2,-0.05) node[below] {$2$};
			\draw (0,1) -- (-0.05, 1) node[left] {$1$};

			% Function graph
			\draw[thick,domain=-1:0] plot (\x,{(2-2*\x)/(3-\x)});
			\draw[thick,domain=0:2] plot (\x,{(2-\x)/(3+\x)});

			% Dots
			\fill[black] (0,2/3) ellipse (0.03 and 0.03);
			\draw[black] (0,2/3) ellipse (0.03 and 0.03);
			\node[below left] at (0,2/3) {$\frac23$};
		\end{tikzpicture}
		\caption{Flow function of link operator $1$ of Example}
		\label{figure:alpha_simple:flow_example}
	\end{figure}
\end{frame}

\begin{frame}{Sensitivity split | Properties}
	\begin{figure}
		\centering
		\begin{tikzpicture}[x=2cm, y=2cm, every node/.style={font=\small}]
			% Axes
			\draw[->] (0,0) -- (-1.1,0);
			\draw[->] (0,0) -- (2.1,0) node[right] {$t_1 - t_2$};
			\draw[->] (0,0) -- (0,2.2) node[above] {$\as(t)$};

			% Tick marks
			\draw (-1,0) -- (-1,-0.05) node[below] {$-1$};
			\draw (0,0) -- (0,-0.05) node[below] {$t_2$};
			\draw (2,0) -- (2,-0.05) node[below] {$2$};
			\draw (0,2) -- (-0.05, 2) node[left] {$2$};
			\draw (0,4/3) -- (0.05,4/3) node[right] {$\frac43$};
			\draw (0,5/3) -- (-0.05,5/3) node[left] {$\frac53$};
			\draw (0,1) -- (-0.05,1) node[left] {$1$};

			% Function graph
			\draw[thick,domain=-1:0] plot (\x,{4/(3-\x)});
			\draw[thick,domain=0:2] plot (\x,{5/(3+\x)});

			% Dots
			\fill[white] (0,4/3) ellipse (0.03 and 0.03);
			\draw[black] (0,4/3) ellipse (0.03 and 0.03);
			\fill[white] (0,5/3) ellipse (0.03 and 0.03);
			\draw[black] (0,5/3) ellipse (0.03 and 0.03);
		\end{tikzpicture}
		\caption{Split function of Example}
		\label{figure:alpha_simple:split_example}
	\end{figure}
\end{frame}

\begin{frame}{Sensitivity split | Upper bound}
	We observe that $\as$ might never take some upper values of $\alpha$.\\
	Are then those values relevant?

	\begin{theorem}
		Let $\Gm_1 = (N, \ell, \alpha^{(1)}), \Gm_2 = (N, \ell, \alpha^{(2)})$.\\
		For $x(0) = x^{(1)}(0) = x^{(2)}(0)$, if
		\[\alpha^{(1)}(p) = \alpha^{(2)}(p), \forall p: p < 1 - \min_{i \in N}\{x_i(0)\}\]
		then $\Gm_1$ and $\Gm_2$ are equivalent.
	\end{theorem}
\end{frame}

\begin{frame}{Sensitivity split | Upper bound}
	For the Example we have
	\[\max\{x_1(0), x_2(0)\} = \max\left\{\tfrac23, \tfrac13\right\} = \tfrac23\]
	so if we used a distribution function such as
	\[
		\alpha(p) =
		\begin{cases}
			p + 1 & p \le \tfrac23 \\
			\mathrm{e}^p & p > \tfrac23
		\end{cases}
	\]
	then $\as(t)$ would remain the same for all $t \in \R_+^n$
\end{frame}


\section{Pseudo-heterogeneous Pricing Games}

\begin{frame}{Pseudo-heterogeneous Pricing Games}
	Consider distribution functions $\alpha(p) = c$ for some $c > 0$.

	Players again all view link costs the same.
	\[c_i(p) = \ell_i(x_i) + c t_i\]
	Behaviour should be homogeneous with tolls "mapped" to $t \rightarrow t/c$.

	Also note that 
	\[c_i(p) = c \left(\frac1c \ell_i(x_i) + t_i\right)\]
	where $\frac1c \ell_i(x_i) = \frac{a_i}{c} x_i + \frac{b_i}{c}$ affine.
\end{frame}

\begin{frame}{Pseudo-heterogeneous Pricing Games}
	\begin{lemma}
		Let $\Gm_1 = (N, \ell^{(1)}, \alpha)$ where $\forall p \in [0, 1] \; \alpha(p) = c$ for some $c \in \R_+$ and $\Gm_2 = (N, \ell^{(2)}, 1)$ where $\ell^{(2)} = \frac{\ell^{(1)}}{c}$ for the same $c$.
		Then the two games are equivalent.
	\end{lemma}
	\begin{proof}
		For any $t \in \R_+^n$ and any flow $x_i$
		\[
			\sum_{i \in N} \left(\ell_i^{(1)}(x_i^{(2)}(t)) + c t_i\right) \left(x_i^{(2)}(t) - x_i\right) \leq 0
		\]
		$x^{(2)}(t)$ is the unique N.E. for $t$ in $\Gm_1$ so $x^{(1)}(t) = x^{(2)}(t)$.
	\end{proof}
	Costs are proportional.
\end{frame}

\begin{frame}{Pseudo-heterogeneous Pricing Games}
	We will call such games \textit{pseudo-heterogeneous}.

	Uniform increase in money-sensitivity is equivalent to uniform decrease in latencies (lost for general $\alpha$ functions).

	\begin{lemma}
		Let $\Gm_1 = \left(N, \frac{\ell}{c_1}, 1\right), \Gm_2 = \left(N, \frac{\ell}{c_2}, 1\right)$.
		Then $x^{(1)}(c_2 t) = x^{(2)}(c_1 t)$.
	\end{lemma}
	\begin{proof}
		\[
			\begin{aligned}[t]
				&\sum_{i \in N} \left(\frac{1}{c_1}\ell_i(x_i^{(1)}(c_2 t)) + c_2 t_i\right) \left(x_i^{(1)}(c_2 t) - x_i\right) \le 0\\
				\Leftrightarrow &\sum_{i \in N} \left(\frac{1}{c_2}\ell_i(x_i^{(1)}(c_2 t)) + c_1 t_i\right) \left(x_i^{(1)}(c_2 t) - x_i\right) \le 0
			\end{aligned}
		\]
	\end{proof}
\end{frame}

\begin{frame}{Pseudo-heterogeneous Pricing Games}
\begin{lemma}
	Let $\Gm_1 = (N, \ell, c_1)$ and $\Gm_2 = (N, \ell, c_2)$, $c_1, c_2 > 0$.
	Then
	\begin{enumerate}[$(i)$]
		\item $x^{(1)}(c_2 t) = x^{(2)}(c_1 t)$
		\item $\frac{\Pi_i^{(1)}(c_2 t)}{\Pi_i^{(2)}(c_1 t)} = \frac{c_2}{c_1}$
		\item $\frac{B_i^{(1)}(c_2t_{-i})}{B_i^{(2)}(c_1 t_{-i})} = \frac{c_2}{c_1}$
	\end{enumerate}
\end{lemma}
\begin{proof}
	\begin{enumerate}[$(i)$]
		\item Consider $\Gm_3 = (N, \ell/c_1, 1), \Gm_4 = (N, \ell/c_2, 1)$.\\
		Then $x^{(1)}(c_2 t) = x^{(3)}(c_2 t) = x^{(4)}(c_1 t) = x^{(2)}(c_1 t)$.
		\item and $(iii)$ follow from definitions.
		\vspace{-20pt}
	\end{enumerate}
\end{proof}
\end{frame}

\begin{frame}{Pseudo-heterogeneous Pricing Games}
	\begin{lemma}
		Let $\Gm_1 = (N, \ell, c_1)$ and $\Gm_2 = (N, \ell, c_2)$, $c_1, c_2 > 0$.
		Then
		\begin{enumerate}[$(i)$]
			\item $x^{(1)}(t) = x^{(2)}\left(\frac{c_1}{c_2} t\right)$
			\item $\Pi_i^{(1)}(t) = \frac{c_2}{c_1} \Pi_i^{(2)}\left(\frac{c_1}{c_2} t\right)$
			\item $B_i^{(1)}(t_{-i}) = \frac{c_2}{c_1} B_i^{(2)}\left(\frac{c_1}{c_2} t_{-i}\right)$
		\end{enumerate}
	\end{lemma}
\end{frame}

\begin{frame}{Pseudo-heterogeneous Pricing Games}
	\begin{figure}
		\centering
		\begin{tikzpicture}[x=3cm, y=3cm, every node/.style={font=\small}]
			% Axes
			\draw[->] (0,0) -- (-1.1,0);
			\draw[->] (0,0) -- (2.1,0) node[right] {$t_1 - t_2$};
			\draw[->] (0,0) -- (0,1.1) node[above] {$x_1(t)$};

			% Tick marks
			\draw (-1,0) -- (-1,-0.05) node[below] {$-1$};
			\draw (-1/3,0) -- (-1/3,-0.05) node[left=9,below] {$-\frac13$};
			\draw (-1/4,0) -- (-1/4,-0.05) node[right=3,below] {$-\frac14$};
			\draw (0,0) -- (0,-0.05) node[below] {$t_2$};
			\draw (2/4,0) -- (2/4,-0.05) node[below] {$\frac12$};
			\draw (2/3,0) -- (2/3,-0.05) node[below] {$\frac23$};
			\draw (2,0) -- (2,-0.05) node[below] {$2$};
			\draw (0,1) -- (0.05,1) node[right] {$1$};
			\draw (0,2/3) -- (-0.05,2/3) node[below=4,left] {$\frac23$};

			% Flow function graphs
			\draw[thick, blue!33] (-1,1) -- (2,0) node[above=20,left=5,black] {$\alpha = 1$};
			\draw[thick, blue!66] (-1/3,1) -- (2/3,0) node[above right,black] {$\alpha = 3$};
			\draw[thick, blue!100] (-1/4,1) -- (2/4,0) node[black] at (0.2,0.17) {$\alpha = 4$};

			% Example graph in corner
			\begin{scope}[shift={(1.5,1)},scale=0.2]
				% Nodes
				\node[circle, draw, fill=blue!20, minimum size=0.4cm] (s) at (0,0) {$s$};
				\node[circle, draw, fill=blue!20, minimum size=0.4cm] (t) at (4,0) {$t$};

				% Links
				\draw[->, thick] (s) to[bend left=35] node[midway, above] {$2x$} (t);
				\draw[->, thick] (s) to[bend right=35] node[midway, below] {$x + 1$} (t);
			\end{scope}
		\end{tikzpicture}
		\caption{Flow function for link $1$ of Example for $\alpha(p) = 1, 3, 4$.}
	\end{figure}
\end{frame}

\begin{frame}{Pseudo-heterogeneous Pricing Games}
	\begin{lemma}
		Let $\Gm_1 = (N, \ell, c_1)$ and $\Gm_2 = (N, \ell, c_2)$, $c_1, c_2 > 0$.
		Then
		\begin{enumerate}[$(i)$]
			\item $t^{*(1)} = \frac{c_2}{c_1} t^{*(2)}$
			\item $x^{(1)}(t^{*(1)}) = x^{(2)}(t^{*(2)})$
		\end{enumerate}
	\end{lemma}
	\begin{proof}
		\begin{enumerate}[$(i)$]
			% https://tex.stackexchange.com/a/98399
			\item
			$\begin{aligned}[t]
				t_i^{*(1)} &= B_i^{(1)}(t_{-i}^{*(1)}) \Leftrightarrow \cdots \Leftrightarrow \frac{c_1}{c_2} t_i^{*(1)} = B_i^{(2)}\left(\frac{c_1}{c_2} t_{-i}^{*(1)}\right)
			\end{aligned}$
			\item follows from previous lemma.
			\vspace{-20pt}
		\end{enumerate}
	\end{proof}
	Pricing Nash Equilibrium is always at the same flow.
\end{frame}

\section{Step distribution functions}

\begin{frame}{Heterogeneous Pricing Games}
	Zero money-sensitivity creates issues as it can be exploited.
	\begin{lemma}
		Let $(N, \ell, \alpha)$ where $\alpha(p) = 0$ for $p \in [0, \epsilon]$ for some $\epsilon > 0$.\\
		Then $\exists i \in N$ where $B_i(t_{-i})$ is unbounded for all $t_{-i} \in \R_+^{n - 1}$.
	\end{lemma}
	\begin{proof}
		Links with $\ell_i(0) = \min_{j \in N} \ell_i(0)$ can select $t_i > \max t_{-i}$ and become the lowest-latency link, with players $\alpha(p) = 0$.

		Those players will never leave as $t_i \rightarrow +\infty$, making $x_i$ fixed and therefore $\lim_{t_i \rightarrow +\infty} \Pi_i(t_i, t_{-i}) = +\infty$.
	\end{proof}
\end{frame}

\begin{frame}{Step distribution functions}
	We will limit our model to instances with $2$ links.

	In $2$-link games:
	\begin{itemize}
		\item $\as^{(1, 2)}(t) = \as(t)$
		\item There always exists an $\alpha$ for which all link costs are equal.
	\end{itemize}
\end{frame}

\begin{frame}{Step distribution functions}
	\begin{lemma}
		Let $\Gm_1 = ([2], \ell, \alpha^{(1)})$, fixed tolls $t \ne 0$ and $\Gm_2 = ([2], \ell, \alpha^{(2)})$ with fixed $\alpha^{(2)}(p) = \as^{(1)}(t)$.
		Then
		\begin{enumerate}[$(i)$]
			\item $x^{(1)}(t) = x^{(2)}(t)$
			\item $\Pi_i^{(1)}(t) = \Pi_i^{(2)}(t)$ for each respective link operator $i$
		\end{enumerate}
	\end{lemma}
\end{frame}

\begin{frame}{Step distribution functions}
	\begin{example}
		Let $([2], \ell, \alpha)$ with
		$
			\alpha(p) =
			\begin{cases}
				1 & p \le \tfrac15 \\
				3 & p \in \left(\tfrac15, \tfrac12\right) \\
				4 & p \ge \tfrac12
			\end{cases}
		$
	\end{example}

	\begin{center}
		\begin{multicols}{2}
			% Left Column: Network Graph
			\begin{tikzpicture}[scale=1, baseline=50pt, every node/.style={font=\small}]
				% Nodes
				\node[circle, draw, fill=blue!20, minimum size=0.8cm] (s) at (0,0) {$s$};
				\node[circle, draw, fill=blue!20, minimum size=0.8cm] (t) at (4,0) {$t$};

				% Links
				\draw[->, thick] (s) to[bend left=35] node[midway, above] {$2x$} (t);
				\draw[->, thick] (s) to[bend right=35] node[midway, below] {$x + 1$} (t);
			\end{tikzpicture}

			% Right Column: Distribution function Graph
			\begin{tikzpicture}[x=3cm, y=0.7cm, every node/.style={font=\small}]
				% Axes
				\draw[->] (0,0) -- (1.2,0) node[right] {$p$};
				\draw[->] (0,0) -- (0,4.2) node[above] {$\as(p)$};

				% Tick marks
				\draw (0,0) -- (0,-0.05) node[below] {$0$};
				\draw (0.2,0) -- (0.2,-0.05) node[below] {$\frac15$};
				\draw (0.5,0) -- (0.5,-0.05) node[below] {$\frac12$};
				\draw (1,0) -- (1,-0.05) node[below] {$1$};
				\draw (0,1) -- (-0.05,1) node[left] {$1$};
				\draw (0,3) -- (-0.05,3) node[left] {$3$};
				\draw (0,4) -- (-0.05,4) node[left] {$4$};

				% Function graph
				\draw[thick] (0,1) -- (0.2,1);
				\draw[thick] (0.2,3) -- (0.5,3);
				\draw[thick] (0.5,4) -- (1,4);

				% Closed / Open dots
				\fill[black] (0.2,1) ellipse (0.014 and 0.06);
				\draw[black] (0.2,1) ellipse (0.014 and 0.06);
				\fill[white] (0.2,3) ellipse (0.014 and 0.06);
				\draw[black] (0.2,3) ellipse (0.014 and 0.06);
				\fill[white] (0.5,3) ellipse (0.014 and 0.06);
				\draw[black] (0.5,3) ellipse (0.014 and 0.06);
				\fill[black] (0.5,4) ellipse (0.014 and 0.06);
				\draw[black] (0.5,4) ellipse (0.014 and 0.06);
			\end{tikzpicture}
		\end{multicols}
	\end{center}
\end{frame}

\begin{frame}{Step distribution functions}
	\begin{figure}
		\centering
		\begin{tikzpicture}[x=3cm, y=3cm, every node/.style={font=\small}]
			% Axes
			\draw[->] (0,0) -- (-1.1,0);
			\draw[->] (0,0) -- (2.1,0) node[right] {$t_1 - t_2$};
			\draw[->] (0,0) -- (0,1.1) node[above] {$x_1(t)$};

			% Tick marks
			\draw (-1,0) -- (-1,-0.05) node[below] {$-1$};
			\draw (-2/5,0) -- (-2/5,-0.05) node[below] {$-\frac25$};
			\draw (-2/15,0) -- (-2/15,-0.05) node[below] {$-\frac{2}{15}$};
			\draw (0,0) -- (0,-0.05) node[below] {$t_2$};
			\draw (1/8,0) -- (1/8,-0.05) node[below] {$\frac18$};
			\draw (1/6,0) -- (1/6,-0.05) node[below] {$\frac16$};
			\draw (7/15,0) -- (7/15,-0.05) node[below] {$\frac{7}{15}$};
			\draw (7/5,0) -- (7/5,-0.05) node[below] {$\frac75$};
			\draw (2,0) -- (2,-0.05) node[below] {$2$};
			\draw (0,1) -- (-0.05,1) node[left] {$1$};
			\draw (0,4/5) -- (0.05,4/5) node[right] {$\frac45$};
			\draw (0,2/3) -- (-0.05,2/3) node[left] {$\frac23$};
			\draw (0,1/2) -- (-0.05,1/2) node[left] {$\frac12$};
			\draw (0,1/5) -- (-0.05,1/5) node[left] {$\frac15$};

			% Function graph
			% Base flow functions
			\draw[gray,dashed, very thin] (-1,1) -- (2,0);
			\draw[gray,dashed, very thin] (-1/3,1) -- (2/3,0);
			\draw[gray,dashed, very thin] (-1/4,1) -- (2/4,0);

			% Left side
			\draw[thick] (-1,1) -- (-2/5,4/5);
			\draw[thick] (-2/5,4/5) -- (-2/15,4/5);
			\draw[thick] (-2/15,4/5) -- (0,2/3);
			% Right side
			\draw[thick] (0,2/3) -- (1/8,1/2);
			\draw[thick] (1/8,1/2) -- (1/6,1/2);
			\draw[thick] (1/6,1/2) -- (7/15,1/5);
			\draw[thick] (7/15,1/5) -- (7/5,1/5);
			\draw[thick] (7/5,1/5) -- (2,0);
		\end{tikzpicture}
		\caption{Flow function for link $1$ of Example}
		\label{figure:alpha_step:flow}
	\end{figure}
\end{frame}

\begin{frame}{Step distribution functions}
	\begin{figure}
		\centering
		\begin{tikzpicture}[x=3.4cm, y=1cm, scale=0.8, every node/.style={font=\small}]
			% Axes
			\draw[->] (0,0) -- (-1.1,0);
			\draw[->] (0,0) -- (2.1,0) node[right] {$t_1 - t_2$};
			\draw[->] (0,0) -- (0,4.4) node[above] {$\as(t)$};

			% Tick marks
			\draw (-1,0) -- (-1,-0.05) node[below] {$-1$};
			\draw (-2/5,0) -- (-2/5,-0.05) node[below] {$-\frac25$};
			\draw (-2/15,0) -- (-2/15,-0.05) node[below] {$-\frac{2}{15}$};
			\draw (0,0) -- (0,-0.05) node[below] {$t_2$};
			\draw (1/8,0) -- (1/8,-0.05) node[below] {$\frac18$};
			\draw (1/6,0) -- (1/6,-0.05) node[below] {$\frac16$};
			\draw (7/15,0) -- (7/15,-0.05) node[below] {$\frac{7}{15}$};
			\draw (7/5,0) -- (7/5,-0.05) node[below] {$\frac75$};
			\draw (2,0) -- (2,-0.05) node[below] {$2$};
			\draw (0,1) -- (-0.05,1) node[left] {$1$};
			\draw (0,3) -- (0.05,3) node[right] {$3$};
			\draw (0,4) -- (-0.05,4) node[left] {$4$};

			% Function graph
			% Left side
			\draw[thick] (-1,1) -- (-2/5,1);
			\draw[thick] (-2/5,1) -- (-2/15,3);
			\draw[thick] (-2/15,3) -- (0,3);
			% Right side
			\draw[thick] (0,4) -- (1/8,4);
			\draw[thick] (1/8,4) -- (1/6,3);
			\draw[thick] (1/6,3) -- (7/15,3);
			\draw[thick] (7/15,3) -- (7/5,1);
			\draw[thick] (7/5,1) -- (2,1);

			% Dots
			\fill[white] (0,3) ellipse (0.02 and 0.066);
			\draw[black] (0,3) ellipse (0.02 and 0.066);
			\fill[white] (0,4) ellipse (0.02 and 0.066);
			\draw[black] (0,4) ellipse (0.02 and 0.066);
		\end{tikzpicture}
		\caption{Split function of Example}
		\label{figure:alpha_step:split}
	\end{figure}
\end{frame}

\begin{frame}<presentation:0>[noframenumbering]{Step distribution functions}
	Notice that when $x_1$ is fixed then $\as$ isn't and vice-versa.\\
	If the flow is moving, then it does so among players with the same sensitivity type, which is one of the finitely many values $\alpha$ can take.
	If all players of each sensitivity type are using a single link, then $\as$ can take any value between the lowest and highest $\alpha$ value in the upper and lower split respectively.
\end{frame}

\begin{frame}{Step distribution functions}
	Let $\Gm_1 = ([2], \ell, \alpha)$ where $\alpha = (\alpha_k)_{k \in [m]}$.
	\begin{lemma}
		Let $t \ne 0$ where all players with $\alpha(p) = \alpha_k$ are using the same link for all $k \in [m]$.
		Then $t$ is not a Nash equilibrium for the pricing game.
	\end{lemma}
	\begin{theorem}
		If $\Gm_1$ has $t^*$, let $\Gm_2 = ([2], \ell, \as(t^*))$ with fixed $\alpha^{(2)}(p) = \as(t^*)$.
		Then $t^*$ is a Nash equilibrium for the pricing game in $\Gm_2$.
	\end{theorem}
\end{frame}

\begin{frame}{Step distribution functions}
	\begin{example}
		Let $\Gm = ([2], \ell, \alpha)$ a heterogeneous pricing game with latency functions $\ell_1(x) = 2x, \ell_2(x) = x + 1$ and distribution function $\alpha$ such that
		\begin{align*}
			\alpha(p) =
			\begin{cases}
				3 & p \le \tfrac14 \\
				4 & p > \tfrac14
			\end{cases}&
		\end{align*}
		Then $\Gm$ has a N.E. for the pricing game at $t^* = \left(\tfrac{5}{12}, \tfrac{4}{12}\right)$.
	\end{example}
\end{frame}

\begin{frame}{Step distribution functions}
	\begin{figure}
		\centering
		\begin{tikzpicture}[x=8cm, y=10cm, scale=0.7, every node/.style={font=\small}]
			% Axes
			\draw[->] (0,0) -- (-0.4,0);
			\draw[->] (0,0) -- (0.7,0) node[right] {$t_1 - t_2$};
			\draw[->] (0,0) -- (0,0.3) node[above] {$\Pi_1(t)$};

			% Tick marks
			\draw (-1/3,0) -- (-1/3,-0.01) node[below] {$-\frac13$};
			\draw (-1/4,0) -- (-1/4,-0.01) node[below] {$-\frac14$};
			\draw (-1/12,0) -- (-1/12,-0.01) node[left=9,below] {$-\frac{1}{12}$};
			\draw (-1/16,0) -- (-1/16,-0.01) node[right=2,below] {$-\frac{1}{16}$};
			\draw (0,0) -- (0,-0.01) node[below] {$0$} node[below=20] {$t_2^* = \frac{4}{12}$};
			\draw (1/12,0) -- (1/12,-0.01) node[below] {$\frac{1}{12}$};
			\draw (5/16,0) -- (5/16,-0.01) node[below] {$\frac{5}{16}$};
			\draw (5/12,0) -- (5/12,-0.01) node[below] {$\frac{5}{12}$};
			\draw (2/4,0) -- (2/4,-0.01) node[below] {$\frac24$};
			\draw (2/3,0) -- (2/3,-0.01) node[below] {$\frac23$};
			%\draw (0,2/9) -- (-0.02,2/9) node[left] {$\frac29$};
			%\draw (0,1/12) -- (0.01,1/12) node[right] {$\frac{1}{12}$};

			% Pseudo-heterogeneous profits
			\draw[gray,dashed,thick,domain=-1/3:2/3] plot (\x,{((2-3*\x)/3)*(1/3+\x)}) node[above=17,right=-7,black] {$c = 3$};
			\draw[gray,dashed,thick,domain=-1/4:2/4] plot (\x,{((2-4*\x)/3)*(1/3+\x)}) node[above=10,right=-3,black] {$c = 4$};

			% Profit function
			\draw[thick,domain=-1/3:-1/12] plot (\x,{((2-3*\x)/3)*(4/12+\x)});
			\draw[thick] (-1/12,3/16) -- (-1/16, 39/192);
			\fill[black] (-1/12,3/16) ellipse (0.00625 and 0.005) node[below right] {$\frac{3}{16}$};
			\fill[black] (-1/16,39/192) ellipse (0.00625 and 0.005) node[above left] {$\frac{39}{192}$};
			\draw[thick,domain=-1/16:5/16] plot (\x,{((2-4*\x)/3)*(4/12+\x)});
			\draw[thick] (5/16,31/192) -- (5/12,3/16);
			\fill[black] (5/16,31/192) ellipse (0.00625 and 0.005) node[below left] {$\frac{31}{192}$};
			\fill[black] (5/12,3/16) ellipse (0.00625 and 0.005) node[above right] {$\frac{3}{16}$};
			\draw[thick,domain=5/12:2/3] plot (\x,{((2-3*\x)/3)*(4/12+\x)});

			% Max point
			\fill[red] (1/12, 25/108) ellipse (0.00625 and 0.005) node[below,black] {$\frac{25}{108}$};
		\end{tikzpicture}
		\begin{tikzpicture}[x=10cm, y=15cm, scale=0.7, every node/.style={font=\small}]
			% Axes
			\draw[->] (0,0) -- (-0.55,0);
			\draw[->] (0,0) -- (0.4,0) node[right] {$t_2 - t_1$};
			\draw[->] (0,0) -- (0,0.2) node[above] {$\Pi_2(t)$};

			% Tick marks
			%\draw (-2/3,0) -- (-2/3,-0.01) node[below] {$-\frac23$};
			\draw (-2/4,0) -- (-2/4,-0.01) node[below] {$-\frac24$};
			\draw (-5/12,0) -- (-5/12,-0.01) node[below] {$-\frac{5}{12}$};
			\draw (-5/16,0) -- (-5/16,-0.01) node[below] {$-\frac{5}{16}$};
			\draw (-1/12,0) -- (-1/12,-0.01) node[below] {$-\frac{1}{12}$};
			\draw (0,0) -- (0,-0.01) node[below] {$0$} node[below=20] {$t_1^* = \frac{5}{12}$};
			\draw (1/16,0) -- (1/16,-0.01) node[left=3,below] {$\frac{1}{16}$};
			\draw (1/12,0) -- (1/12,-0.01) node[right=3,below] {$\frac{1}{12}$};
			\draw (1/4,0) -- (1/4,-0.01) node[below] {$\frac14$};
			\draw (1/3,0) -- (1/3,-0.01) node[below] {$\frac13$};
			%\draw (0,5/36) -- (-0.02,5/36) node[below=7,left] {$\frac{5}{36}$};

			% Pseudo-heterogeneous profits
			\draw[gray,dashed,thick,domain=-5/12:1/3] plot (\x,{((1-3*\x)/3)*(5/12+\x)}) node[above=15,right=-7,black] {$c = 3$};
			\draw[gray,dashed,thick,domain=-5/12:1/4] plot (\x,{((1-4*\x)/3)*(5/12+\x)}) node[above=15,left=17,black] {$c = 4$};

			% Profit function
			\draw[thick] (-5/12,0) -- (-5/16,5/64);
			\fill[black] (-5/12,0) ellipse (0.0045 and 0.003) node[above left] {$0$};
			\fill[black] (-5/16,5/64) ellipse (0.0045 and 0.003) node[above left] {$\frac{5}{64}$};
			\draw[thick,domain=-5/16:1/16] plot (\x,{((1-4*\x)/3)*(5/12+\x)});
			\draw[thick] (1/16,23/192) -- (1/12,1/8);
			\fill[black] (1/16,23/192) ellipse (0.0045 and 0.003) node[below left] {$\frac{23}{192}$};
			\fill[black] (1/12,1/8) ellipse (0.0045 and 0.003) node[above right] {$\frac18$};
			\draw[thick,domain=1/12:1/3] plot (\x,{((1-3*\x)/3)*(5/12+\x)});

			% Max point
			\fill[red] (-1/12, 4/27) ellipse (0.0045 and 0.003) node[above,black] {$\frac{4}{27}$};
		\end{tikzpicture}
		\caption{Profit functions for link operators $1$ and $2$ of Example when opposed against the candidate Nash Equilibrium $t^* = \left(\tfrac{5}{12}, \tfrac{4}{12}\right)$. Red dot indicates the max profit at $t^*$. Black dots indicate tolls when relevant intervals start or end for the respective pseudo-heterogeneous profits (dashed lines) they correspond.}
		\label{figure:alpha_step:profit_ne}
	\end{figure}
\end{frame}

\begin{frame}{Step distribution functions}
	\begin{example}
		Let $\Gm = ([2], \ell, \alpha)$ a heterogeneous pricing game with latency functions $\ell_1(x) = 2x, \ell_2(x) = x + 1$ and distribution function $\alpha$ such that
		\begin{align*}
			\alpha(p) =
			\begin{cases}
				2 & p \le \tfrac14 \\
				4 & p > \tfrac14
			\end{cases}&
		\end{align*}
		Then $\Gm$ has no N.E. for the pricing game.
	\end{example}
\end{frame}

\begin{frame}{Step distribution functions}
	\begin{figure}
		\centering
		\begin{tikzpicture}[x=6cm, y=10cm, scale=0.7, every node/.style={font=\small}]
			% Axes
			\draw[->] (0,0) -- (-0.4,0);
			\draw[->] (0,0) -- (1.1,0) node[right] {$t_1 - t_2$};
			\draw[->] (0,0) -- (0,0.3) node[above] {$\Pi_1(t)$};

			% Tick marks
			\draw (-1/3,0) -- (-1/3,-0.01) node[below] {$-\frac13$};
			\draw (-1/4,0) -- (-1/4,-0.01) node[below] {$-\frac14$};
			\draw (-1/8,0) -- (-1/8,-0.01) node[left=5,below] {$-\frac18$};
			\draw (-1/16,0) -- (-1/16,-0.01) node[below] {$-\frac{1}{16}$};
			\draw (0,0) -- (0,-0.01) node[below] {$0$} node[below=20] {$t_2^* = \frac{4}{12}$};
			\draw (1/12,0) -- (1/12,-0.01) node[below] {$\frac{1}{12}$};
			\draw (5/16,0) -- (5/16,-0.01) node[below] {$\frac{5}{16}$};
			\draw (2/4,0) -- (2/4,-0.01) node[below] {$\frac24$};
			\draw (5/8,0) -- (5/8,-0.01) node[below] {$\frac58$};
			\draw (1,0) -- (1,-0.01) node[below] {$1$};
			%\draw (0,2/9) -- (-0.02,2/9) node[left] {$\frac29$};
			%\draw (0,1/12) -- (0.01,1/12) node[right] {$\frac{1}{12}$};

			% Pseudo-heterogeneous profits
			\draw[gray,dashed,thick,domain=-1/3:2/2] plot (\x,{((2-2*\x)/3)*(1/3+\x)}) node[above=20,right=-5,black] {$c = 2$};
			\draw[gray,dashed,thick,domain=-1/4:2/4] plot (\x,{((2-4*\x)/3)*(1/3+\x)}) node[above=15,right,black] {$c = 4$};

			% Profit function
			\draw[thick,domain=-1/3:-1/8] plot (\x,{((2-2*\x)/3)*(4/12+\x)});
			\draw[thick] (-1/8, 5/32) -- (-1/16, 39/192);
			\fill[black] (-1/8, 5/32) ellipse (0.008333 and 0.005) node[below right] {$\frac{5}{32}$};
			\fill[black] (-1/16,39/192) ellipse (0.008333 and 0.005) node[above left] {$\frac{39}{192}$};
			\draw[thick,domain=-1/16:5/16] plot (\x,{((2-4*\x)/3)*(4/12+\x)});
			\draw[thick] (5/16,31/192) -- (5/8,23/96);
			\fill[black] (5/16,31/192) ellipse (0.008333 and 0.005) node[below left] {$\frac{31}{192}$};
			\fill[black] (5/8,23/96) ellipse (0.008333 and 0.005) node[above right] {$\frac{23}{96}$};
			\draw[thick,domain=5/8:1] plot (\x,{((2-2*\x)/3)*(4/12+\x)});

			% Max point
			\fill[red] (1/12, 25/108) ellipse (0.008333 and 0.005) node[below,black] {$\frac{25}{108}$};
		\end{tikzpicture}
		\begin{tikzpicture}[x=8cm, y=15cm, scale=0.7, every node/.style={font=\small}]
			% Axes
			\draw[->] (0,0) -- (-0.55,0);
			\draw[->] (0,0) -- (0.6,0) node[right] {$t_2 - t_1$};
			\draw[->] (0,0) -- (0,0.2) node[above] {$\Pi_2(t)$};

			% Tick marks
			%\draw (-2/3,0) -- (-2/3,-0.01) node[below] {$-\frac23$};
			\draw (-2/4,0) -- (-2/4,-0.01) node[below] {$-\frac24$};
			\draw (-5/12,0) -- (-5/12,-0.01) node[below] {$-\frac{5}{12}$};
			\draw (-5/16,0) -- (-5/16,-0.01) node[below] {$-\frac{5}{16}$};
			\draw (-1/12,0) -- (-1/12,-0.01) node[below] {$-\frac{1}{12}$};
			\draw (0,0) -- (0,-0.01) node[below] {$0$} node[below=20] {$t_1^* = \frac{5}{12}$};
			\draw (1/16,0) -- (1/16,-0.01) node[left=3,below] {$\frac{1}{16}$};
			\draw (1/12,0) -- (1/12,-0.01) node[right=3,below] {$\frac{1}{12}$};
			\draw (1/4,0) -- (1/4,-0.01) node[below] {$\frac14$};
			\draw (1/2,0) -- (1/2,-0.01) node[below] {$\frac12$};
			%\draw (0,5/36) -- (-0.02,5/36) node[below=7,left] {$\frac{5}{36}$};

			% Pseudo-heterogeneous profits
			\draw[gray,dashed,thick,domain=-5/12:1/2] plot (\x,{((1-2*\x)/3)*(5/12+\x)}) node[above=20,right=-5,black] {$c = 2$};
			\draw[gray,dashed,thick,domain=-5/12:1/4] plot (\x,{((1-4*\x)/3)*(5/12+\x)}) node[above=15,right,black] {$c = 4$};

			% Profit function
			\draw[thick] (-5/12,0) -- (-5/16,5/64);
			\fill[black] (-5/16,5/64) ellipse (0.008 and 0.004) node[above left] {$\frac{5}{64}$};
			\draw[thick,domain=-5/16:1/16] plot (\x,{((1-4*\x)/3)*(5/12+\x)});
			\draw[thick] (1/16,23/192) -- (1/8,13/96);
			\fill[black] (1/16,23/192) ellipse (0.008 and 0.004) node[left=6,below] {$\frac{23}{192}$};
			\fill[black] (1/8,13/96) ellipse (0.008 and 0.004) node[above right] {$\frac{13}{96}$};
			\draw[thick,domain=1/8:1/2] plot (\x,{((1-2*\x)/3)*(5/12+\x)});

			% Max point
			\fill[red] (-1/12, 4/27) ellipse (0.008 and 0.004) node[above,black] {$\frac{4}{27}$};
		\end{tikzpicture}
		\caption{Profit functions for link operators $1$ and $2$ of Example when opposed against the candidate Nash Equilibrium $t^* = \left(\tfrac{5}{12}, \tfrac{4}{12}\right)$. Red dot indicates the max profit at $t^*$. Black dots indicate tolls when relevant intervals start or end for the respective pseudo-heterogeneous profits (dashed lines) they correspond.}
		\label{figure:alpha_step:profit_no_ne}
	\end{figure}
\end{frame}

\subsection*{Pricing N.E. Algorithm}

\begin{frame}{Step distribution functions | Pricing N.E. Algorithm}
	\textbf{Summary:} Given $\Gm = ([2], \ell, \alpha), \alpha = (\alpha_k)_{k \in [m]}$
	\begin{enumerate}
		\item Consider $\Gm^\prime = ([2], \ell, 1)$ and calculate $t^{*\prime}, x^\prime(t^{*\prime})$.
		\item Calculate $\as(t^*)$ as the $\alpha$ value of the lower split flow in $x^\prime(t^{*\prime})$
		\item Calculate the candidate $t^* = \tfrac{t^{*\prime}}{\as(t^*)}$ and $\Pi_1(t^*), \Pi_2(t^*)$.
		\item For each $\alpha_k \ne \as(t^*)$, and for each $i \in [2]$
		\begin{enumerate}[$4.1$]
			\item Consider $\Gm_k = ([2], \ell, \alpha_k)$ and calculate $B_{i, k}(t_{-i}^*)$ and tolls $t_{i, k}$ for which $\as(t_i, t_{-i}^*) = \alpha_k$, resulting in two toll ranges for $t_i$.
			\item Search for $\Pi_{i, k}$, the max profit for $i$ within those $t_i$ ranges:
			\begin{enumerate}[$4.2.1$]
				\item If $B_{i, k}(t_{-i}^*)$ is within the $t_i$ ranges, $\Pi_{i, k} = \Pi_i(B_{i, k}(t_{-i}^*), t_{-i}^*)$.
				\item Otherwise, $\Pi_{i, k}$ is max profit among the ranges' edge tolls ($4$).
			\end{enumerate}
			\item If $\Pi_i < \Pi_{i, k}$, then $\Gm$ has no pricing Nash Equilibrium.
		\end{enumerate}
		\item If all of the above checks fail, return $t^*$.
	\end{enumerate}
	\textbf{Complexity:} $\mathcal{O}(m)$
\end{frame}

\section{Future work}

\begin{frame}{Conclusion}
	In our work
	\begin{itemize}
		\item Investigated heterogeneity by defining and analysing $\as$
		\item Developed a connection between different fixed distribution functions
		\item Developed an algorithm for $2$-link step distribution functions
	\end{itemize}
\end{frame}

\begin{frame}{Future work}
	\begin{itemize}
		\item Pseudo-heterogeneous games
		\item Step distribution functions
		\begin{itemize}
			\item $2$-link networks
			\item $n$-link networks
		\end{itemize}
		\item Continuous distribution functions
		\item Function series approximation
		\item Split function for series-parallel links
	\end{itemize}
\end{frame}

\begin{frame}[noframenumbering,plain]{}
	\centering
    \huge Thank you!\\
    \normalsize Thomas Pappas\\
    thpappas@di.uoa.gr
\end{frame}

\end{document}
