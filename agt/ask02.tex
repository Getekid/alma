% -*- mode: LaTeX; coding: utf-8 -*-
% Typeset with: XeLaTeX

\documentclass[a4paper,11pt]{article}
\usepackage{a4wide}

% Greek fonts
\RequirePackage{fontspec}
\defaultfontfeatures{Ligatures=TeX}
  % you may want to try: {Liberation Serif} or {Times New Roman}
\setmainfont{FreeSerif}
  % you may want to try: {Liberation Sans} or {Arial}
\setsansfont[Scale=MatchLowercase]{FreeSans}
  % you may want to try: {FreeMono} or {Courier New}
\setmonofont[Scale=MatchLowercase]{FreeMono}

\usepackage{mathtools}
% Required packages for plotting
\usepackage{pgfplots}
%\usepackage{pgfplotstable}
\pgfplotsset{compat=1.16}
%\usepackage{tikz}


\newcommand{\overeq}[1]{\stackrel{\text{#1}}{=}}
\newcommand{\vi}{\mathrm{v}}
\newcommand{\Expect}{{\rm I\kern-.3em E}}

% Main document
\begin{document}
\title{Αλγοριθμική Θεωρία Παιγνίων - 2η σειρά ασκήσεων}
\author{Θωμάς Παππάς}
%\date{}
\maketitle

\section*{Πρόβλημα 1}

\paragraph{(i)}
Αν $b_1,b_2$ οι προσφορές του κάθε παίκτη έχουμε
\[
	u_1(b_1,b_2) =
	\begin{cases}
		K-b_1 & b_1>b_2\\
		K/2-b_1 & b_1=b_2\\
		-b_1 & b_1<b_2
	\end{cases}
	\quad\quad
	u_2(b_1,b_2) =
	\begin{cases}
		K-b_2 & b_2>b_1\\
		K/2-b_2 & b_2=b_1\\
		-b_2 & b_2<b_1
	\end{cases}
\]

\paragraph{(ii)}
Έστω $(b_1,b_2)$ μια στρατηγική του παιγνίου, θα δείξουμε ότι πάντα υπάρχει μια καλύτερη στρατηγική για τουλάχιστον έναν από τους δύο παίκτες.
\begin{itemize}
	\item αν $b_1<b_2$ τότε ο π.2 έχει καλύτερη στρατηγική την προσφορά $b_2^\prime=b_2-(b_2-b_1)/2$ αφού εφόσον $b_1<b_2^\prime<b_2$ άρα
		\[u_2(b_2,b_2)=K-b_2<K-b_2+\frac{b_2-b_1}{2}=K-b_2^\prime=u_2(b_1,b_2^\prime)\]
		Ομοίως και η περίπτωση $b_2<b_1$ για τον π.1 λόγω συμμετρικότητας του προβλήματος
	\item αν $b_1=b_2$ τότε για οποιονδήποτε παίκτη (εδώ θα πάρουμε τον π.1) διακρίνουμε τις εξής περιπτώσεις
		\begin{itemize}
			\item αν $b_1<K/2$ τότε για  $b_1^\prime=b_1+K/2$ έχουμε $b_1^\prime>b_1$ και $u_1(b_1^\prime,b_2) = K-b_1^\prime = K/2-b_1$ και άρα
				\[u_1(b_1,b_2)=K/2-b_1<K/2-b_1^\prime=u_1(b_1^\prime,b)\]
			\item αν $b_1>K/2$ τότε $u_1(b_1,b_2)=K/2-b_1<0$ και άρα για $b_1^\prime=0$ παίρνουμε
				\[u_1(b_1,b_2)<0=u_1(b_1^\prime,b_2)\]
		\end{itemize}
\end{itemize}
Συνεπώς το παίγνιο δεν έχει σημεία ισορροπίας.

\pagebreak

\section*{Πρόβλημα 2}

Ο αλγόριθμος που περιγράφει η άσκηση είναι μονότονος διότι αποτελείται από δύο διαδικασίες που η καθεμιά είναι μονότονη.
\\[8pt]
Στην 1η φάση η ανάθεση λειτουργεί όπως μια Knapsack δημοπρασία και άρα είναι μονότονη.
Το ίδιο ισχύει και για τη 2η φάση αν τη δούμε μεμονωμένα.
Τώρα αν ένας παίκτης που έχει κερδίσει στη 2η φάση αυξήσει την προσφορά του, τότε η μόνη αλλαγή που ενδέχεται να γίνει είναι να ανέβει στην κατάταξη του άπληστου αλγόριθμου και να κερδίσει στην 1η φάση, όπου και πάλι παραμένει κερδισμένος.
\\[8pt]
Άρα και στο σύνολό του ο αλγόριθμος είναι μονότονος.


\section*{Πρόβλημα 3}

Θα θέσουμε μηχανισμό όπου ο η εισφορά του κάθε δημότη θα είναι το κομμάτι της εκτίμησης τους που συνεισφέρει στο συνολικό άθροισμα να φτάσει το στόχο $C$, δηλαδή αν $b=(b_1,b_2,\dots,b_n)$ οι προσφορές από τους δημότες τότε
\[
	p_i =
	\begin{dcases}
		max\{C-\sum_{j\neq i}b_j,0\} & \sum_{i=1}^n b_i \geq C\\
		0 & \sum_{i=1}^n b_i < C
	\end{dcases}
\]
Θα δείξουμε ότι ο μηχανισμός αυτός είναι φιλαλήθης.
Αρχικά βλέπουμε ότι για ένα σύνολο προσφορών $b$ αν το πάρκο χτιστεί τότε $u_i(b)= \vi_i-p_i$ ενώ αν δεν χτιστεί τότε $u_i(b)=0$.
\\[8pt]
Στην περίπτωση που $\sum_{j\neq i}^n b_j \geq C$ τότε $\forall b_i$ το πάρκο θα χτιστεί με $p_i=0$ και $u_i(b) = u_i(\vi_i,b_{-i}) = \vi_i-0 = \vi_i$.
\\[8pt]
Τώρα αν $\sum_{j\neq i}^n b_j < C$ έχουμε
\begin{itemize}
	\item αν $\vi_i+\sum_{j\neq i}^n b_j < C$ τότε το πάρκο δεν θα χτιστεί με $b_i=\vi_i$, ενώ αν $b_i$ αρκετά μεγάλο ώστε το πάρκο να χτιστεί τότε
		\[u_i(b)=\vi_i-p_i=\vi_i-C+\sum_{j\neq i} b_j < 0 = u_i(\vi_i,b_{-i})\]
	\item αν $\vi_i+\sum_{j\neq i}^n b_j \geq C$ τότε το πάρκο θα χτιστεί με $b_i=\vi_i$ με
		\[u_i(\vi_i,b_{-i}) = \vi_i-p_i = \vi_i-C+\sum_{j\neq i} b_j > 0\]
		όπου $0$ είναι το $u_i$ στην περίπτωση που το $b_i$ είναι αρκετά μικρό ώστε το πάρκο να μην χτιστεί
\end{itemize}
Άρα σε κάθε περίπτωση η στρατηγική $(\vi_i,b_{-i})$ μεγιστοποιεί την ωφέλεια του $i$ παίκτη, και άρα ο μηχανισμός είναι φιλαλήθης.

\pagebreak


\section*{Πρόβλημα 4}

\paragraph{(i)}
Θα σχεδιάσουμε μηχανισμό όπου ο κάθε παίκτης θα κερδίζει ωφέλεια ανάλογα με το πόσο συνεισέφερε στο να ελαχιστοποιηθεί η συνολική αξία του ελάχιστου συνδετικού δέντρου σε σχέση με το αν δεν χρησιμοποιούσαμε την ακμή του.
Ορίζουμε λοιπόν ως $T_{-e}$ το ελάχιστο συνδετικό δέντρο που θα πάρουμε εάν δεν χρησιμοποιήσουμε την ακμή $e$.
Αυτό πάντα υπάρχει αφού το $G$ δεν έχει γέφυρες.
\\[8pt]
Επίσης για διευκόλυνση στους ορισμούς θέτουμε ως $u(S)$ την αξία ενός δέντρου, δηλαδή αν $S \subseteq E$ δέντρο τότε
\[u(S) = \sum_{e \in S} b_e\]
\\[8pt]
Οπότε για ένα σύνολο προσφορών $b=\{b_e\}_{e\in E}$ και ελάχιστο συνδετικό δέντρο $T$ ορίζουμε τις αμοιβές ως εξής
\[p_e = u(T_{-e}) - u(T \setminus \{e\})\]
Θα δείξουμε τώρα ότι ο αυτός ο μηχανισμός είναι φιλαλήθης.
Έστω ότι έχουμε $b=(c_e,b)$ σύνολο προσφορών όπου το παίκτης $A_e$ δηλώνει την πραγματική του αξία.
\\[8pt]
Στην περίπτωση που $e \notin T$ τότε έχουμε $p_e = 0$ αφού $T_{-e}=T$ και $T \setminus \{e\} = T$, και άρα $u_e(c_e,b_{-e}) = 0$.
Προφανώς το ίδιο ισχύει και για κάθε $b_e>c_e$.
Έστω τώρα ότι ο $A_e$ επέλεγε να δηλώσει $b_e<c_e$ έτσι ώστε $e\in T$.
Εφόσον αρχικά είχαμε $e \notin T$ αυτό σημαίνει ότι αν πάρουμε $T_e = \min_{u(T)}\{T|e\in T\}$ τότε έχουμε $u(T) \leq u(T_e)$ και με την $b_e$ δήλωση ο $A_e$ κερδίζει ωφέλεια
\begin{align*}
	u_e(b_e,b_{-e}) = p_e - c_e &= u(T_{-e}) - u(T_e \setminus\{e\}) -c_e = u(T) - u(T_e) \leq 0\\
		&\Rightarrow u_e(b_e,b_{-e}) \leq u_e(c_e,b_{-e})
\end{align*}
Μετά στην περίπτωση που $e\in T$ τότε
\[u_e(c_e,b_{-e}) = p_e - c_e = u(T_{-e}) - u(T\setminus \{e\}) - c_e = u(T_{-e}) - u(T)\]
και εφόσον $T$ ελάχιστο συνδετικό δέντρο τότε $u(T) \leq u(T_{-e})$ και άρα $u_e(c_e,b_{-e}) \geq 0$.
Προφανώς και για κάθε $b_e$ τέτοιο ώστε $e \in T$ η αμοιβή δεν αλλάζει αφού είναι ανεξάρτητη του $b_e$, ενώ για $b_e > c_e$ σε σημείο που $e \notin T$ τότε $u_e(b_e,b_{-e}) = 0 \leq u_e(c_e,b_{-e})$.
\\[8pt]
Άρα έχουμε πάντα $u_e(b_e,b_{-e}) \leq u_e(c_e,b_{-e})$ και άρα ο μηχανισμός είναι φιλαλήθης.

%\paragraph{(ii)}
%Η ιδιότητα που θα πρέπει να έχει το γράφημα $G$ ώστε ο παραπάνω αλγόριθμος να είναι φιλαλήθης είναι ότι καμία ακμή $e$ δεν είναι γέφυρα.

\paragraph{(iii)}
Αν οι παίκτες μπορεί να έχουν στην κατοχή τους περισσότερες από μία ακμές τότε γενικεύουμε τους ορισμούς ως εξής
\begin{itemize}
	\item ο παίκτης $A_S$ έχει ένα σύνολο ακμών $S \subseteq E$
	\item $T_S \subseteq S$ είναι το σύνολο ακμών του $A_S$ που συμμετέχουν στο $T$ 
		\begin{itemize}
			\item αν δεν συμμετέχει καμία τότε $T_S=\emptyset$
		\end{itemize}
	\item $T_{-S}$ είναι το ελάχιστο συνεκτικό δέντρο που θα πάρουμε αν δεν χρησιμοποιήσουμε κάποια ακμή από το $S$
\end{itemize}
Τότε η αμοιβή προσαρμόζεται ως εξής
\[p_S = u(T_{-S}) - u(T \setminus S) = u(T_{-S}) - u(T \setminus T_S)\]
και αντίστοιχα, εφόσον ο παίκτης θα χρεωθεί μόνο το κόστος των ακμών που συνεισέφεραν στο $T$, η ωφέλειά του γίνεται
\[u_S = p_S - \sum_{e\in T_S} c_e = u(T_{-S}) - u(T \setminus T_S) -  \sum_{e\in T_S} c_e = u(T_{-S}) - u(T)\]
Η ιδιότητα τώρα που πρέπει να έχει το γράφημα ώστε ο παραπάνω μηχανισμός να είναι φιλαλήθης, είναι ότι για κάθε $S$ σύνολο ακμών ενός παίκτη το $G(V,E \setminus S)$ θα πρέπει να παραμένει συνεκτικό.
Με άλλα λόγια δεν γίνεται ένας παίκτης να ελέγχει σύνολο ακμών που περιέχουν τουλάχιστον μια ακμή από κάθε δέντρο του γραφήματος.


\section*{Πρόβλημα 5}

\paragraph{Για δύο παίκτες} θα ξεκινήσουμε βρίσκοντας το $SW^*=\max SW$ για κάθε περίπτωση.
Για το συνολικό $SW$ εφόσον δεν μπορούμε να πετύχουμε θετική αποτίμηση και για τους δύο παίκτες τότε έχουμε $SW=1$ με τις εξής βέλτιστες αναθέσεις
\[
	S^1=(\{a,b\},\emptyset) \qquad S^2=(\emptyset,\{a,b\}) \qquad S^3=(\emptyset,\{a\})
\]
Μετά στις μεμονωμένες περιπτώσεις προφανώς πετυχαίνουμε $SWM$ δίνοντας όλα τα αντικείμενα στον εκάστοτε παίκτη, οπότε εντέλει έχουμε
\[
	SW^* = 1 \qquad SW^*_{-1} = 1 \qquad SW^*_{-2} = 1
\]
Τώρα υπολογίζουμε τις πληρωμές.
Θεωρούμε ότι έχουμε την ανάθεση $S^1=(\{a,b\},\emptyset)$.
Τότε
\[
	\begin{aligned}
		p_1 &= SW^*_{-1} - v_2(\emptyset)\\
			&= 1-0\\
			&= 1
	\end{aligned}
	\qquad\qquad
	\begin{aligned}
		p_2 &= SW^*_{-2} - v_1(\{a,b\})\\
			&= 1-1\\
			&= 0
	\end{aligned}
\]
Οι πληρωμές βγαίνουν αντίστροφα $p_1=0,p_2=1$ στις αναθέσεις $S^2,S^3$.

\paragraph{Για τρεις παίκτες} βλέπουμε ότι μπορούμε να πετύχουμε $SW^*=2$ με την ανάθεση $S=(\emptyset,\{a\},\{b\})$.
Μετά έχουμε
\begin{itemize}
	\item $SW^*_{-1} = 2$ με την ανάθεση $(\{a\},\{b\})$
	\item $SW^*_{-2} = 1$ αντίστοιχα με το προηγούμενο ερώτημα
	\item $SW^*_{-3} = 1$ από το προηγούμενο ερώτημα
\end{itemize}
και άρα οι πληρωμές γίνονται
\[
	\begin{aligned}
		p_1 &= SW^*_{-1} - v_2(\{a\}) - v_3(\{b\})\\
			&= 2-1-1\\
			&= 0
	\end{aligned}
	\qquad
	\begin{aligned}
		p_2 &= SW^*_{-2} - v_1(\emptyset) - v_3(\{b\})\\
			&= 1-0-1\\
			&= 0
	\end{aligned}
	\qquad
	\begin{aligned}
		p_3 &= SW^*_{-3} - v_1(\emptyset) - v_2(\{a\})\\
			&= 1-0-1\\
			&= 0
	\end{aligned}
\]
Τέλος για τις πληρωμές παρατηρούμε ότι ακόμα και οι παίκτες που συνεισφέρουν στο $SW^*$ καταλήγουν μην πληρώνουν τίποτα.
Αυτό διότι η συνεισφορά τους είναι ίση με την αποτίμηση που κερδίζουν.


\section*{Πρόβλημα 6}

Έτσι ώστε ο μηχανισμός να μεγιστοποιεί την κοινωνική ωφέλεια θα πρέπει να γίνεται βέλτιστη ανάθεση των κομματιών σοκολάτας στους παίκτες, οπότε ο αλγόριθμος ανάθεσης θα λειτουργεί ως εξής
\begin{itemize}
	\item ταξινομούμε τους παίκτες έτσι ώστε $b_{i_1}/w_{i_1} > b_{i_2}/w_{i_2} > \dots > b_{i_n}/w_{i_n}$
	\item λύνουμε το \textrm{Subset Sum} πρόβλημα για κάθε παίκτη με την παραπάνω σειρά για τα αντικείμενα που έχουν μείνει και του αναθέτουμε τη λύση
	\item συνεχίζουμε μέχρι να τελειώσουν τα αντικείμενα ή αν περάσουμε όλους τους παίκτες
\end{itemize}
Εφόσον το \textrm{Subset Sum} είναι NP-complete πρόβλημα τότε ο παραπάνω αλγόριθμος δεν είναι πολυωνυμικά αποδοτικός, εκτός κι αν τον προσεγγίζουμε πολυωνυμικά.
\\[8pt]
Για την πληρωμή τώρα, έστω $S$ η βέλτιστη ανάθεση και $S_{-i}$ η βέλτιστη ανάθεση χωρίς τον παίκτη $i$.
Οπότε έχουμε
\[p_i = u(S) - \sum_{j \neq i} b_j \cdot w_j\]
Ο μηχανισμός είναι φιλαλήθης με $u_i(\vi_i,b_{-i}) \geq 0 $ άρα για ο κάθε μεμονωμένος παίκτης έχει $u_i \geq 0$ και άρα ο μηχανισμός είναι και individually rational.

\paragraph{(ii)} Αν εφαρμόσουμε τη λύση του (i) στην παραλλαγή του προβλήματος εδώ τότε το \textrm{Subset Sum} θα επιστρέφει πάντα την greedy λύση.
Μπορούμε αντί αυτού να κοιτάζουμε απευθείας για κάθε παίκτη αν τα κομμάτια που έχουν μείνει είναι $\geq w_i$ και αν ναι να του αναθέτουμε $w_i$ κομμάτια.
Με αυτόν τον τρόπο κρατάμε τις ιδιότητες του προηγούμενου (φιλαλήθης, μεγιστοποίηση κοινωνικής ωφέλειας, individually rational) ενώ έχουμε αποδοτικό αλγόριθμο που να τον υπολογίζει.


\section*{Πρόβλημα 7}

\paragraph{(i)}
Θέτω τις εξής τυχαίες μεταβλητές
\begin{itemize}
	\item $X_1,X_2$: οι προσφορές του π.1 και π.2 αντίστοιχα, $X_1,X_2 \sim U(0,1)$
	\item $X_m = \min\{X_1,X_2\}$
	\item $R$ το κέρδος της δημοπρασίας
\end{itemize}
Προφανώς εδώ $R=X_m$, οπότε θα υπολογίσουμε την αναμενόμενη τιμή της $X_m$
\begin{align*}
	F_{X_m}(x) &= P(X_m \leq x) = 1-P(X_m \geq x)\\
		&= 1-P(X_1 \geq x) \cdot P(X_2 \geq x) &\\
		&=1-(1-P(X_1<x))(1-P(X_2<x)) = 1-(1-x)(1-x) = 2x-x^2\\
	f_{X_m}(x) &= \frac{d}{dx}F_{X_m}(x) = 2-2x\\
	\Expect[X_m] &= \int_0^1 x \cdot f_{X_m}(x) dx = \int_0^1 2x-2x^2 dx = \left[x^2-\frac23 x^3\right]_0^1 = 1-\frac23 = \frac13
\end{align*}
Άρα όντως το αναμενόμενο κέρδος από τη δημοπρασία Vickrey είναι $1/3$.

\paragraph{(ii)}
Θεωρούμε την τυχαία μεταβλητή $X_M = max\{X_1,X_2\}$.
Με reserve price το $1/2$ έχουμε ότι το αναμενόμενο κέρδος είναι η αναμενόμενη τιμή του $X_m$ όταν $X_m \geq 1/2$ και $1/2$ όταν $X_m<1/2$ και $X_M \geq 1/2$. Δηλαδή
\[
	\Expect[R] = \Expect \left[X_m|X_m \geq 1/2\right] \cdot P(X_m\geq 1/2)) + 1/2 \cdot P \left(X_M>1/2,X_m<1/2\right)
\]
Για το $\Expect[X_m|X_m \geq 1/2]$ μπορούμε να δούμε ότι με δεδομένο ότι $X_m\geq 1/2$ τότε το πείραμα επαναλαμβάνεται στο διάστημα $[1/2,1]$.
Οπότε αν ξεκινήσουμε από το $1/2$ τότε η αναμενόμενη τιμή θα είναι αναλογικά το $1/3$ του διαστήματος $[1/2,1]$, δηλαδή
\[
	\Expect \left[X_m|X_m \geq 1/2\right] = \frac12 + \frac13 \left(1-\frac12\right) = \frac12 + \frac16 = \frac23
\]
Για το δεύτερο μέρος τώρα έχουμε
\begin{align*}
	P \left(X_M>1/2,X_m<1/2\right) &= P(X_1>1/2,X_2<1/2) + P(X_2>1/2,X_1<1/2) &\\
		&\stackrel{X_1,X_2 \textrm{ ανεξ.}}{=\joinrel=\joinrel=\joinrel=\joinrel=} P(X_1>1/2) \cdot P(X_2<1/2) + P(X_2>1/2) \cdot P(X_1<1/2)\\
		&=\frac12 \cdot \frac12 + \frac12 \cdot \frac12 = \frac12
\end{align*}
Οπότε εφόσον επίσης $P(X_m\geq 1/2) = 1-P(X_m<1/2) = 1-2\cdot\frac12+(\frac12)^2 = 1/4$ εντέλει παίρνουμε
\[
	\Expect[R] = \frac23 \cdot \frac14 + \frac12 \cdot \frac12 = \frac5{12}
\]
Άρα το επιπλέον κέρδος που πήραμε χρησιμοποιώντας reserve price το $1/2$ είναι $\frac5{12}-\frac13=\frac1{12}$.

\paragraph{(iii)}
Θέτω $Y_2$ τ.μ. για τη μεσαία τιμή ανάμεσα στις $X_1,X_2,X_3$.
Για την $F_{Y_2}(y)$ τώρα αρκεί να παρατηρήσουμε ότι η μεσαία τιμή είναι $<y$ όταν είτε και οι τρεις προσφορές είναι $<y$ είτε είναι μόνο οι δύο από αυτές.
Εφόσον $X_1,X_2,X_3$ ανεξάρτητες τότε έχουμε
\begin{align*}
	&F_{Y_2}(y) &= P(Y_2 \leq y) = &P(X_1 \leq y,X_2 \leq y,X_3 \leq y) + P(X_1 \leq y,X_2 \geq y,X_3 > y) \\
		&&&+ P(X_1 \leq y,X_2 > y,X_3 \leq y) + P(X_1 > y,X_2 \geq y,X_3 \leq y)\\
		&&\stackrel{\textrm{ ανεξ.}}{=\joinrel=\joinrel=}& P(X_1 \leq y)P(X_2 \leq y)P(X_3 \leq y) + P(X_1 \leq y)P(X_2 \geq y)P(X_3 > y) \\
		&&&+ P(X_1 \leq y)P(X_2 > y)P(X_3 \leq y) + P(X_1 > y)P(X_2 \geq y)P(X_3 \leq y)\\
		&&=&y^3+3y^2(1-y) = -2y^3+3y^2
\end{align*}
\begin{align*}
	f_{Y_2}(y) &= \frac{d}{dy}F_{X_2}(y) = -6y^2+6y\\
	\Expect[Y_2] &= \int_0^1 y f_{Y_2}(y) dy = \int_0^1 y (-6y^2+6y) dy = \int_0^1 -6y^2+6y^2\\
		&= \left[-\frac64 y^4 + 2y^3\right]_0^1 = -\frac32 + 2 = \frac12
\end{align*}


\section*{Πρόβλημα 8}
Αρχικά έχουμε ότι
\[f(z) = \frac{d}{dz}F(z) = \frac1{(z+1)^{2c}}\]
οπότε και το virtual valuation function $\phi(z)$ θα είναι
\[\phi(z) = z - \frac{1-F(z)}{f(z))} = z - \frac{1/(z+1)^c}{1/(z+1)^{2c}} = z - (z+1)^c\]
Τώρα για να είναι η κατανομή αυτή regular θα πρέπει να είναι μη-φθίνουσα, δηλαδή
\[\frac{d}{dz}\phi(z) \geq 0, \forall z \in [0,\infty)\]
Έχουμε $\frac{d}{dz}\phi(z) = 1-c(z+1)^{c-1}$.
Για $z=0$ παίρνουμε
\[\frac{d}{dz}\phi(0) \geq 0 \Rightarrow 1 - c \cdot 1^{c-1} \geq 0 \Rightarrow 1-c \leq 0 \Rightarrow c \leq 1\]
Όντως αν $c\leq 1$ βλέπουμε ότι
\begin{align*}
	c \leq 1 &\Rightarrow c-1 \leq 0 \stackrel{z+1\geq 1}{=\joinrel=\joinrel=\joinrel\Rightarrow} (z+1)^{c-1} \leq (z+1)^0 \Rightarrow (z+1)^{c-1} \leq 1\\
		&\stackrel{c \geq 0}{=\joinrel\Rightarrow} c(z+1)^{c-1} \leq c \stackrel{c \leq 1}{=\joinrel=\joinrel\Rightarrow} c(z+1)^{c-1} \leq 1 \Rightarrow 1-c(z+1)^{c-1} \geq 0\\
		&\Rightarrow \frac{d}{dz}\phi(z) \geq 0
\end{align*}
Άρα όντως η κατανομή είναι φθίνουσα για $c \leq 1$.

\end{document}
