% -*- mode: LaTeX; coding: utf-8 -*-
% Typeset with: XeLaTeX

\documentclass[a4paper,11pt]{article}
\usepackage{a4wide}

% Greek fonts
\RequirePackage{fontspec}
\defaultfontfeatures{Ligatures=TeX}
  % you may want to try: {Liberation Serif} or {Times New Roman}
\setmainfont{FreeSerif}
  % you may want to try: {Liberation Sans} or {Arial}
\setsansfont[Scale=MatchLowercase]{FreeSans}
  % you may want to try: {FreeMono} or {Courier New}
\setmonofont[Scale=MatchLowercase]{FreeMono}

\usepackage{amsmath}


% Main document
\begin{document}
\title{Αλγοριθμική Επιστήμη Δεδομένων - 1η σειρά ασκήσεων}
\author{Θωμάς Παππάς}
%\date{}
\maketitle

\section*{Άσκηση 1}

\subsection*{Άσκηση 6.3.1}

\paragraph{(a)} Το support για τα μεμονωμένα αντικείμενα, κάνοντας ένα πέρασμα από τα δεδομένα, είναι
\begin{center}
	% Result printed: [4, 6, 8, 8, 6, 4]
	\begin{tabular}{| c | c | c | c | c | c |}
		\hline
		$1$ & $2$ & $3$ & $4$ & $5$ & $6$ \\
		\hline
		$4$ & $6$ & $8$ & $8$ & $6$ & $4$ \\
		\hline
	\end{tabular}
\end{center}

ενώ για τα ζευγάρια αντικειμένων, χρησιμοποιώντας τη μέθοδο Triangular-Matrix βρίσκουμε ότι είναι
\begin{center}
	% Result printed: [2, 3, 2, 1, 0, 3, 4, 2, 1, 4, 4, 2, 3, 3, 2]
	\begin{tabular}{| c | c || c | c || c | c |}
		\hline
		$\{1,2\}$ & $2$ & $\{2,3\}$ & $3$ &$\{3,5\}$ & $4$ \\
		\hline
		$\{1,3\}$ & $3$ & $\{2,4\}$ & $4$ &$\{3,6\}$ & $2$ \\
		\hline
		$\{1,4\}$ & $2$ & $\{2,5\}$ & $2$ &$\{4,5\}$ & $3$ \\
		\hline
		$\{1,5\}$ & $1$ & $\{2,6\}$ & $1$ &$\{4,6\}$ & $3$ \\
		\hline
		$\{1,6\}$ & $0$ & $\{3,4\}$ & $4$ &$\{5,6\}$ & $2$ \\
		\hline
	\end{tabular}
\end{center}

\paragraph{(b)} Χρησιμοποιώντας τον κανόνα $\{i,j\} \rightarrow i \times j \bmod 11$ βρίσκουμε ότι τα ζευγάρια αντικειμένων αντιστοιχίζονται ως εξής:
\begin{center}
	\begin{tabular}{| c | c || c | c || c | c |}
		\hline
		$\{1,2\}$ & $2$ & $\{2,3\}$ & $6$ &$\{3,5\}$ & $4$ \\
		\hline
		$\{1,3\}$ & $3$ & $\{2,4\}$ & $8$ &$\{3,6\}$ & $7$ \\
		\hline
		$\{1,4\}$ & $4$ & $\{2,5\}$ & $10$ &$\{4,5\}$ & $9$ \\
		\hline
		$\{1,5\}$ & $5$ & $\{2,6\}$ & $1$ &$\{4,6\}$ & $2$ \\
		\hline
		$\{1,6\}$ & $6$ & $\{3,4\}$ & $1$ &$\{5,6\}$ & $8$ \\
		\hline
	\end{tabular}
\end{center}

% Hash support on first pass: [0, 5, 5, 3, 6, 1, 3, 2, 6, 3, 2]
\paragraph{(c)} Από τα buckets που φτιάξαμε από το hash table στο first pass, αυτά που είναι συχνά είναι (ξεκινώντας το μέτρημα από το $0$) τα $1,2,4,8$.

\paragraph{(d)} Τα ζευγάρια που θα μετρηθούν στο 2ο πέρασμα του αλγόριθμου PCY είναι αυτά στα οποία και τα δύο στοιχεία είναι συχνά, αλλά και το ζευγάρι αντιστοιχεί σε hash bucket που είναι συχνός.
Εφόσον όλα τα μονοσύνολα είναι συχνά, από τα hash buckets παίρνουμε ότι θα μετρηθούν τα:
\begin{itemize}
	\item $\{2,6\},\{3,4\}$ (hash bucket $1$)
	\item $\{1,2\},\{4,6\}$ (hash bucket $2$)
	\item $\{1,4\},\{3,5\}$ (hash bucket $4$)
	\item $\{2,4\},\{5,6\}$ (hash bucket $8$)
\end{itemize}


\subsection*{Άσκηση 6.3.2}

Τρέχουμε τον Multistage αλγόριθμο με τα ίδια στοιχεία και το νέο hash κανόνα.
% Hash support on second pass: [0, 3, 2, 2, 0, 2, 4, 4, 5]
Στο νέο hash table μόνο τα buckets $6,7,8$ γίνονται συχνά, άρα τα υποψήφια ζευγάρια όντως μειώνονται στα
\[\{2,4\},\{2,6\},\{3,4\},\{3,5\}\]


\subsection*{Άσκηση 6.4.2}

\paragraph{(a)} Τρέχοντας τον αλγόριθμο Toivonen στα προηγούμενα δεδομένα στο δείγμα\\ $\{1,2,3\},\{2,3,4\},\{3,4,5\},\{4,5,6\}$ με support $1$ βρίσκουμε τα παρακάνω συχνά σύνολα:
\begin{itemize}
	\item $\{1\},\{2\},\{3\},\{4\},\{5\},\{6\}$
	\item $\{1,2\},\{1,3\},\{2,3\},\{2,4\},\{3,4\},\{3,5\},\{4,5\},\{4,6\},\{5,6\}$
	\item $\{1,2,3\},\{2,3,4\},\{3,4,5\},\{4,5,6\}$
\end{itemize}

\paragraph{(b)} Το negative border του δείγματος είναι όλα τα ζευγάρια που δεν είναι συχνά (αφού όλα τα μονοσύνολα είναι συχνά), δηλαδή
\[\{1,4\},\{1,5\},\{1,6\},\{2,5\},\{2,6\},\{3,6\}\]
και κανένα από τα σύνολα με $3$ στοιχεία αφού αυτά που έχουν όλα τους τα υποσύνολα συχνά βρίσκονται ήδη στο δείγμα.

\paragraph{(c)} Εφόσον το negative border αποτελείται μόνο από ζευγάρια στοιχείων, βλέποντας τον πίνακα συχνών ζευγαριών από την άσκηση $6.3.1$-a βρίσκουμε ότι \textbf{κανένα} από τα στοιχεία του negative border δεν είναι συχνό σε ολόκληρο το σύνολο δεδομένων.


%\section*{Άσκηση 2}


\section*{Άσκηση 3}

\paragraph{(α)} Έστω τα στοιχεία $k_1,k_2 \in U$ με $k_1 \leq m$ και $k_2 = k_1+m \leq m^k$.
Προφανώς $k_1 \neq k_2$ και $k_1 \equiv k_2 \bmod m$.
Θα δείξουμε ότι $\forall a \in [m] \setminus 0, b \in [m]$ θα έχουμε $h_{a,b}(k_1) = h_{a,b}(k_2)$.
Όντως
\[k_1 \equiv k_2 \bmod m \Rightarrow ak_1+b \equiv ak_2+b \bmod m \Leftrightarrow h_{a,b}(k_1) = h_{a,b}(k_2)\]
και εφόσον αυτό ισχύει $\forall a$ άρα η πιθανότητα να έχουμε $h_{a,b}(k_1) = h_{a,b}(k_2)$ είναι $1 \geq \frac1{m}$, και άρα αυτή η οικογένεια συναρτήσεων κατακερματισμού δεν είναι καθολική.

\paragraph{(β)} Έστω $k_1,k_2 \in U$ με $k_1 \neq k_2$.
Θα δείξουμε ότι $h_{a,b}(k_1) \neq h_{a,b}(k_2), \forall h \in H$, ή αλλιώς ότι $\Pr_{h_{a,b} \in H}[h_{a,b}(k_1) = h_{a,b}(k_2)] = 0$.
\\[8pt]
Εφόσον $p$ πρώτος, τότε $\forall a \in [p] \setminus 0, \exists a^{-1} : a \cdot a^{-1} \equiv 1 \bmod p$.
Οπότε έστω ότι υπάρχουν $a,b$ τέτοια ώστε $h_{a,b}(k_1) = h_{a,b}(k_2)$.
Τότε έχουμε
\begin{flalign*}
	h_{a,b}(k_1) = h_{a,b}(k_2) &\Rightarrow ak_1+b \equiv ak_2+b \bmod p \Rightarrow ak_1 \equiv ak_2 \bmod p &\\
		&\Rightarrow a^{-1} \cdot ak_1 \equiv a^{-1} \cdot ak_2 \bmod p \Rightarrow k_1 \equiv k_2 \bmod p
\end{flalign*}
με $0 \leq k_1,k_2 \leq m^k < p$, και άρα $k_1 = k_2$, άτοπο.
\\[8pt]
Άρα έχουμε $h_{a,b}(k_1) \neq h_{a,b}(k_2), \forall h \in H$, ή αλλιώς
\[\Pr_{h_{a,b} \in H}[h_{a,b}(k_1) = h_{a,b}(k_2)] = 0 < \frac1{m}\]
και άρα η οικογένεια συναρτήσεων $H$ είναι καθολική.

\paragraph{(γ)} Θέτοντας $U=[p]$ παρατηρούμε ότι η παραπάνω απόδειξη παραμένει απαράλλαχτη, εφόσον για μια τυχαία επιλογή  $k_1,k_2 \in U$ έχουμε πάλι $h_{a,b}(k_1) = h_{a,b}(k_2) \Rightarrow k_1 \equiv k_2 \bmod p$, και αφού $0 \leq k_1,k_2 < p$ παίρνουμε πάλι $k_1 \equiv k_2 \bmod p \Rightarrow k_1 = k_2$.
\\[8pt]
Άρα με τον ίδιο συλλογισμό καταλήγουμε ότι η ιδιότητα της καθολικότητας διατηρείται.


%\section*{Άσκηση 4}

%\paragraph{(α)} Ο χρόνος εισαγωγής του $i$-οστού στοιχείου είναι ένα τυχαίο γεγονός που ακολουθεί την κατανομή Bernoulli($m-i/m$).
%Άρα ο αναμενόμενος αριθμός επαναλήψεων μέχρι την πρώτη επιτυχία είναι $m/m-i$ (εφόσον αυτό ακολουθεί Geom($m-i/m$)).
%\paragraph{(α)} Μετά από την εισαγωγή $n$ στοιχείων, η επιτυχής αναζήτηση .
%Άρα ο αναμενόμενος αριθμός επαναλήψεων μέχρι την πρώτη επιτυχία είναι $1/(1-a)$ (εφόσον αυτό ακολουθεί Geom($1-a$)).
%\\[8pt]
%Από την άλλη, με την ίδια λογική, ο χρόνος εισαγωγής του $i$-οστού στοιχείου είναι $m/m-i$, εφόσον εδώ η πιθανότητα επιτυχίας να βρεθεί κενή θέση είναι $m-i/m$.

\end{document}
