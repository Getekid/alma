% -*- mode: LaTeX; coding: utf-8 -*-
% Typeset with: XeLaTeX

\documentclass[a4paper,11pt]{article}
\usepackage{a4wide}

% Greek fonts
\RequirePackage{fontspec}
\defaultfontfeatures{Ligatures=TeX}
  % you may want to try: {Liberation Serif} or {Times New Roman}
\setmainfont{FreeSerif}
  % you may want to try: {Liberation Sans} or {Arial}
\setsansfont[Scale=MatchLowercase]{FreeSans}
  % you may want to try: {FreeMono} or {Courier New}
\setmonofont[Scale=MatchLowercase]{FreeMono}

\usepackage{amsmath}
\usepackage{hhline}


% Main document
\begin{document}
\title{Αλγοριθμική Επιστήμη Δεδομένων - 1η Σειρά Ασκήσεων}
\author{Θωμάς Παππάς}
%\date{}
\maketitle

\section*{Άσκηση 1}

\subsection*{Άσκηση 6.3.1 (MMDS)}

\paragraph{(a)} Το support για τα μεμονωμένα αντικείμενα, κάνοντας ένα πέρασμα από τα δεδομένα, είναι
\begin{center}
	% Result printed: [4, 6, 8, 8, 6, 4]
	\begin{tabular}{| r || c | c | c | c | c | c |}
		\hline
		item & $1$ & $2$ & $3$ & $4$ & $5$ & $6$ \\ \hline
		support & $4$ & $6$ & $8$ & $8$ & $6$ & $4$ \\ \hline
	\end{tabular}
\end{center}

ενώ για τα ζευγάρια αντικειμένων, χρησιμοποιώντας τη μέθοδο Multistage (με indexing structure Triangular-Matrix) βρίσκουμε ότι είναι
\begin{center}
	% Result printed: [2, 3, 2, 1, 0, 3, 4, 2, 1, 4, 4, 2, 3, 3, 2]
	\begin{tabular}{| c | c || c | c || c | c |}
		\hline
		pair & support & pair & support & pair & support \\ \hhline{|=|=#=|=#=|=|}
		$\{1,2\}$ & $2$ & $\{2,3\}$ & $3$ & $\{3,5\}$ & $4$ \\ \hline
		$\{1,3\}$ & $3$ & $\{2,4\}$ & $4$ & $\{3,6\}$ & $2$ \\ \hline
		$\{1,4\}$ & $2$ & $\{2,5\}$ & $2$ & $\{4,5\}$ & $3$ \\ \hline
		$\{1,5\}$ & $1$ & $\{2,6\}$ & $1$ & $\{4,6\}$ & $3$ \\ \hline
		$\{1,6\}$ & $0$ & $\{3,4\}$ & $4$ & $\{5,6\}$ & $2$ \\ \hline
	\end{tabular}
\end{center}

\paragraph{(b)} Χρησιμοποιώντας τον κανόνα $\{i,j\} \rightarrow i \times j \bmod 11$ βρίσκουμε ότι τα ζευγάρια αντικειμένων αντιστοιχίζονται στα hash buckets ως εξής:
\begin{center}
	\begin{tabular}{| c | c || c | c || c | c |}
		\hline
		pair & bucket & pair & bucket & pair & bucket \\ \hhline{|=|=#=|=#=|=|}
		$\{1,2\}$ & $2$ & $\{2,3\}$ & $6$ & $\{3,5\}$ & $4$ \\ \hline
		$\{1,3\}$ & $3$ & $\{2,4\}$ & $8$ & $\{3,6\}$ & $7$ \\ \hline
		$\{1,4\}$ & $4$ & $\{2,5\}$ & $10$ & $\{4,5\}$ & $9$ \\ \hline
		$\{1,5\}$ & $5$ & $\{2,6\}$ & $1$ & $\{4,6\}$ & $2$ \\ \hline
		$\{1,6\}$ & $6$ & $\{3,4\}$ & $1$ & $\{5,6\}$ & $8$ \\ \hline
	\end{tabular}
\end{center}

% Hash support on first pass: [0, 5, 5, 3, 6, 1, 3, 2, 6, 3, 2]
\paragraph{(c)} Από τα buckets που φτιάξαμε από το hash table στο 1ο πέρασμα, εφόσον αυξάνουμε κατά 1 το counter του κάθε bucket όταν συναντάμε ένα σύνολο που κάνει hash σε αυτό, παίρνουμε
\begin{center}
	\begin{tabular}{| r | c || r | c || r | c || r | c |}
		\hline
		bucket & count & bucket & count & bucket & count & bucket & count \\
		\hhline{|=|=#=|=#=|=#=|=|}
		$0$ & 0 & $3$ & 3 & $6$ & 3 & $9$ & 3 \\ \hline
		$1$ & 5 & $4$ & 6 & $7$ & 2 & $10$ & 2 \\ \hline
		$2$ & 5 & $5$ & 1 & $8$ & 6 &  & \\ \hline
	\end{tabular}
\end{center}
οπότε τα συχνά buckets είναι τα $1,2,4,8$.

\paragraph{(d)} Τα ζευγάρια που θα μετρηθούν στο 2ο πέρασμα του αλγόριθμου PCY είναι αυτά στα οποία και τα δύο στοιχεία είναι συχνά, αλλά και το ζευγάρι αντιστοιχεί σε hash bucket που είναι συχνός.
Εφόσον όλα τα μονοσύνολα είναι συχνά, από τα hash buckets παίρνουμε ότι θα μετρηθούν τα:
\begin{itemize}
	\item $\{2,6\},\{3,4\}$ (hash bucket $1$)
	\item $\{1,2\},\{4,6\}$ (hash bucket $2$)
	\item $\{1,4\},\{3,5\}$ (hash bucket $4$)
	\item $\{2,4\},\{5,6\}$ (hash bucket $8$)
\end{itemize}


\subsection*{Άσκηση 6.3.2 (MMDS)}
% Hash support on second pass: [0, 3, 2, 2, 0, 2, 4, 4, 5]
Τρέχουμε τον Multistage αλγόριθμο με τα ίδια στοιχεία όπως στην Άσκηση 6.3.1 και το νέο hash κανόνα και παίρνουμε
\begin{center}
	\begin{tabular}{| r | c || r | c || r | c |}
		\hline
		bucket & count & bucket & count & bucket & count \\ \hhline{|=|=#=|=#=|=|}
		$0$ & 0 & $3$ & 2 & $6$ & 4 \\ \hline
		$1$ & 3 & $4$ & 0 & $7$ & 4 \\ \hline
		$2$ & 2 & $5$ & 2 & $8$ & 5 \\ \hline
	\end{tabular}
\end{center}
οπότε μόνο τα buckets $6,7,8$ γίνονται συχνά, και άρα τα υποψήφια ζευγάρια όντως μειώνονται στα
\[\{2,4\},\{2,6\},\{3,4\},\{3,5\}\]


\subsection*{Άσκηση 6.4.1}

Με οχτώ items $A,B,\dots,H$ και συχνά itemsets $\{A,B\},\{B,C\},\{A,C\},\{A,D\},\{E\},\{F\}$, το Negative Border αποτελείται από
\begin{itemize}
	\item τα μη συχνά 1-itemsets $\{G\},\{H\}$
	\item όλα τα μη συχνά 2-itemsets που είναι συνδυασμός από συχνά 1-itemsets, εδώ τα $A,B,\dots,F$
		\[
			\{A,E\},\{A,F\},\{B,D\},\{B,E\},\{B,F\},\{C,D\},\{C,E\},\{C,F\},\{D,E\},\{D,F\},\{E,F\}
		\]
	\item το $\{A,B,C\}$, το μοναδικό 3-itemset που έχει όλα τα υποσύνολά του συχνά
\end{itemize}

\end{document}
