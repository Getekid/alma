% -*- mode: LaTeX; coding: utf-8 -*-
% Typeset with: XeLaTeX

\documentclass[a4paper,11pt]{article}
\usepackage{a4wide}

% Greek fonts
\RequirePackage{fontspec}
\defaultfontfeatures{Ligatures=TeX}
  % you may want to try: {Liberation Serif} or {Times New Roman}
\setmainfont{FreeSerif}
  % you may want to try: {Liberation Sans} or {Arial}
\setsansfont[Scale=MatchLowercase]{FreeSans}
  % you may want to try: {FreeMono} or {Courier New}
\setmonofont[Scale=MatchLowercase]{FreeMono}

\usepackage{amsmath}
\usepackage{hhline}

\newcommand{\RightarrowArg}[1]{\stackrel{#1}{\Longrightarrow}}
\newcommand{\indeq}[1]{\stackrel{\text{#1}}{=}}


% Main document
\begin{document}
\title{Αλγοριθμική Επιστήμη Δεδομένων - 2η σειρά ασκήσεων}
\author{Θωμάς Παππάς}
\date{4 Ιουλίου 2022}
\maketitle

\section*{Άσκηση 1}

\paragraph{(α)} Έστω τα στοιχεία $k_1,k_2 \in U$ με $k_1 \leq m$ και $k_2 = k_1+m \leq m^k$.
Προφανώς $k_1 \neq k_2$ και $k_1 \equiv k_2 \bmod m$.
Θα δείξουμε ότι $\forall a \in [m] \setminus 0, b \in [m]$ παίρνουμε $h_{a,b}(k_1) = h_{a,b}(k_2)$.
Όντως
\[k_1 \equiv k_2 \bmod m \Rightarrow ak_1+b \equiv ak_2+b \bmod m \Leftrightarrow h_{a,b}(k_1) = h_{a,b}(k_2)\]
και εφόσον αυτό ισχύει $\forall a, b$, η πιθανότητα να έχουμε $h_{a,b}(k_1) = h_{a,b}(k_2)$ είναι $1 \geq \frac1{m}$, και άρα αυτή η οικογένεια συναρτήσεων κατακερματισμού δεν είναι καθολική.

\paragraph{(β)} Έστω $k_1,k_2 \in U$ με $k_1 \neq k_2$. Παρατηρούμε τα εξής:

\begin{itemize}
	\item Το πλήθος των πιθανών $h_{a,b}$ συναρτήσεων είναι $|H|=p(p-1)$
	\item Εφόσον $p$ πρώτος, $\forall a,b \in [p], a \neq 0$ υπάρχει μοναδικό ζευγάρι $(y_1,y_2) \in [p] \times [p]$ όπου\\
		$y_1 \equiv ak_1+b \bmod p, y_2 \equiv ak_2+b \bmod p$ και $y_1 \neq y_2$, αφού
		\[y_1=y_2 \Rightarrow ak_1+b \equiv ak_2+b \bmod p \RightarrowArg{p \text{ πρώτος}} k_1 \equiv k_2 \bmod p \RightarrowArg{k_1,k_2<p} k_1=k_2 \text{ άτοπο}\]
		Δηλαδή τα $p(p-1)$ το πλήθος ζευγάρια $(y_1,y_2)$ αντιστοιχούνται 1-1 με τις $p(p-1)$ πιθανές επιλογές των $a,b$
	\item Έχουμε $h_{a,b}(k_1) = h_{a,b}(k_2) \Leftrightarrow y_1 \equiv y_2 \bmod m$
\end{itemize}
Μετράμε λοιπόν τα πιθανά ζευγάρια $y_1,y_2$ τέτοια ώστε $y_1 \equiv y_2 \bmod m$.
Έστω ότι $y_1 \equiv 0 \bmod m$ όπου υπάρχουν $\lfloor p/m \rfloor + 1$ πιθανές τιμές για το $y_1$. Για καθεμία έχουμε $\lfloor p/m \rfloor$ το πλήθος $y_2$ τέτοια ώστε $y_2 \equiv 0 \bmod p$ (εξαιρούμε την περίπτωση $y_2=y_1$).
Επαναλαμβάνουμε το παραπάνω για $y_1 \equiv i \bmod m, \forall i = 1,2,\dots,[(p-1) \bmod m]$ διότι μέχρι εκεί θα ισχύουν τα ίδια πλήθη. Για $i=(p \bmod m),\dots,m-1$ έχουμε $\lfloor p/m \rfloor$ πιθανές τιμές για το $y_1$ και $\lfloor p/m \rfloor - 1$ για το $y_2$.

Θέτουμε $p_m = p \bmod m$ και αναφέρουμε ότι
\begin{equation}
	m \left\lfloor \frac{p}{m} \right\rfloor = p - p_m \label{eq1}
\end{equation}

Συνολικά λοιπόν έχουμε 
\begin{flalign*}
	|h_{a,b}|_{h_{a,b}(k_1)=h_{a,b}(k_2)} &= p_m \left(\left\lfloor \frac{p}{m} \right\rfloor + 1\right) \left\lfloor \frac{p}{m} \right\rfloor + (m-p_m) \left\lfloor \frac{p}{m} \right\rfloor \left(\left\lfloor \frac{p}{m} \right\rfloor - 1\right) \\
		&= \left\lfloor \frac{p}{m} \right\rfloor \left( p_m \left\lfloor \frac{p}{m} \right\rfloor + p_m + (m-p_m) \left\lfloor \frac{p}{m} \right\rfloor - m + p_m \right) \\
		&= \left\lfloor \frac{p}{m} \right\rfloor \left( m \left\lfloor \frac{p}{m} \right\rfloor - m + 2p_m \right) \\
		&\indeq{(\ref{eq1})} \left\lfloor \frac{p}{m} \right\rfloor (p - p_m - m + 2p_m) \\
		&= \left\lfloor \frac{p}{m} \right\rfloor (p - (m - p_m)) \\
		&\leq \left(\frac{p}{m}\right) (p - 1) = \frac{|H|}{m}
\end{flalign*}
Άρα η $H$ είναι καθολική οικογένεια συναρτήσεων κατακερματισμού.

\paragraph{(γ)} Θέτοντας $U=[p]$ παρατηρούμε ότι η παραπάνω απόδειξη παραμένει απαράλλαχτη, εφόσον για μια τυχαία επιλογή  $k_1,k_2 \in U = [p]$ έχουμε πάλι ότι $\forall a,b \in [p], a \neq 0$ υπάρχει 1-1 αντιστοιχία σε ένα ζευγάρι $(y_1, y_2) \in [p] \times [p]$, οπότε με τον ίδιο συλλογισμό καταλήγουμε ότι η οικογένεια συναρτήσεων κατακερματισμού παραμένει καθολική.


%\section*{Άσκηση 2}

%\paragraph{(α)} Ο χρόνος εισαγωγής του $i$-οστού στοιχείου είναι ένα τυχαίο γεγονός που ακολουθεί την κατανομή Bernoulli($m-i/m$).
%Άρα ο αναμενόμενος αριθμός επαναλήψεων μέχρι την πρώτη επιτυχία είναι $m/m-i$ (εφόσον αυτό ακολουθεί Geom($m-i/m$)).
%\paragraph{(α)} Μετά από την εισαγωγή $n$ στοιχείων, η επιτυχής αναζήτηση .
%Άρα ο αναμενόμενος αριθμός επαναλήψεων μέχρι την πρώτη επιτυχία είναι $1/(1-a)$ (εφόσον αυτό ακολουθεί Geom($1-a$)).
%\\[8pt]
%Από την άλλη, με την ίδια λογική, ο χρόνος εισαγωγής του $i$-οστού στοιχείου είναι $m/m-i$, εφόσον εδώ η πιθανότητα επιτυχίας να βρεθεί κενή θέση είναι $m-i/m$.


\section*{Άσκηση 3}

\paragraph{(β)} Στην περίπτωση που ο γράφος είναι δέντρο τότε κάθε ακμή είναι γέφυρα και άρα τα συντομότερα μονοπάτια που περνάνε από την ακμή αυτή είναι αυτά που ξεκινάνε από έναν κόμβο της μιας πλευράς και καταλήγουν σε κόμβο της άλλης.
Οπότε αν η ακμή χωρίζει το γράφο στους υπογράφους $G_1,G_2$ τότε το edge betweeness της ακμής είναι $|G_1| \times |G_2|$.

Άρα ο αλγόριθμος μπορεί να απλοποιηθεί στο να κάνουμε ένα BFS στο δέντρο όπου σε κάθε κόμβο $v$ που κάνουμε visit παίρνουμε το υποδέντρο με ρίζα $v$, έστω $T$, και τότε το edge betweeness της ακμής που καταλήγει στον $v$ είναι $|T| \times |G \setminus T|$.

Αυτός ο αλγόριθμος έχει πολυπλοκότητα όσο και το BFS, δηλαδή $\mathcal{O}(|V|+|E|)$.


\section*{Άσκηση 4}

\subsection*{10.4.1 (MMDS)}

Στους παρακάτω πίνακες η σειρά των κόμβων είναι η αλφαβητική.

\paragraph{(a)} Πίνακας γειτνίασης:
\[
	\begin{bmatrix}
		0 & 1 & 1 & 0 & 0 & 0 & 0 & 0 & 0 \\
		1 & 0 & 1 & 0 & 0 & 0 & 0 & 1 & 0 \\
		1 & 1 & 0 & 1 & 0 & 0 & 0 & 0 & 0 \\
		0 & 0 & 1 & 0 & 1 & 1 & 0 & 0 & 0 \\
		0 & 0 & 0 & 1 & 0 & 1 & 1 & 0 & 0 \\
		0 & 0 & 0 & 1 & 1 & 0 & 0 & 0 & 0 \\
		0 & 0 & 0 & 0 & 1 & 0 & 0 & 1 & 1 \\
		0 & 1 & 0 & 0 & 0 & 0 & 1 & 0 & 1 \\
		0 & 0 & 0 & 0 & 0 & 0 & 1 & 1 & 0
	\end{bmatrix}
\]

\paragraph{(b)} Πίνακας βαθμών:
\[
	\begin{bmatrix}
		2 & 0 & 0 & 0 & 0 & 0 & 0 & 0 & 0 \\
		0 & 3 & 0 & 0 & 0 & 0 & 0 & 0 & 0 \\
		0 & 0 & 3 & 0 & 0 & 0 & 0 & 0 & 0 \\
		0 & 0 & 0 & 3 & 0 & 0 & 0 & 0 & 0 \\
		0 & 0 & 0 & 0 & 3 & 0 & 0 & 0 & 0 \\
		0 & 0 & 0 & 0 & 0 & 2 & 0 & 0 & 0 \\
		0 & 0 & 0 & 0 & 0 & 0 & 3 & 0 & 0 \\
		0 & 0 & 0 & 0 & 0 & 0 & 0 & 3 & 0 \\
		0 & 0 & 0 & 0 & 0 & 0 & 0 & 0 & 2
	\end{bmatrix}
\]

\paragraph{(c)} Πίνακας Laplace:
\[
	\begin{bmatrix}
		2 & -1 & -1 & 0 & 0 & 0 & 0 & 0 & 0 \\
		-1 & 3 & -1 & 0 & 0 & 0 & 0 & -1 & 0 \\
		-1 & -1 & 3 & -1 & 0 & 0 & 0 & 0 & 0 \\
		0 & 0 & -1 & 3 & -1 & -1 & 0 & 0 & 0 \\
		0 & 0 & 0 & -1 & 3 & -1 & -1 & 0 & 0 \\
		0 & 0 & 0 & -1 & -1 & 2 & 0 & 0 & 0 \\
		0 & 0 & 0 & 0 & -1 & 0 & 3 & -1 & -1 \\
		0 & -1 & 0 & 0 & 0 & 0 & -1 & 3 & -1 \\
		0 & 0 & 0 & 0 & 0 & 0 & -1 & -1 & 2
	\end{bmatrix}
\]

\subsection*{10.4.2 (MMDS)}

Χρησιμοποιώντας python βρίσκουμε ότι η δεύτερη μικρότερη ιδιοτιμή είναι $0.7$ και ο αντίστοιχος ιδιοπίνακας ο
\[[-0.33,-0.22,-0.33,-0.52,0.09,0,-0.36,-0.39,-0.41]\]
το οποίο και προτείνει τη διαμέριση $\{A,B,C,D,G,H,I\},\{E,F\}$.


\section*{Άσκηση 5}

\paragraph{(α)} Η τιμή της normalised cut για το γράφο του σχήματος 10.16 (MMDS) και τη διαμέριση $\{1,2,3\},\{4,5,6\}$ είναι
\[
	\frac{Cut(\{1,2,3\},\{4,5,6\})}{Vol(\{1,2,3\})} + \frac{Cut(\{1,2,3\},\{4,5,6\})}{Vol(\{4,5,6\})} = \frac25 + \frac25 = \frac45
\]

\paragraph{(β)} Για να βρούμε τις τιμές modularity για το γράφο $G$ του σχήματος 10.16 (MMDS) θα χρησιμοποιήσουμε τους βαθμούς για τον κάθε κόμβο, δηλαδή
\begin{center}
	\begin{tabular}{| c || c | c | c | c | c | c |}
		\hline
		κόμβοι & $1$ & $2$ & $3$ & $4$ & $5$ & $6$ \\ \hline
		βαθμός & $3$ & $2$ & $3$ & $3$ & $2$ & $3$ \\ \hline
	\end{tabular}
\end{center}
Επίσης έχουμε $m=8$. Οπότε για διαμέριση $S=\{ \{1,2,3\},\{4,5,6\} \}$ έχουμε
\begin{flalign*}
	Q(G,S) &= \left[ (3+2+3) - \frac{1}{2 \cdot 5}(3 \cdot 2 + 3 \cdot 3 + 2 \cdot 3) \right] + \left[ (3+2+3) - \frac{1}{2 \cdot 5}(3 \cdot 2 + 3 \cdot 3 + 2 \cdot 3) \right]\\
		&= \left( 8-\frac{21}{10} \right) + \left( 8-\frac{21}{10} \right) = 16 - 4.2 = 11.6
\end{flalign*}
Μετά για $S=\{ \{1\},\{2,3\},\{4,5\},\{6\} \}$ έχουμε
\[
	Q(G,S) = \left[ 3 - 0 \right] + \left[ (2+3) - \frac{2 \cdot 3}{2 \cdot 5} \right] + \left[ (3+2) - \frac{3 \cdot 2}{2 \cdot 5} \right] + \left[ 3 - 0 \right] = 16 - 1.2 = 14.8
\]


\section*{Άσκηση 6}

\subsection*{8.2.1 (MMDS)}

Θεωρούμε όπως και στο παράδειγμα της εκφώνησης το $n$ να είναι πόσες μέρες έχω πάει για σκι προηγουμένως.
Επίσης θεωρούμε $k$ τον αριθμό της μέρας που θα τα παρατήσω.

Αρχικά, θα πρέπει να δούμε πώς δουλεύει ο offline αλγόριθμος.
Με το $k$ γνωστό, αρκεί να δει αν $k \leq 10$ για να κάνει rent (cost $\$10k$), ενώ αν $k > 10$ κάνει buy (cost $\$100$).

Τώρα ο αλγόριθμός μας έχει μόνο το $n$ σαν πληροφορία, οπότε η στρατηγική του θα έχει τη μορφή "ξεκίνα με renting, και για κάποιο $m>0$ αν $n>m$ τότε άλλαξε σε buy", όπου η χειρότερη περίπτωση θα είναι να τα παρατήσω την ίδια μέρα $(k=m)$.
Οπότε online cost $=10m+100$.

Βλέπουμε τώρα για το competitive ratio:
\begin{itemize}
	\item αν $m \leq 10$:
		\[
			\frac{\text{offline}}{\text{online}} = \frac{10m}{10m+100} = \frac{m}{m+10} = 1 - \frac{10}{m+10}
		\]
		το οποίο γίνεται maximum στο $1/2$ για $m=10$
	\item αν $m \geq 10$:
		\[
			\frac{\text{offline}}{\text{online}} = \frac{100}{10m+100}
		\]
		το οποίο γίνεται πάλι maximum στο $1/2$ για $m=10$
\end{itemize}
Άρα ο αλγόριθμος που κάνει rent μέχρι τις $10$ μέρες και μετά buy έχει το μεγαλύτερο competitive ratio $=1/2$.

\subsection*{8.4.1 (MMDS)}

\begin{center}
	\begin{tabular}{| c || c | c | c |}
		\hline
		& $A$ & $B$ & $C$ \\ \hhline{|=#=|=|=|}
		bids & $x$ & $x,y$ & $x,y,z$ \\ \hline
		budget & $2$ & $2$ & $2$ \\ \hline
	\end{tabular}
\end{center}

\paragraph{(a)} Ξέρουμε ότι ο greedy αλγόριθμος θα αναθέσει τυχαία ένα query σε έναν advertiser αρκεί να έχουν budget.
Για input: $xxyyzz$, αρχικά βλέπουμε τα δύο πρώτα $x$ queries θα πάνε σίγουρα σε κάποιον, εφόσον όλοι τα θέλουν και έχουν όλοι αρκετό budget.
Μετά βλέπουμε ότι τα δύο επόμενα $y$ queries τα έχουν κάνει bid οι $B,C$ με budget $2$ o καθένας, οπότε όποιος και να έχει πάρει τα δύο προηγούμενα $x$ και πάλι θα βρεθεί διαθέσιμος advertiser για να γίνουν assign τα δύο $y$.

Οπότε ο greedy αλγόριθμος θα κάνει assign σίγουρα τουλάχιστον $4$ queries.

\paragraph{(b)} Θα εξετάσουμε το query: $xxzzzz$.

Ο optimal αλγόριθμος θα έκανε assign τα πρώτα δύο $x$ στον $A$, τα επόμενα δύο $z$ στον $C$ και τα τελευταία δύο $z$ σε κανέναν, οπότε συνολικός αριθμός assignments $=4$.

Ο greedy αλγόριθμος θα μπορούσε στη χειρότερη να κάνει assign τα δύο πρώτα $x$ στον $C$ με τα υπόλοιπα τέσσερα $z$ να μην γίνουν καθόλου, οπότε συνολικός αριθμός assignments $=2=4/2$.


\section*{Άσκηση 7}

\subsection*{9.2.3 (MMDS)}

\paragraph{(a)} Αρχικά έχουμε ότι ο μέσος όρος των ratings του χρήστη είναι $\frac{4+2+5}{3} = 11/3$.
Οπότε τα normalised ratings είναι
\begin{center}
	\begin{tabular}{| c || c | c | c |}
		\hline
		& $A$ & $B$ & $C$ \\ \hline
		user & $1/3$ & $-5/3$ & $4/3$ \\ \hline
	\end{tabular}
\end{center}

\paragraph{(b)} Το user profile του χρήστη θα αποτελείται από τις στήλες των υπολογιστών $A,B,C$, για τις οποίες έχουμε
\begin{flalign*}
	\text{Processor Speed:}& \quad \frac{3.06+2.68+2.92}{3} = \frac{88.6}{3} \approx 2.89\\
	\text{Disk Size:}& \quad \frac{500+320+640}{3} = \frac{1460}{3} \approx 486.67 \\
	\text{Main-Memory Size:}& \quad \frac{6+4+6}{3} = \frac{16}{3} \approx 5.33
\end{flalign*}
Οπότε το user profile του χρήστη είναι η γραμμή του παρακάτω πίνακα.
\begin{center}
	\begin{tabular}{| c || c | c | c |}
		\hline
		& $A$ & $B$ & $C$ \\ \hhline{|=#=|=|=|}
		user & $88.6/3$ & $1460/3$ & $16/3$ \\ \hline
	\end{tabular}
\end{center}

\subsection*{9.3.2 (MMDS)}

Παίρνουμε τον πίνακα 9.8 (MMDS) και αντικαθιστούμε τα $3,4,5$ με $1$ και τα $1,2$ και τα κενά με 0
\begin{center}
	\begin{tabular}{| c || c | c | c | c | c | c | c | c |}
		\hline
		& $a$ & $b$ & $c$ & $d$ & $e$ & $f$ & $g$ & $h$ \\ \hhline{|=#=|=|=|=|=|=|=|=|}
		$A$ & $1$ & $1$ & $0$ & $1$ & $0$ & $0$ & $1$ & $0$ \\ \hline
		$B$ & $0$ & $1$ & $1$ & $1$ & $0$ & $0$ & $0$ & $0$ \\ \hline
		$C$ & $0$ & $0$ & $0$ & $1$ & $0$ & $1$ & $1$ & $1$ \\ \hline
	\end{tabular}
\end{center}
και υπολογίζουμε τη Jaccard distance ($1-$ Jaccard similarity) για όλα τα ζευγάρια
\begin{center}
	\begin{tabular}{c | c c c c c c c c}
		& $a$ & $b$ & $c$ & $d$ & $e$ & $f$ & $g$ & $h$ \\ \hline
		$a$ & & $1/2$ & $1$ & $2/3$ & $1$ & $1$ & $1/2$ & $1$ \\
		$b$ & & & $1/2$ & $1/3$ & $1$ & $1$ & $2/3$ & $1$ \\
		$c$ & & & & $2/3$ & $1$ & $1$ & $1$ & $1$ \\
		$d$ & & & & & $1$ & $2/3$ & $1/3$ & $2/3$ \\
		$e$ & & & & & & $1$ & $1$ & $1$ \\
		$f$ & & & & & & & $1/2$ & $0$ \\
		$g$ & & & & & & & & $1/2$ \\
	\end{tabular}
\end{center}
όπου αξίζει να σημειώσουμε ότι οι τιμές $0,1/3,1/2,2/3,1$ είναι οι μόνες πιθανές τιμές για Jaccard distance μεταξύ συνόλων με τρία στοιχεία.

Θεωρούμε κάθε στήλη ένα cluster.
Προφανώς το πρώτο merge που θα κάνουμε είναι μεταξύ των $f,h$ που έχουν τη μικρότερη απόσταση, ενώ ταυτόχρονα επειδή οι δύο αυτές στήλες είναι ίδιες οι αποστάσεις μεταξύ των cluster παραμένουν ίδιες.
Συνεχίζουμε κάνοντας merge τις $b,d,g$ που έχουν την αμέσως μικρότερη απόσταση $(1/3)$.
Μπορούμε να κάνουμε και τις δύο διότι εφόσον οι $b,d,g$ δεν έχουν απόσταση $0$ με κάποια στήλη, τότε όποιες δύο και να κάνουμε πρώτα merge, οι νέες αποστάσεις δεν θα είναι καμία $0$ και το $1/3$ θα παραμείνει το αμέσως μικρότερο.

Με τα παραπάνω merges βλέπουμε τον ανανεωμένο πίνακα αποστάσεων
\begin{center}
	\begin{tabular}{c | c c c c c}
		& $a$ & $bdg$ & $c$ & $e$ & $fh$ \\ \hline
		$a$ & & $1/2$ & $1$ & $1$ & $1$ \\
		$bdg$ & & & $1/2$ & $1$ & $1/2$ \\
		$c$ & & & & $1$ & $1$ \\
		$e$ & & & & & $1$ \\
	\end{tabular}
\end{center}
οπότε το επόμενο merge μπορεί να είναι είτε το $a$ είτε το $c$ με το $bdg$.
Επιλέγουμε το $a$ και έτσι το τελικό clustering σε 4 στοιχεία είναι το
\[\{(abdg), (c), (e), (fh)\}\]

\paragraph{(b)} Υπολογίζουμε τις νέες τιμές για τα cluster profiles.
Συγκεκριμένα για το $(abdg)$ έχουμε
\[
	A: \frac{4+5+5+3}{4} = \frac{17}{4} \qquad B: \frac{3+3+1}{3} = \frac73 \qquad C: \frac{2+3+5}{3}=\frac{10}{3}
\]
ενώ για το $(fh)$ αρκεί να βρούμε για το $C: (4+3)/2 = 7/2$.
Οπότε ο νέος πίνακας με τις τιμές ανά cluster είναι
\begin{center}
	\begin{tabular}{| c || c | c | c | c |}
		\hline
		& $(abdg)$ & $(c)$ & $(e)$ & $(fg)$ \\ \hhline{|=#=|=|=|=|}
		$A$ & $17/4$ & & $1$ & $2$ \\ \hline
		$B$ & $7/3$ & $4$ & $1$ & $2$\\ \hline
		$C$ & $10/3$ & $1$ & & $7/2$ \\ \hline
	\end{tabular}
\end{center}

\paragraph{(c)}
\begin{flalign*}
	cos(\hat{AB}) &= 1 - \frac{\frac{17}{4} \cdot \frac73 + 0 + 1 + 2 \cdot 2}{\sqrt{\left(\frac{17}{4}\right)^2 + 1^2 + 2^2} \cdot \sqrt{\left(\frac73\right)^2 + 4^2 + 1^2 + 2^2}} \approx 1-0.6 = 0.4\\
	cos(\hat{AC}) &= 1 - \frac{\frac{17}{4} \cdot \frac{10}{3} + 0 + 0 + 2 \cdot \frac72}{\sqrt{\left(\frac{17}{4}\right)^2 + 1^2 + 2^2} \cdot \sqrt{\left(\frac{10}{3}\right)^2 + 1^2 + \left(\frac72\right)^2}} \approx 1-0.9 = 0.1 \\
	cos(\hat{BC}) &= 1 - \frac{\frac73 \cdot \frac{10}{3} + 4 + 0 + 2 \cdot \frac72}{\sqrt{\left(\frac73\right)^2 + 4^2 + 1^2 + 2^2} \cdot \sqrt{\left(\frac{10}{3}\right)^2 + 1^2 + \left(\frac72\right)^2}} \approx 1-0.74 = 0.26
\end{flalign*}

\end{document}
