% -*- mode: LaTeX; coding: utf-8 -*-
% Typeset with: XeLaTeX

\documentclass{beamer}
\mode<presentation>
{
  \usetheme[progressbar=foot,numbering=fraction,background=light]{metropolis} 
  \usecolortheme{default} % or try albatross, beaver, crane, ...
  \usefonttheme{default}  % or try serif, structurebold, ...
  \setbeamertemplate{navigation symbols}{}
  \setbeamertemplate{caption}[numbered]
  %\setbeamertemplate{frame footer}{My custom footer}
}
\centering

% Greek fonts
\usepackage{fontspec}
%\RequirePackage[cm-default]{fontspec}
\defaultfontfeatures{Mapping=tex-text}
  % you may want to try: {FreeSerif} or {Times New Roman}
%\setmainfont{Liberation Serif}
  % you may want to try: {FreeSans} or {Arial}
\setsansfont[Scale=MatchLowercase]{Liberation Sans}
  % you may want to try: {FreeMono} or {Courier New}
\setmonofont[Scale=MatchLowercase]{Liberation Mono}

\usepackage{graphicx}
\graphicspath{ {./figures/} }
\usepackage{hyperref,xcolor}
\newcommand{\link}[2]{\href{#1}{\textcolor{blue}{\underline{#2}}}}

% Main document
\begin{document}
\title{Κατηγορηματικός Λογισμός}
\subtitle{Μαθηματική Λογική}
\author{Θωμάς Παππάς}
\date{Παρασκευή 12 Νοεμβρίου 2020}
\maketitle

\begin{frame}{Agenda}
  \tableofcontents[hideallsubsections]
  %\tableofcontents
\end{frame}

%\AtBeginSection{}

\section{Αλφαβητικές Παραλλαγές}

\begin{frame}{Μεταθεώρημα Ύπαρξης Αλφαβητικών Παραλλαγών}
  \begin{block}{}
    Έστω τύπος $\phi$, όρος $t$ και μεταβλητή $x$.\\
    Τότε μπορούμε να βρούμε έναν τύπο $\phi^\prime$ που διαφέρει από τον $\phi$ μόνο στην επιλογή των ποσοδεικτών, τέτοιο ώστε
    \begin{itemize}
      \item[(a)] $\phi \vdash \phi^\prime$ και $\phi^\prime \vdash \phi$
      \item[(b)] $t$ είναι αντικαταστάσιμος στον $\phi^\prime$
    \end{itemize}
  \end{block}
\end{frame}

\begin{frame}{Μεταθεώρημα Ύπαρξης Αλφαβητικών Variants}
  \begin{block}{Απόδειξη}
  %\end{block}
  %\begin{block}{}
    Σταθεροποιούμε τα $t,x$ και κατασκευάζουμε τον $\phi^\prime$ αναδρομικά:
    \begin{itemize}
      \item Αν $\phi$ ατομικός τύπος, τότε $\phi^\prime = \phi$
      \item $(\neg\phi)^\prime = (\neg\phi^\prime)$
      \item $(\phi \rightarrow \psi)^\prime = (\phi^\prime \rightarrow \psi^\prime)$
    \end{itemize}
    Για την περίπτωση $(\forall y \; \phi)^\prime$ επιλέγουμε μεταβλητή $z$ που δεν εμφανίζεται στα $\phi^\prime, t, x$. Ορίζουμε $(\forall y \; \phi)^\prime = \forall z (\phi^\prime)_z^y$.\\
    Από επαγωγική υπόθεση έχουμε ότι $t$ αντικαταστάσιμος για τη $x$ στον $\phi^\prime$ και αφού $x \neq z$ τότε $t$ αντικαταστάσιμος για τη $x$ στον $(\phi^\prime)_z^y$. Άρα το (b) ισχύει.
  \end{block}
  \begin{block}{}
    Μένει ν.δ.ο. ισχύει και το (a) για την περίπτωση $(\forall y \; \phi)^\prime$, δηλαδή\\
    $\forall y \; \phi \vdash \forall z (\phi^\prime)_z^y$ και $\forall z (\phi^\prime)_z^y \vdash \forall y \; \phi$.
  \end{block}
\end{frame}

\begin{frame}{Μεταθεώρημα Ύπαρξης Αλφαβητικών Variants}
  \begin{block}{Απόδειξη (συν.)}
    Για το πρώτο έχουμε
    \begin{itemize}
      \item[1.] $\phi \vdash \phi^\prime$ από επαγωγική υπόθεση και άρα $\forall y \; \phi \vdash \forall y \; \phi^\prime$
      \item[2.] $\forall y \; \phi^\prime \vdash (\phi^\prime)_z^y$ από Αξ 2 αφού η $z$ δεν εμφανίζεται στον $\phi^\prime$
      \item[3.] $\forall y \; \phi^\prime \vdash \forall z (\phi^\prime)_z^y$ από το Μεταθεώρημα Γενίκευσης στον $2$
      \item[4.] $\forall y \; \phi \vdash \forall z (\phi^\prime)_z^y$ από $1,3$
    \end{itemize}
    Για το δεύτερο έχουμε
    \begin{itemize}
      \item[1.] $\forall z (\phi^\prime)_z^y \vdash ((\phi^\prime)_z^y)_y^z$ από Αξ 2 ενώ επίσης $((\phi^\prime)_z^y)_y^z = \phi^\prime$
      \item[2.] $\phi^\prime \vdash \phi$ από επαγωγική υπόθεση
      \item[3.] $\forall z (\phi^\prime)_z^y \vdash \phi$ από $1,2$
      \item[4.] $\forall z (\phi^\prime)_z^y \vdash \forall y \; \phi$ από Μεταθεώρημα Γενίκευσης στον $3$
    \end{itemize}
  \end{block}
\end{frame}

\section{Ισότητα}

\begin{frame}{Ισότητα}
  \begin{block}{}
    Εδώ κάποιες ισότητες που θα χρειαστούμε αργότερα:
    \begin{itemize}
      \item Eq1: $\vdash \forall x \; x = x$
      \item Eq2: $\vdash \forall x \forall y (x = y \rightarrow y = x)$
      \item Eq3: $\vdash \forall x \forall y \forall z (x = y \rightarrow y = z \rightarrow x = z)$
      \item Eq4: $\vdash \forall x_1 \forall x_2 \forall y_1 \forall y_2 (x_1 = y_1 \rightarrow x_2 = y_2 \rightarrow Px_1x_2 = Py_1y_2)$
        \begin{itemize}
          \item $P$ διμελές κατηγορηματικό σύμβολο
          \item ομοίως για $n$-μελή κατηγορηματικά σύμβολα
        \end{itemize}
      \item Eq5: $\vdash \forall x_1 \forall x_2 \forall y_1 \forall y_2 (x_1 = y_1 \rightarrow x_2 = y_2 \rightarrow fx_1x_2 = fy_1y_2)$
        \begin{itemize}
          \item $f$ διμελής συνάρτηση
          \item ομοίως για $n$-μελείς συναρτήσεις
        \end{itemize}
    \end{itemize}
  \end{block}
\end{frame}

\end{document}
