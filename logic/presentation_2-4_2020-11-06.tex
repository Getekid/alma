% -*- mode: LaTeX; coding: utf-8 -*-
% Typeset with: XeLaTeX

\documentclass{beamer}
\mode<presentation>
{
  \usetheme[progressbar=foot,numbering=fraction,background=light]{metropolis} 
  \usecolortheme{default} % or try albatross, beaver, crane, ...
  \usefonttheme{default}  % or try serif, structurebold, ...
  \setbeamertemplate{navigation symbols}{}
  \setbeamertemplate{caption}[numbered]
  %\setbeamertemplate{frame footer}{My custom footer}
}
\centering

% Greek fonts
\usepackage{fontspec}
%\RequirePackage[cm-default]{fontspec}
\defaultfontfeatures{Mapping=tex-text}
  % you may want to try: {FreeSerif} or {Times New Roman}
%\setmainfont{Liberation Serif}
  % you may want to try: {FreeSans} or {Arial}
\setsansfont[Scale=MatchLowercase]{Liberation Sans}
  % you may want to try: {FreeMono} or {Courier New}
\setmonofont[Scale=MatchLowercase]{Liberation Mono}

\usepackage{graphicx}
\graphicspath{ {./figures/} }
\usepackage{hyperref,xcolor}
\newcommand{\link}[2]{\href{#1}{\textcolor{blue}{\underline{#2}}}}
\usepackage{soul}

% Main document
\begin{document}
\title{Κατηγορηματικός Λογισμός}
\subtitle{Μαθηματική Λογική}
\author{Θωμάς Παππάς}
\date{Παρασκευή 6 Νοεμβρίου 2020}
\maketitle

\begin{frame}{Agenda}
  \tableofcontents[hideallsubsections]
  %\tableofcontents
\end{frame}

%\AtBeginSection{}

\section{Τυπικές αποδείξεις και Μεταθεωρήματα}

\subsection{Κανόνας $T$}

\begin{frame}{Κανόνας $T$}
  Αν $\Gamma \vdash \alpha_1, \ldots, \Gamma \vdash \alpha_n$ και $\{\alpha_1, \ldots, \alpha_n\} \models^{\textrm{T}\Sigma} \beta$, τότε $\Gamma \vdash \beta$.
  \begin{block}{Απόδειξη}
    Ο τύπος $\alpha_1 \rightarrow \cdots \rightarrow \alpha_n \rightarrow \beta$ είναι ταυτολογία και άρα Λογικό Αξίωμα.
    Εφαρμόζουμε Modus Ponens $n$ φορές.
  \end{block}
\end{frame}

\subsection{Μεταθεώρημα Συναγωγής}

\begin{frame}{Μεταθεώρημα Συναγωγής}
  Αν $\Gamma ; \gamma \vdash \phi$ τότε $\Gamma \vdash (\gamma \rightarrow \phi)$.\\
  \begin{small}Το αντίστροφο είναι το Modus Ponens\end{small}
  \begin{block}{1η Απόδειξη}
    Έχουμε ότι $\Gamma ; \gamma \vdash \phi$
    \begin{itemize}
      \item[] ανν $(\Gamma ; \gamma) \cup \Lambda \models^{\textrm{T}\Sigma} \phi$
      \item[] ανν $\Gamma \cup \Lambda \models^{\textrm{T}\Sigma} (\gamma \rightarrow \phi)$
      \item[] ανν $\Gamma \vdash (\gamma \rightarrow \phi)$
    \end{itemize}
  \end{block}
\end{frame}
\begin{frame}{Μεταθεώρημα Συναγωγής}
  Αν $\Gamma ; \gamma \vdash \phi$ τότε $\Gamma \vdash (\gamma \rightarrow \phi)$.
  \begin{block}{2η Απόδειξη}
    Θα δείξουμε με επαγωγή ότι για κάθε $\phi$ θεώρημα του $\Gamma ; \gamma$, ο τύπος $(\gamma \rightarrow \phi)$ είναι επίσης θεώρημα του $\Gamma$
    \begin{itemize}
      \item Περίπτωση 1: $\phi = \gamma$. Τότε προφανώς $\vdash (\gamma \rightarrow \phi)$
      \item Περίπτωση 2: $\phi$ λογικό αξίωμα ή μέλος του $\Gamma$.
        Τότε $\Gamma \vdash \phi$ και αφού $\phi \models^{\textrm{T}\Sigma} (\gamma \rightarrow \phi)$ τότε από τον κανόνα $T$ παίρνουμε $\Gamma \vdash (\gamma \rightarrow \phi)$.
      \item Περίπτωση 3: $\phi$ είναι από Modus Ponens στις $\psi$ και $(\psi \rightarrow \phi)$.\\
        Από επαγωγική υπόθεση $\Gamma \vdash (\gamma \rightarrow \psi)$ και $\Gamma \vdash (\gamma \rightarrow (\psi \rightarrow \phi))$.\\
        Τέλος, εφόσον $\{\gamma \rightarrow \psi, \gamma \rightarrow (\psi \rightarrow \phi)\} \models^{\textrm{T}\Sigma} \gamma \rightarrow \phi$ τότε από τον κανόνα $T$ παίρνουμε $\Gamma \vdash (\gamma \rightarrow \phi)$
    \end{itemize}
  \end{block}
\end{frame}

\subsection{Μεταθεώρημα Αντιθετοαναστροφής}

\begin{frame}{Μεταθεώρημα Αντιθετοαναστροφής}
  $\Gamma ; \phi \vdash \neg\psi$ ανν $\Gamma ; \psi \vdash \neg\phi$
  \begin{block}{Απόδειξη}
    \begin{itemize}
      \item[] $\Gamma ; \phi \vdash \neg\psi$ ανν $\Gamma \vdash \phi \rightarrow \neg\psi$ από το Μεταθεώρημα Συναγωγής
      \item[] ανν $\Gamma \vdash \psi \rightarrow \neg\phi$ από τον κανόνα $T$, αφού $\{\phi \rightarrow \neg\psi\} \models^{\textrm{T}\Sigma} (\psi \rightarrow \neg\phi)$
      \item[] ανν $\Gamma ; \psi \vdash \neg\phi$ από Modus Ponens
    \end{itemize}
  \end{block}
\end{frame}

\subsection{Μεταθεώρημα εις άτοπον απαγωγής}

\begin{frame}{Μεταθεώρημα Απαγωγής σε άτοπο}
  \begin{block}{Ορισμός}
    Ένα σύνολο από τύπους $\Sigma$ θα λέγεται \textit{αντιφατικό} ανν για κάποιο τύπο $\beta$  ισχύει ότι $\Sigma \vdash \beta$ και $\Sigma \vdash \neg\beta$
  \end{block}
  
  \begin{block}{Μεταθεώρημα Απαγωγής σε άτοπο}
    Αν $\Gamma ; \phi$ είναι αντιφατικό τότε $\Gamma \vdash \neg\phi$.
  \end{block}
  
  \begin{block}{Απόδειξη}
    Αφού $\Gamma ; \phi$ αντιφατικό τότε υπάρχει τύπος $\beta$ τέτοιος ώστε $\Sigma \vdash \beta$ και $\Sigma \vdash \neg\beta$.\\
    Απο Μεταθεώρημα Συναγωγής παίρνουμε $\Gamma \vdash (\phi \rightarrow \beta)$ και $\Gamma \vdash (\phi \rightarrow \neg\beta)$\\
    Επίσης $\{\phi \rightarrow \beta, \phi \rightarrow \neg\beta\} \models^{\textrm{T}\Sigma} \neg\phi$ και άρα από τον κανόνα $T$ παίρνουμε $\Gamma \vdash \neg\phi$.
  \end{block}
\end{frame}

\begin{frame}{Παράδειγμα}
  \begin{block}{}
    Θα δείξουμε ότι $\vdash \exists x \forall y \phi \rightarrow \forall y \exists x \phi$.
    \begin{itemize}
      \item[] από Μεταθεώρημα Συναγωγής, αρκεί ν.δ.ο. $\exists x \forall y \phi \vdash \forall y \exists x \phi$
      \item[] από Μεταθεώρημα Γενίκευσης, αρκεί ν.δ.ο. $\exists x \forall y \phi \vdash \exists x \phi$
      \item[] ισοδύναμα, αρκεί ν.δ.ο. $\neg\forall x \neg\forall y \phi \vdash \neg\forall x \neg\phi$
      \item[] από το Μεταθεώρημα Αντιθετοαναστροφής, αρκεί ν.δ.ο. $\forall x \neg\phi \vdash \forall x \neg\forall y \phi$
      \item[] από το Μεταθεώρημα Γενίκευσης, αρκεί ν.δ.ο. $\forall x \neg\phi \vdash \neg\forall y \phi$
      \item[] από το Μεταθεώρημα Απαγωγής σε άτοπο, αρκεί ν.δ.ο. το $\{\forall x \neg\phi, \forall y \phi\}$ είναι αντιφατικό
      \begin{itemize}
        \item $\forall x \neg\phi \vdash \neg\phi$ από το Αξίωμα 2 και Modus Ponens
        \item $\forall y \phi \vdash \phi$ για τον ίδιο λόγο
      \end{itemize}
    \end{itemize}
  \end{block}
\end{frame}

\section{Στρατηγικές Τυπικών Αποδείξεων και Παραδείγματα}

\begin{frame}{Στρατηγικές Τυπικών Αποδείξεων}
  \begin{block}{}
    Θα δούμε στρατηγικές για να κατασκευάζουμε τυπικές αποδείξεις προτάσεων.
  \end{block}
  \begin{block}{}
    Αν ένας τύπος $\phi$ είναι της μορφής
    \begin{itemize}
      \item $(\psi \rightarrow \theta)$, τότε αρκεί ν.δ.ο. $\Gamma ; \psi \vdash \theta$
      \item $\forall x \psi$
        \begin{itemize}
          \item Αν η $x$ δεν εμφανίζεται ελεύθερη στον $\psi$ τότε αρκεί ν.δ.ο. $\Gamma \vdash \psi$
          \item Διαφορετικά, θα υπάρχει πάντα μια μεταβλητή $y$ τέτοια ώστε $\Gamma \vdash \forall y \psi_y^x$ και $\forall y \psi_y^x \vdash \forall x \psi$
        \end{itemize}
      \item άρνησης ενός τύπου της μορφής
        \begin{itemize}
          \item $\neg(\psi \rightarrow \theta)$, τότε αρκεί ν.δ.ο. $\Gamma \vdash \psi$ και $\Gamma \vdash \neg\theta$
          \item $\neg\neg\psi$, τότε αρκεί ν.δ.ο. $\Gamma \vdash \psi$
          \item $\neg\forall x \; \psi$, τότε αρκεί ν.δ.ο. $\Gamma \vdash \neg\psi_t^x$ όπου $t$ όρος.
            \begin{itemize}
              \item Όχι πάντα δυνατό, πχ. $\Gamma = \emptyset, \psi = \neg(Px \rightarrow \forall y Py)$.\\
              (Άρα $\phi = \neg\forall x \; \psi = \exists x (Px \rightarrow \forall y Py)$)
            \end{itemize}
        \end{itemize}
    \end{itemize}
  \end{block}
\end{frame}

\begin{frame}{Αντιπαράδειγμα $\neg\forall x \psi$ (Drinker's principle)}
  \begin{block}{}
  Δείξτε ότι
  \[
    \vdash \exists x (Px \rightarrow \forall y Py)
  \]
  \end{block}
  \begin{block}{Απόδειξη (ελλειπής)}
    \begin{enumerate}
      \item $\forall x Px \vdash \exists x (Px \rightarrow \forall y Py)$, \st{$T$ αφού $\forall x Px \models^{\textrm{T}\Sigma} \exists x (Px \rightarrow \forall y Py)$}
      \item $\neg\forall x Px \vdash \exists x (Px \rightarrow \forall y Py)$, \st{$T$ αφού $\neg\forall x Px \models^{\textrm{T}\Sigma} \exists x (Px \rightarrow \forall y Py)$}
      \item $\neg\exists x (Px \rightarrow \forall y Py) \vdash \neg\forall x Px$ (Μεταθεώρημα Αντ. στο $1$)
      \item $\neg\exists x (Px \rightarrow \forall y Py) \vdash \forall x Px$ (Μεταθεώρημα Αντ. στο $2$)
    \end{enumerate}
    Άρα $\{\neg\exists x (Px \rightarrow \forall y Py)\}$ αντιφατικό λόγω $3,4$ και από το Μεταθεώρημα της Απαγωγής σε άτοπο παίρνουμε το ζητούμενο.
  \end{block}
\end{frame}

\begin{frame}{Αντιπαράδειγμα $\neg\forall x \psi$ (Drinker's principle)}
  Τώρα θα δείξουμε ότι για κάθε όρο $t$
  \[
    \nvdash (Px \rightarrow \forall y Py)^x_t
  \]
  \begin{block}{Απόδειξη (μη φορμαλιστική)}
    Έστω για έναν τυχαίο όρο $t$ ότι $\vdash (Px \rightarrow \forall y Py)_t^x$. Βλέπουμε ότι μπορούμε να θεωρήσουμε μια αυθαίρετη δομή $\mathfrak{A}$ (πχ. όπου $\neg Pt$ για κάθε $t$) στην οποία και θα έχουμε $\not\models_\mathfrak{A} (Px \rightarrow \forall y Py)_t^x$.\\
    Άρα έχουμε ότι $\not\models (Px \rightarrow \forall y Py)_t^x$ και από το Μεταθεώρημα Αξιοπιστίας παίρνουμε το ζητούμενο.
  \end{block}
\end{frame}

\begin{frame}{Παράδειγμα 1}
  \begin{block}{}
    Αν η $x$ δεν εμφανίζεται ελεύθερη στον $\alpha$ τότε
    \[
      \vdash (\alpha \rightarrow \forall x \beta) \leftrightarrow \forall x (\alpha \rightarrow \beta)
    \]
    Για να το αποδείξω αρκεί ν.δ.ο. (από κανόνα $T$)
    \[
      \vdash (\alpha \rightarrow \forall x \beta) \rightarrow \forall x (\alpha \rightarrow \beta)
    \]
    και
    \[
      \vdash \forall x (\alpha \rightarrow \beta) \rightarrow (\alpha \rightarrow \forall x \beta)
    \]
  \end{block}
\end{frame}

\begin{frame}{Παράδειγμα 1}
  \begin{block}{}
    \[
      \vdash (\alpha \rightarrow \forall x \beta) \rightarrow \forall x (\alpha \rightarrow \beta)
    \]
    Για την πρώτη περίπτωση, αρκεί ν.δ.ο.
    \begin{itemize}
      \item $\{\alpha \rightarrow \forall x \beta\} \vdash \forall x (\alpha \rightarrow \beta)$ (Μεταθεώρημα Συναγωγής)
      \item $\{\alpha \rightarrow \forall x \beta\} \vdash (\alpha \rightarrow \beta)$ (Μεταθεώρημα Γενίκευσης)
      \item $\{\alpha \rightarrow \forall x \beta, \alpha\} \vdash \beta$ (Μεταθεώρημα Συναγωγής)
    \end{itemize}
    Το οποίο ισχύει αφού $\{\alpha \rightarrow \forall x \beta, \alpha\} \vdash \forall x \beta$ (MP)\\
    και επίσης $(\forall x \beta \rightarrow \beta)$ είναι αξίωμα.
  \end{block}
\end{frame}

\begin{frame}{Παράδειγμα 1}
  \begin{block}{}
    \[
      \vdash \forall x (\alpha \rightarrow \beta) \rightarrow (\alpha \rightarrow \forall x \beta)
    \]
    Για τη δεύτερη περίπτωση, αρκεί ν.δ.ο.
    \begin{itemize}
      \item $\{\forall x (\alpha \rightarrow \beta)\} \vdash (\alpha \rightarrow \forall x \beta)$ (Μεταθεώρημα Συναγωγής)
      \item $\{\forall x (\alpha \rightarrow \beta), \alpha\} \vdash \forall x \beta$ (Μεταθεώρημα Συναγωγής)
      \item $\{\forall x (\alpha \rightarrow \beta), \alpha\} \vdash \beta$ (Μεταθεώρημα Γενίκευσης)
    \end{itemize}
    το οποίο ισχύει αφού $\forall x (\alpha \rightarrow \beta) \rightarrow (\alpha \rightarrow \beta)$ λογικό αξίωμα και άρα
    \[
      \{\forall x (\alpha \rightarrow \beta), \alpha\} \vdash (\alpha \rightarrow \beta)
    \]
    και με άλλο ένα Modus Ponens παίρνουμε το ζητούμενο.
  \end{block}
\end{frame}

\begin{frame}{Συντομογραφίες (συμβολισμός)}
  \begin{block}{}
    Οι τύποι θα τοποθετούνται σε μια αριθμημένη λίστα, ο καθένας με μια αιτιολογήση ως προς το πώς προέκυψε
    \begin{itemize}
      \item Αξ 1: Λογικό Αξίωμα 1
      \item $1,2$·$T$: κανόνας $T$ στους τύπους των γραμμών $1,2$
      \item $1,2$·MP: Μodus Ponens στους τύπους των γραμμών $1,2$
      \item $1$·γεν: Μεταθεώρημα Γενίκευσης στον τύπο της γραμμής $1$
      \item $1$·γεν$^2$: Μεταθεώρημα Γενίκευσης 2 φορές στον τύπο της γραμμής $1$
      \item $1$·συναγ: Μεταθεώρημα Συναγωγής στον τύπο της γραμμής $1$
      \item $1$·ΑΣΑ: Μεταθεώρημα Απαγωγής Σε Άτοπο στον τύπο της γραμμής $1$
    \end{itemize}
  \end{block}
\end{frame}

\begin{frame}{Παράδειγμα 2}
  \begin{block}{}
  Δείξτε ότι
    \[
      \forall x \forall y (x = y \rightarrow y = x)
    \]
  \end{block}
  \begin{block}{Απόδειξη}
    \begin{enumerate}
      \item $\vdash x = y \rightarrow x = x \rightarrow y = x$, Αξ 6
      \item $\vdash x = x$, Αξ 5
      \item $\vdash x = y \rightarrow y = x$, $1,2;T$
      \item $\vdash \forall x \forall y (x = y \rightarrow y = x)$, $3;$γεν$^2$
    \end{enumerate}
  \end{block}
\end{frame}

\begin{frame}{Παράδειγμα 3}
  \begin{block}{}
  Δείξτε ότι
    \[
      \vdash x = y \rightarrow \forall z \; Pxz \rightarrow \forall z \; Pyz
    \]
  \end{block}
  \begin{block}{Απόδειξη}
    \begin{enumerate}
      \item $\vdash x = y \rightarrow (Pxz \rightarrow Pyz)$, Αξ 6
      \item $\vdash \forall z \; Pxz \rightarrow Pxz$, Αξ 2
      \item $\vdash x = y \rightarrow (\forall z \; Pxz \rightarrow Pyz)$, $1,2;T$
      \item $\{x = y, \forall z \; Pxz\} \vdash Pyz$, $3$·MP$^2$
      \item $\{x = y, \forall z \; Pxz\} \vdash \forall z Pyz$, $4$·γεν$^2$
      \item $\vdash x = y \rightarrow \forall z \; Pxz \rightarrow \forall z \; Pyz$, $5$·συναγ$^2$
    \end{enumerate}
  \end{block}
\end{frame}

\begin{frame}{Παράδειγμα 4}
  \begin{block}{}
  Έστω $f$ μια διμελής συνάρτηση. Τότε
    \[
      \vdash \forall x_1 \forall x_2 \forall y_1 \forall y_2 (x_1 = y_1 \rightarrow x_2 = y_2 \rightarrow fx_1x_2 = fy_1y_2)
    \]
  \end{block}
  \begin{block}{Απόδειξη}
    \begin{enumerate}
      \item $\vdash x_1 = y_1 \rightarrow fx_1x_2 = fx_1x_2 \rightarrow fx_1x_2 = fy_1x_2$, Αξ 6
      \item $\vdash x_2 = y_2 \rightarrow fx_1x_2 = fy_1x_2 \rightarrow fx_1x_2 = fy_1y_2$, Αξ 6
      \item $\vdash fx_1x_2 = fx_1x_2$, Αξ 5
      \item $\vdash (x_1 = y_1 \rightarrow x_2 = y_2 \rightarrow fx_1x_2 = fy_1y_2)$, $1,2,3$·$T$
      \item $\vdash \forall x_1 \forall x_2 \forall y_1 \forall y_2 (x_1 = y_1 \rightarrow x_2 = y_2 \rightarrow fx_1x_2 = fy_1y_2)$, $1,2,3$·γεν$^4$
    \end{enumerate}
  \end{block}
\end{frame}

\begin{frame}{Μεταθεώρημα Γενίκευσης Σταθερών}
  \begin{block}{}
    Έστω $\Gamma \vdash \phi$ και $c$ μια σταθερά που δεν εμφανίζεται στο $\Gamma$. Τότε υπάρχει μεταβλητή $y$ που δεν εμφανίζεται στον $\phi$ τέτοια ώστε $\Gamma \vdash \forall y \; \phi_y^c$.\\
    Επιπλέον, υπάρχει τυπική απόδειξη του $\forall y \; \phi_y^c$ από το $\Gamma$ στην οποία η $c$ δεν εμφανίζεται.
  \end{block}
  \begin{block}{Απόδειξη}
    Έστω $\langle \alpha_0, \ldots, \alpha_n \rangle$ τυπική απόδειξη του $\phi$ από το $\Gamma$. (άρα $\alpha_n= \phi$)\\
    Έστω $y$ μεταβλητή που δεν εμφανίζεται σε κανένα από τα $\alpha_i$.\\
    Ισχυριζόμαστε ότι η
    \begin{equation}
      \label{q-def}
      \langle(\alpha_0)_y^c, \ldots, (\alpha_n)_y^c \rangle
    \end{equation}
    είναι τυπική απόδειξη της $\phi_y^c$ στο $\Gamma$
  \end{block}
\end{frame}

\begin{frame}{Μεταθεώρημα Γενίκευσης Σταθερών}
  \begin{block}{Απόδειξη (συν.)}
  Θα χρησιμοιήσουμε επαγωγή
  \begin{itemize}
    \item Περίπτωση 1: $\alpha_k \in \Gamma$. Τότε $(\alpha_k)_y^c = \alpha_k$
    \item Περίπτωση 2: $\alpha_k$ Λογικό Αξίωμα. Τότε και $(\alpha_k)_y^c$ Λογικό Αξίωμα
    \item Περίπτωση 3: $\alpha_k$ είναι από MP στις $\alpha_i$ και $\alpha_j$, για $i,j \leq k$\\
      και όπου $\alpha_j = (\alpha_i \rightarrow \alpha_k)$.\\
      Βλέπουμε ότι $(\alpha_j)_y^c = (\alpha_i \rightarrow \alpha_k)_y^c = ((\alpha_i)_y^c \rightarrow (\alpha_k)_y^c)$.\\
      Η $(\alpha_k)_y^c$ προκύπτει από MP στις $(\alpha_i)_y^c$ και $(\alpha_j)_y^c$.
  \end{itemize}
  Άρα η \eqref{q-def} είναι τυπική απόδειξη της $\phi_y^c$.
  \end{block}
\end{frame}

\begin{frame}{Μεταθεώρημα Γενίκευσης Σταθερών}
  \begin{block}{Απόδειξη (συν.)}
  \end{block}
  \begin{block}{}
    Έστω $\Phi$ το (πεπερασμένο) υποσύνολο του $\Gamma$ το οποίο χρησιμοποιήθηκε στην \eqref{q-def}. Τότε η \eqref{q-def} είναι τυπική απόδειξη της $\phi_y^c$ από το $\Phi$ στο οποίο η $y$ δεν εμφανίζεται. Από το Μεταθεώρημα Γενίκευσης παίρνουμε $\Phi \vdash \forall y \; \phi_y^c$.
  \end{block}
  \begin{block}{}
  Επιπλέον, υπάρχει τυπική απόδειξη της $\forall y \; \phi_y^c$ από το $\Phi$ όπου η $c$ δεν εμφανίζεται, διότι στην απόδειξη του Μεταθεωρήματος Γενίκευσης δεν εισάγαμε κάποιο νέο σύμβολο.
  \end{block}
\end{frame}

\begin{frame}{Πόρισμα 1}
  \begin{block}{}
    Έστω $\Gamma \vdash \phi_c^x$ όπου η $c$ δεν εμφανίζεται στο $\Gamma$ ή στον $\phi$.\\
    Τότε $\Gamma \vdash \forall x \; \phi$ και υπάρχει τυπική απόδειξη από το $\Gamma$ όπου η $c$ δεν εμφανίζεται.
  \end{block}
  \begin{block}{Απόδειξη}
    Εφόσον $\Gamma \vdash \phi_c^x$, από το Μεταθεώρημα Γενίκευσης Σταθερών παίρνουμε $\Gamma \vdash \forall y \; (\phi_c^x)_y^c$. Αλλά αφού η $c$ δεν εμφανίζεται στον $\phi$ τότε
    \[
      (\phi_c^x)_y^c = \phi_y^x
    \]
    Άρα αρκεί ν.δ.ο. $\forall y \; \phi_y^x \vdash \forall x \; \phi$.\\
    Από το Λογικό Αξίωμα 3 παίρνουμε $\forall y \; \phi_y^x \rightarrow (\phi_y^x)_x^y$.\\
    Όμως $(\phi_y^x)_x^y = \phi$ και άρα παίρνουμε το ζητούμενο.
  \end{block}
\end{frame}

\begin{frame}{Πόρισμα 2 (Κανόνας EI)}
  \begin{block}{}
    Έστω σταθερά $c$ που δεν εμφανίζεται στους $\phi, \psi$ ή στο $\Gamma$ και ότι $\Gamma ; \phi_c^x \vdash \psi$.\\
    Τότε $\Gamma ; \exists x \; \phi \vdash \psi$ και υπάρχει τυπική απόδειξη της $\psi$ από το $\Gamma$ όπου η $c$ δεν εμφανίζεται.
  \end{block}
  \begin{block}{Απόδειξη}
    Από το Μεταθεώρημα Αντιθετοαναστροφής παίρνουμε $\Gamma ; \neg\psi \vdash \neg\phi_c^x$.\\
    Από το Πόρισμα 1 παίρνουμε $\Gamma ; \neg\psi \vdash \forall x \; \neg\phi$\\
    Από το Μεταθεώρημα Αντιθετοαναστροφής παίρνουμε $\Gamma ; \neg(\forall x \; \neg\phi) \vdash \psi$ από το οποίο και καταλήγουμε στο $\Gamma ; \exists x \; \phi \vdash \psi$.
  \end{block}
\end{frame}

\begin{frame}{Παράδειγμα 5}
  \begin{block}{}
    Δείξτε ότι
    \[
      \vdash \exists x \forall y \; \phi \rightarrow \forall y \exists x \; \phi
    \]
  \end{block}
  \begin{block}{Απόδειξη}
    Αρκεί ν.δ.ο.
    \begin{itemize}
      \item $\exists x \forall y \; \phi \vdash \forall y \exists x \; \phi$ (Μεταθεώρημα Συναγωγής)
      \item $\forall y \; \phi_c^x \vdash \forall y \exists x \; \phi$ (Κανόνας EI)
      \item $\forall y \; \phi_c^x \vdash \exists x \; \phi$ (Μεταθεώρημα Γενίκευσης)
      \item $\phi_c^x \vdash \exists x \; \phi$ (εφόσον $\forall y \; \phi_c^x \vdash \phi_c^x$)
      \item $\forall x \; \neg\phi \vdash \neg\phi_c^x$ (Μεταθεώρημα Αντιθετοαναστροφής)
    \end{itemize}
    το οποίο και προκύπτει από Modus Ponens στον $(\forall x \; \neg\phi \rightarrow (\neg\phi)_c^x)$\\
    (Λογικό Αξίωμα 2).
  \end{block}
\end{frame}

\end{document}
